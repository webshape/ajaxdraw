\input{../TeX/base} %BASE!!!
\usepackage{multirow}
\begin{document}

\renewcommand{\insertversion}{0.4} %INSERIRE LA VERSIONE QUI DENTRO STILE x.x.xx
\renewcommand{\TITOLODOC}{Decomposizione del Capitolato in Aspettative} %INSERIRE IL TITOLO DEL DOCUMENTO DA FAR COMPARIRE A PIE PAGINA

\begin{titlepage}
\begin{center}
	\begin{Large}	\today \end{Large}
\end{center}

\vspace{20pt}

\begin{center}
	\begin{Huge}
				\textbf{AJAXDRAW}
	\end{Huge}
\end{center}			

\begin{center}
	\begin{large}
				\textbf{Software per il Disegno Grafico\\ in Tecnologie Web}
	\end{large}
\end{center}			

\vspace{20pt}

\begin{center}
\includegraphics[width=150pt]{logo}
\end{center}

\vspace{160pt}
\begin{center} %INSERIRE ALL'INTERNO IL TITOLO DOCUMENTO CHE COMPARIRA NELLA PAGINA INIZIALE				
	\begin{Huge}
				\textbf{\TITOLODOC}
	\end{Huge}
			\\
\end{center}
\vspace{100pt}
\begin{center}
Versione: \insertversion
\end{center}
\end{titlepage}

\newpage

\sezione{Decomposizione del Capitolato in Aspettative}

\subsezione{Obbligatorie (ACO)}
\begin{elenconumerato}[\textbf{ACO-}]{\subsecindent}
\item Il sistema AJAXDRAW dovr\`a avere caratteristiche atte a poter elaborare figure/linee/testi vettoriali di vario genere.
\item Il sistema sar\`a in grado di fornire all'utente le seguenti funzioni:

\begin{list}{-}{}
\item selezione
\item linea retta/curva di Bezier
\item zoom
\item quadrato/rettangolo
\item cerchio/elisse
\item poligoni
\item testo
\end{list}

\item Il sistema permetter\`a all'utente di:

\begin{list}{-}{}
\item cambiare il colore di riempimento
\item cambiare il colore del bordo
\item cambiare il colore della linea
\item cambiare l'opacit\`a
\item cambiare il posizionamento all'interno dell'area di disegno
\item ridimensionare gli oggetti
\end{list}

\item Il sistema sar\`a in grado di leggere e salvare il contenuto dell'area di disegno in formato SVG.
\item Il sistema sar\`a implementato seguendo gli standard del linguaggio di markup HTML 5. Il software dovr\`a quindi essere in grado di funzionare perfettamente in quei browser che implementano gli standard HTML 5 (Firefox, Safari, Opera e Chrome). 
\item Il software utilizzer\`a il tag "canvas" appositamente creato per l'elaborazione del disegno vettoriale su web.
\item Il software AJAXDRAW deve fornire la possibilit\`a di attivare un aiuto (manuale), per l'utilizzazione del sistema. 
\item Il sistema AJAXDRAW deve fornire un manuale per chiunque voglia estendere l'applicazione. 
\end{elenconumerato}

\subsezione{Altamente Desiderabili (ACD)}
\begin{elenconumerato}[\textbf{ACD-}]{\subsecindent}
\item Il sistema prevede la pubblicazione sul sito "www.sourceforge.net", in conformit\`a con i relativi requisiti di natura open-source, per favorire la continuit\`a del prodotto risultante.
\end{elenconumerato}

\subsezione{Facoltative (ACF)}
\begin{elenconumerato}[\textbf{ACF-}]{\subsecindent}
\item Il sistema AJAXDRAW prevede di avvicinarsi il pi\`u possibile al programma client (open e free) "Inkscape". 
\item Il sistema AJAXDRAW prevede di implementare le seguenti funzioni:

\begin{list}{-}{}
\item disegno a mano libera
\item spirali
\item disegno calligrafico
\item connettori per diagrammi
\end{list}

\item Il sistema AJAXDRAW prevede la compatibilit\`a con i browser che non implementano a pieno HTML 5 (p. es. Internet Explorer). Usufruendo della libreria "excanvas", open source disponibile su googlecode, creata appositamente per la simulazione degli oggetti canvas, sar\`a possibile far funzionare il disegnatore web su questi browser. 
\end{elenconumerato}

\end{document}
