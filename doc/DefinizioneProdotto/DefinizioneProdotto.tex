	\input{../TeX/base} %BASE!!!

\title{\TITOLODOC}
\author{Piero Bizzotto}

\begin{document}

\renewcommand{\insertversion}{0.0} %INSERIRE LA VERSIONE QUI DENTRO STILE x.x.xx
\renewcommand{\TITOLODOC}{Definizione di Prodotto} %INSERIRE IL TITOLO DEL DOCUMENTO DA FAR COMPARIRE A PIE PAGINA
\renewcommand{\glosspath}{.\glossario} %INSERIRE PERCORSO RELATIVO

%%%%%%%%%%%%%%%%%%%%%%PARTE DA NON MODIFICARE%%%%%%%%%%%%%%%%%
\begin{titlepage}
\begin{center}
	\begin{Large}	\today \end{Large}
\end{center}

\vspace{20pt}

\begin{center}
	\begin{Huge}
				\textbf{\ajax}
	\end{Huge}
\end{center}			

\begin{center}
	\begin{large}
				\textbf{Software per il Disegno Grafico in Tecnologie Web}
	\end{large}
\end{center}			

\vspace{20pt}

\begin{center}
\includegraphics[width=150pt]{../logo/logo}
\end{center}

\vspace{170pt}
\begin{center} %INSERIRE ALL'INTERNO IL TITOLO DOCUMENTO CHE COMPARIRA NELLA PAGINA INIZIALE				
	\begin{Huge}
				\textbf{\TITOLODOC}
	\end{Huge}
			\\
\end{center}
\vspace{210pt}
\begin{center}
Versione: \insertversion
\end{center}
\end{titlepage}

\newpage
%%%%%%%%%%%%%%%%%%%%%%FINE PARTE DA NON MODIFICARE%%%%%%%%%%%%%%%%%

\begin{center} %INSERIRE ALL'INTERNO IL TITOLO DOCUMENTO CHE COMPARIRA NELLA PAGINA INIZIALE
	\begin{Huge}	
				\textbf{\TITOLODOC}
			\\
	\end{Huge}
\end{center}

%\setlength{\parindent}{18pt} %settato indentazione di default 
\section*{\Large Sommario:} %SEZIONE SOMMARIO
\indent \indent
Il presente documento descrive in modo dettagliato le classi e i relativi metodi che andranno a comporre il sistema.

\section*{\Large Stato del documento:}
\indent \indent
	Formale Esterno

\section*{\Large Redazione:}
	\begin{elencopuntato}[\normindent]
		\item[-] Redattore N. 1
		\item[-] Redattore N. 2
		\item[-] Redattore N. 3
	\end{elencopuntato}

\section*{\Large Approvazione:}
	\begin{elencopuntato}[\normindent]
		\item Approvatore N. 1
		\item Approvatore N. 2
		\item Approvatore N. 3
	\end{elencopuntato}

\section*{\LARGE Lista di Distribuzione:}

	\begin{elenconumerato}{\normindent}
		\item WebShape \footnote{Il termine WebShape designa una collettivit\`a di individui come da organigramma contenuto nel piano di progetto fornito in allegato al presente documento}
		\item I committenti Vardanega Tullio e Conte Renato in rappresentanza \\  dell'azienda proponente Zucchetti SPA
		%\item Il committente Conte Renato
	\end{elenconumerato}

\newpage



\section*{\Large Registro delle Modifiche:}


\begin{center}
	\begin{table}[h]
		  \begin{tabular*}
			{1\textwidth}%
				{@{\extracolsep{\fill}}|p{0.1\textwidth}|p{0.54\textwidth}|p{0.26\textwidth}|}
			 \hline
%%%%%%%%%%%%%%INTESTAZIONE COLONNE%%%%%%%%%%%%%%%%%%%%%%%%%%%%%%%%%%%%%%%%%%%%%%%%%%%%%%%%%%%%%%
			\textbf{Versione}  & \textbf{Descrizione} & \textbf{Autore} \\
%%%%%%%%%%%%%%FINE INTESTAZIONE COLONNE%%%%%%%%%%%%%%%%%%%%%%%%%%%%%%%%%%%%%%%%%%%%%%%%%%%%%%%%%%%%%%
		 \hline
%%%%%%%%%%% PARTE DA MODIFICARE %%%%%%%%%%%%%%%%%%%%%%%%%%%%%%%%%%%%%%%%%%%%%%%%%%%%%%%%%%%%%%%%%		
    	 	0.0 & 	 11$\slash$02$\slash$2009 Bozza & Dissegna Stefano \\

		\hline %%FINE RIGA
%%%%%%%%%%% FINE PARTE DA MODIFICARE %%%%%%%%%%%%%%%%%%%%%%%%%%%%%%%%%%%%%%%%%%%%%%%%%%%%%%%%%%%
		\end{tabular*}
	\caption{didascalia tabella 	MODIFICHE} %INSERIRE DIDASCALIA - SE NECESSARIA - 
	\label{tab:modifiche}
	\end{table}
\end{center}


\newpage
\thispagestyle{fancy}
\tableofcontents
\thispagestyle{fancy}
\newpage

%qui inizia l'indice, che viene creato automaticamente

\sezione{Introduzione}

\subsezione{Scopo del documento}
Il documento descrive a livello di dettaglio l'architettura del sistema.

\subsezione{Scopo del prodotto}
Il software AJAXDRAW \`e proposto per verificare e dimostrare la fattibilit\`a di realizzazione di un'applicazione di disegno grafico, in grado di poter elaborare figure vettoriali primitive e complesse utilizzando le tecnologie web.

\subsezione{Glossario}
All'interno del documento \textit{Glossario}, sono presentati i termini tecnici utilizzati in tutti i documenti. Il glossario \`e fornito in allegato al presente documento.
\subsezione{Riferimenti}
Il presente documento \`e redatto utilizzando le convenzioni inserite nel documento \textit{NormeDiProgetto}, consultabile dal repository pubblico al quale WebShape si appoggia per i suoi progetti.
\subsubsezione{Normativi}
\begin{elencopuntato}[\subsubsecindent]
\item[-] \textit{NormeDiProgetto.pdf}
\item[-] Raccomandazione SVG: http://www.w3.org/TR/SVG11/
\item[-] Raccomandazione CSS: http://www.w3.org/TR/CSS21/
\item[-] Working Draft HTML 5: http://www.w3.org/TR/html5/
\item[-] Specifiche UML: http://www.omg.org/spec/UML/2.1.2/
\end{elencopuntato}

\sezione{Standard di progetto}
\subsezione{Standard di progettazione architetturale}
La progettazione verr\`a presentata tramite una descrizione testuale delle componenti accompagnata da diagrammi UML.
\subsezione{Standard di documentazione del codice}
Si vedano le \textit{NormeDiprogetto}.
\subsezione{Standard di denominazione di entità e relazioni}
Si vedano le \textit{NormeDiprogetto}.
\subsezione{Standard di programmazione}
Si vedano le \textit{NormeDiprogetto}.
\subsezione{Strumenti di lavoro}
Si vedano le \textit{NormeDiprogetto}.

\sezione{Specifica delle componenti}

\sezione{Tracciamento della relazione componenti - requisiti}

\end{document}
