	\input{../TeX/base} %BASE!!!

\title{\TITOLODOC}
\author{Stefano Dissegna}

\begin{document}

\renewcommand{\insertversion}{0.0} %INSERIRE LA VERSIONE QUI DENTRO STILE x.x.xx
\renewcommand{\TITOLODOC}{Definizione di Prodotto} %INSERIRE IL TITOLO DEL DOCUMENTO DA FAR COMPARIRE A PIE PAGINA
\renewcommand{\glosspath}{.\glossario} %INSERIRE PERCORSO RELATIVO

%%%%%%%%%%%%%%%%%%%%%%PARTE DA NON MODIFICARE%%%%%%%%%%%%%%%%%
\begin{titlepage}
\begin{center}
	\begin{Large}	\today \end{Large}
\end{center}

\vspace{20pt}

\begin{center}
	\begin{Huge}
				\textbf{\ajax}
	\end{Huge}
\end{center}			

\begin{center}
	\begin{large}
				\textbf{Software per il Disegno Grafico\\ in Tecnologie Web}
	\end{large}
\end{center}			

\vspace{20pt}

\begin{center}
\includegraphics[width=150pt]{../logo/logo}
\end{center}

\vspace{170pt}
\begin{center} %INSERIRE ALL'INTERNO IL TITOLO DOCUMENTO CHE COMPARIRA NELLA PAGINA INIZIALE				
	\begin{Huge}
				\textbf{\TITOLODOC}
	\end{Huge}
			\\
\end{center}
\vspace{190pt}
\begin{center}
Versione: \insertversion
\end{center}
\end{titlepage}

\newpage
%%%%%%%%%%%%%%%%%%%%%%FINE PARTE DA NON MODIFICARE%%%%%%%%%%%%%%%%%

\begin{center} %INSERIRE ALL'INTERNO IL TITOLO DOCUMENTO CHE COMPARIRA NELLA PAGINA INIZIALE
	\begin{Huge}	
				\textbf{\TITOLODOC}
			\\
	\end{Huge}
\end{center}

%\setlength{\parindent}{18pt} %settato indentazione di default 
\section*{\LARGE Sommario:} %SEZIONE SOMMARIO
Il presente documento descrive in modo dettagliato le classi e i relativi metodi che andranno a comporre il sistema.

\indent \indent

\section*{\LARGE Stato del documento:}
\indent \indent
	Formale Esterno


\section*{\LARGE Redazione:}
	\begin{table}[!h]
		\begin{center}
			\begin{tabular}
				{|c|c|}
				\hline
				%%%%%%%%%%%%%%INTESTAZIONE COLONNE%%%%%%%%%%%%%%%%%%%%%%%%%%%%%%%%
				\multicolumn{2}{|c|}{ \textbf{Redazione} } \\
				\hline
				\textbf{Fase} & \textbf{Redattori} \\
				%%%%%%%%%%%%%%FINE INTESTAZIONE COLONNE%%%%%%%%%%%%%%%%%%%%%%%%%%%%%%%%%%%%%%
				\hline
				%%%%%%%%%%% PARTE DA MODIFICARE %%%%%%%%%%%%%%%%%%%%%%%%%%%%%%%%%%%%%%%%%%		
				\multirow{2}{*}{Pre-RR} & Cunico Marco\\
										& Dal Bosco Davide\\
				\hline
				\multirow{1}{*}{RR-RPP} & Bizzotto Piero\\
										
				\hline
				%%%%%%%%%%% FINE PARTE DA MODIFICARE %%%%%%%%%%%%%%%%%%%%%%%%%%%
			\end{tabular}
			\caption{Lista Redattori} %INSERIRE DIDASCALIA - SE NECESSARIA - 
			\label{tabredazione}
		\end{center}
	\end{table}
	
	
\section*{\LARGE Approvazione:}
\begin{table}[!h]
	\begin{center}
		\begin{tabular}
			{|c|c|}
			\hline
			%%%%%INTESTAZIONE COLONNE%%%%%%%%%%%%%%%%%%%%%%%%%%%%%%%
			\multicolumn{2}{|c|}{ \textbf{Approvazione} } \\
			\hline
			\textbf{Fase} & \textbf{Approvatori} \\
			%%%%%%%%%%%%%%FINE INTESTAZIONE COLONNE%%%%%%%%%%%%%%%%%%%%%%%%%%%%%%
			\hline
			%%%%%%%%%%% PARTE DA MODIFICARE %%%%%%%%%%%%%%%%%%%%%%%%%%%%%%%%%%%%%%		
			\multirow{1}{*}{Pre-RR} & Geremia Mirco \\
									
			\hline
			\multirow{1}{*}{RR-RPP} & Geremia Mirco\\
									
			\hline
			%%%%%%%%%%% FINE PARTE DA MODIFICARE %%%%%%%%%%%%%%%%%%%%%%%%%%%%%%%%%%%
		\end{tabular}
		\caption{Lista Approvatori} %INSERIRE DIDASCALIA - SE NECESSARIA - 
		\label{tabapprovazione}
	\end{center}
\end{table}

\textbf{}
\newpage
\section*{\LARGE Lista di Distribuzione:}

	\begin{elenconumerato}{\normindent}
		\item WebShape 
		\item I committenti Conte Renato e Vardanega Tullio in rappresentanza \\  dell'azienda proponente Zucchetti SPA
	\end{elenconumerato}



\section*{\Large Registro delle Modifiche:}


\begin{center}
	\begin{table}[h]
		  \begin{tabular*}
			{1\textwidth}%
				{@{\extracolsep{\fill}}|p{0.1\textwidth}|p{0.54\textwidth}|p{0.26\textwidth}|}
			 \hline
%%%%%%%%%%%%%%INTESTAZIONE COLONNE%%%%%%%%%%%%%%%%%%%%%%%%%%%%%%%%%%%%%%%%%%%%%%%%%%%%%%%%%%%%%%
			\textbf{Versione}  & \textbf{Descrizione} & \textbf{Autore} \\
%%%%%%%%%%%%%%FINE INTESTAZIONE COLONNE%%%%%%%%%%%%%%%%%%%%%%%%%%%%%%%%%%%%%%%%%%%%%%%%%%%%%%%%%%%%%%
		 \hline
%%%%%%%%%%% PARTE DA MODIFICARE %%%%%%%%%%%%%%%%%%%%%%%%%%%%%%%%%%%%%%%%%%%%%%%%%%%%%%%%%%%%%%%%%		
    	 	0.0 & 	 11$\slash$02$\slash$2009 Bozza & Dissegna Stefano \\

		\hline %%FINE RIGA
%%%%%%%%%%% FINE PARTE DA MODIFICARE %%%%%%%%%%%%%%%%%%%%%%%%%%%%%%%%%%%%%%%%%%%%%%%%%%%%%%%%%%%
		\end{tabular*}
	\caption{didascalia tabella 	MODIFICHE} %INSERIRE DIDASCALIA - SE NECESSARIA - 
	\label{tab:modifiche}
	\end{table}
\end{center}


\newpage
\thispagestyle{fancy}
\tableofcontents
\thispagestyle{fancy}
\newpage

%qui inizia l'indice, che viene creato automaticamente

\sezione{Introduzione}

\subsezione{Scopo del documento}
Il documento descrive a livello di dettaglio l'architettura del sistema.

\subsezione{Scopo del prodotto}
Il software AJAXDRAW \`e proposto per verificare e dimostrare la fattibilit\`a di realizzazione di un'applicazione di disegno grafico, in grado di poter elaborare figure vettoriali primitive e complesse utilizzando le tecnologie web.

\subsezione{Glossario}
All'interno del documento \textit{Glossario}, sono presentati i termini tecnici utilizzati in tutti i documenti. Il glossario \`e fornito in allegato al presente documento.
\subsezione{Riferimenti}
Il presente documento \`e redatto utilizzando le convenzioni inserite nel documento \textit{NormeDiProgetto}, consultabile dal repository pubblico al quale WebShape si appoggia per i suoi progetti.
\subsubsezione{Normativi}
\begin{elencopuntato}[\subsubsecindent]
\item[-] \textit{NormeDiProgetto.pdf}
\item[-] Raccomandazione SVG: http://www.w3.org/TR/SVG11/
\item[-] Raccomandazione CSS: http://www.w3.org/TR/CSS21/
\item[-] Working Draft HTML 5: http://www.w3.org/TR/html5/
\item[-] Specifiche UML: http://www.omg.org/spec/UML/2.1.2/
\end{elencopuntato}

\sezione{Standard di progetto}
\subsezione{Standard di progettazione architetturale}
La progettazione verr\`a presentata tramite una descrizione testuale delle componenti accompagnata da diagrammi UML.
\subsezione{Standard di documentazione del codice}
Si vedano le \textit{NormeDiprogetto}.
\subsezione{Standard di denominazione di entità e relazioni}
Si vedano le \textit{NormeDiprogetto}.
\subsezione{Standard di programmazione}
Si vedano le \textit{NormeDiprogetto}.
\subsezione{Strumenti di lavoro}
Si vedano le \textit{NormeDiprogetto}.

\sezione{Specifica delle componenti}

\subsezione{ApplicationLogic}

\subsezione{Figure}
\subsubsezione{Tipo, obiettivo e funzione del componente}
Classe astratta che rappresenta una figura generica nel sistema, ovvero un oggetto disegnabile nel canvas.
\subsubsezione{Relazioni d'uso di altre componenti}
Mantiene un'istanza della classe \textit{BoundingRectangle} per rappresentare i limiti della figura. Contiene anche un'istanza di \textit{Colour} che rappresenta il colore del bordo. Deriva, ma non implementa, l'interfaccia \textit{SVGReadable} e l'interfaccia \textit{SVGWritable}, in quanto le sue sottoclassi dovranno essere serializzabili e deserializabili in file SVG.
\subsubsezione{Interfacce e relazioni di uso da altre componenti}
Viene aggregata dalla classe \textit{FigureSet}.
\subsubsezione{Attivit\`a svolte e dati trattati}
\paragrafo{Attributi privati:}
\begin{elencopuntato}[\normindent]
\item[-] \textit{_selected} dice se la figura \`e selezionata
\item[-] \textit{_bounds} \textit{BoundingRectangle} che racchiude la figura
\item[-] \textit{_borderColour} colore del bordo della figura
\end{elencopuntato}
\paragrafo{Metodi pubblici:}
\begin{elencopuntato}[\normindent]
\item[-] \textit{getBounds()} ritorna \textit{_bounds}
\item[-] \textit{getBorderColour()} ritorna \textit{_borderColour}
\item[-]  \textit{draw(canvas)} metodo astratto per disegnare la figura sul canvas.
\item[-]  \textit{getMainPoints} metodo astratto per ottenere i punti della figura manipolabili dall'utente.
\item[-]  \textit{setSelection(value)} permette di selezionare/deselezionare la figura. Modifica \textit{_selected}.
\item[-]  \textit{isSelected()} dice se la figura \`e correntemente selezionata. Ritorna \textit{_selected}.
\item[-]  \textit{eachProperty(fn)} applica la funzione fn ad ogni propriet\`a della figura, ovvero a \textit{_bounds} e \textit{_borderColour}.
\item[-] \textit{drawSelection(canvas)} disegna dei quadratini intorno ai punti principali della figura, ottenuti tramite \textit{getMainPoints}, se la figura \`e selezionata.
\end{elencopuntato}

\subsezione{FigureSet}
\subsubsezione{Tipo, obiettivo e funzione del componente}
Rappresenta una collezione di figure. Tutte le figure disegnate dall'utente sono contenute in questa collezione, la quale ne permette la manipolazione e l'attraversamento. Per cancellare una figura \`e sufficiente rimuoverla dalla collezione.
\subsubsezione{Relazioni d'uso di altre componenti}
Contiene le istanze di \textit{Figure} correntemente attive.
\subsubsezione{Interfacce e relazioni di uso da altre componenti}
Usata dalla classe \textit{SVGWriter} per serializzare in SVG le figure correnti a dall'interfaccia utente per accedere alle figure. Pu\`o venire generata dalla classe \textit{SVGReader} dopo la deserializzazione di un file SVG.
\subsubsezione{Attivit\`a svolte e dati trattati}
\paragrafo{Attributi privati:}
\begin{elencopuntato}[\normindent]
\item[-] \textit{_figures} array contenente le figure.
\end{elencopuntato}
\paragrafo{Metodi pubblici:}
\begin{elencopuntato}[\normindent]
\item[-]  \textit{selectFigure()} permette di selezionare una figura contenente il punto passato come argomento, se tale figura esiste.
\item[-]  \textit{each(fn)} permette di applicare una funzione ad ogni figura. Viene usato per iterare attraverso la collezione. Applica fn ad ogni elemento di \textit{_figures}. 
\item[-]  \textit{add(f)} aggiunge la figura f alla fine di \textit{_figures}.
\item[-]  \textit{rem(f)} rimuove da \textit{_figures} tutte le figure identiche (ovvero aventi lo stesso indirizzo) di f.
\end{elencopuntato}

\sezione{Tracciamento della relazione componenti - requisiti}

\end{document}
