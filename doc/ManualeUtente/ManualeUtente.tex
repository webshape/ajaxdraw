\input{../TeX/base} %BASE!!!
\usepackage{multirow}
\title{\TITOLODOC}
\author{Marco Cunico}

\begin{document}

\renewcommand{\insertversion}{0.0} %INSERIRE LA VERSIONE QUI DENTRO STILE x.x.xx
\renewcommand{\TITOLODOC}{Manuale Utente} %INSERIRE IL TITOLO DEL DOCUMENTO DA FAR COMPARIRE A PIE PAGINA
\renewcommand{\glosspath}{.\glossario} %INSERIRE PERCORSO RELATIVO

%%%%%%%%%%%%%%%%%%%%%%PARTE DA NON MODIFICARE%%%%%%%%%%%%%%%%%
\begin{titlepage}
\begin{center}
	\begin{Large}	\today \end{Large}
\end{center}

\vspace{20pt}

\begin{center}
	\begin{Huge}
				\textbf{\ajax}
	\end{Huge}
\end{center}			

\begin{center}
	\begin{large}
				\textbf{Software per il Disegno Grafico\\ in Tecnologie Web}
	\end{large}
\end{center}			

\vspace{20pt}

\begin{center}
\includegraphics[width=150pt]{../logo/logo}
\end{center}

\vspace{170pt}
\begin{center} %INSERIRE ALL'INTERNO IL TITOLO DOCUMENTO CHE COMPARIRA NELLA PAGINA INIZIALE				
	\begin{Huge}
				\textbf{\TITOLODOC}
	\end{Huge}
			\\
\end{center}
\vspace{190pt}
\begin{center}
Versione: \insertversion
\end{center}
\end{titlepage}

\newpage
%%%%%%%%%%%%%%%%%%%%%%FINE PARTE DA NON MODIFICARE%%%%%%%%%%%%%%%%%

\begin{center} %INSERIRE ALL'INTERNO IL TITOLO DOCUMENTO CHE COMPARIRA NELLA PAGINA INIZIALE
	\begin{Huge}	
				\textbf{\TITOLODOC}
			\\
	\end{Huge}
\end{center}

%\setlength{\parindent}{18pt} %settato indentazione di default 
\section*{\LARGE Sommario:}
Il presente documento descrive l'analisi dei costi e delle risorse effettuata dall'azienda WebShape per lo sviluppo del Capitolato C04.

 %SEZIONE SOMMARIO
\indent \indent

\section*{\LARGE Stato del documento:}
\indent \indent
	Formale Esterno

\section*{\LARGE Redazione:}
	\begin{table}[!h]
		\begin{center}
			\begin{tabular}
				{|c|c|}
				\hline
				%%%%%%%%%%%%%%INTESTAZIONE COLONNE%%%%%%%%%%%%%%%%%%%%%%%%%%%%%%%%
				\multicolumn{2}{|c|}{ \textbf{Redazione} } \\
				\hline
				\textbf{Fase} & \textbf{Redattori} \\
				%%%%%%%%%%%%%%FINE INTESTAZIONE COLONNE%%%%%%%%%%%%%%%%%%%%%%%%%%%%%%%%%%%%%%
				\hline
				%%%%%%%%%%% PARTE DA MODIFICARE %%%%%%%%%%%%%%%%%%%%%%%%%%%%%%%%%%%%%%%%%%		
				\multirow{2}{*}{PPP-RQ} & \\
										& \\
				\hline
				%%%%%%%%%%% FINE PARTE DA MODIFICARE %%%%%%%%%%%%%%%%%%%%%%%%%%%
			\end{tabular}
			\caption{Lista Redattori} %INSERIRE DIDASCALIA - SE NECESSARIA - 
			\label{tabredazione}
		\end{center}
	\end{table}
	
	
\section*{\LARGE Approvazione:}
\begin{table}[!h]
	\begin{center}
		\begin{tabular}
			{|c|c|}
			\hline
			%%%%%INTESTAZIONE COLONNE%%%%%%%%%%%%%%%%%%%%%%%%%%%%%%%
			\multicolumn{2}{|c|}{ \textbf{Approvazione} } \\
			\hline
			\textbf{Fase} & \textbf{Approvatori} \\
			%%%%%%%%%%%%%%FINE INTESTAZIONE COLONNE%%%%%%%%%%%%%%%%%%%%%%%%%%%%%%
			\hline
			%%%%%%%%%%% PARTE DA MODIFICARE %%%%%%%%%%%%%%%%%%%%%%%%%%%%%%%%%%%%%%		
			\multirow{1}{*}{RPP-RQ} & \\
									
			\hline
			%%%%%%%%%%% FINE PARTE DA MODIFICARE %%%%%%%%%%%%%%%%%%%%%%%%%%%%%%%%%%%
		\end{tabular}
		\caption{Lista Approvatori} %INSERIRE DIDASCALIA - SE NECESSARIA - 
		\label{tabapprovazione}
	\end{center}
\end{table}

\textbf{}
\newpage
\section*{\LARGE Lista di Distribuzione:}

	\begin{elenconumerato}{\normindent}
		\item WebShape 
		\item I committenti Conte Renato e Vardanega Tullio in rappresentanza \\  dell'azienda proponente Zucchetti SPA
	\end{elenconumerato}




\section*{\Large Registro delle Modifiche:}


\begin{center}
	\begin{table}[h]
		  \begin{tabular*}
			{1\textwidth}%
				{@{\extracolsep{\fill}}|p{0.1\textwidth}|p{0.54\textwidth}|p{0.26\textwidth}|}
			 \hline
%%%%%%%%%%%%%%INTESTAZIONE COLONNE%%%%%%%%%%%%%%%%%%%%%%%%%%%%%%%%%%%%%%%%%%
			\textbf{Versione}  & \textbf{Descrizione} & \textbf{Autore} \\
%%%%%%%%%%%%%%FINE INTESTAZIONE COLONNE%%%%%%%%%%%%%%%%%%%%%%%%%%%%%%%%%%%%%%%
		 \hline
%%%%%%%%%%% PARTE DA MODIFICARE %%%%%%%%%%%%%%%%%%%%%%%%%%%%%%%%%%%%%%%%%%%
    	  \hline
    	  0.0 & 27/02/2008 Prima stesura & \\

		\hline %%FINE RIGA
%%%%%%%%%%% FINE PARTE DA MODIFICARE %%%%%%%%%%%%%%%%%%%%%%%%%%%%%%%%%%%%%
		\end{tabular*}
	\caption{Registro delle modifiche} %INSERIRE DIDASCALIA - SE NECESSARIA - 
	\label{tab:modifiche}
	\end{table}
\end{center}


\newpage
\thispagestyle{fancy}
\tableofcontents
\thispagestyle{fancy}
\newpage

\sezione{Introduzione}

\subsezione{Scopo del documento}
Il presente documento si propone di fornire la documentazione di supporto necessaria per l'utilizzo del prodotto.\\

\subsezione{Definizione dell'utente del prodotto}
Il prodotto \`e pensato per tutti gli utenti che hanno conoscenze minime in ambiente di grafica vettoriale e che possiedono una connessione ad internet. Il sistema potr\`a anche essere usato da chiunque voglia estendere l'applicazione, secondo le norme che regolano la licenza d'utilizzo.\\

\subsezione{Come leggere il manuale}
Il presente Manuale Utente \`e suddiviso in varie sezioni. Nella sezione ''Introduzione'' il lettore pu\`o trovare riferimenti a documenti utili per l'uso del prodotto e informazioni riguardo eventuali segnalazioni di problemi riscontrati nel suo utilizzo.
Nella sezione ''Descrizione generale'' viene descritto il prodotto illustrandone lo scopo, le modalit\`a di utilizzo, le caratteristiche salienti, i requisiti tecnici. Nella sezione ''Istruzioni d'uso'' sono presenti le operazioni da effettuare per utilizzare il software, a partire dall'installazione fino al termine di una sessione di utilizzo.\\ %integrate anche mediante vari screenshot


\subsezione{Documenti Utili}
\begin{elencopuntato}[\normindent]
	\item[-] \textit{Glossario.pdf}
\end{elencopuntato}

\subsezione{Come riportare problemi e malfunzionamenti}
Nel caso durante l'utilizzo del prodotto sorgano problemi o malfunzionamenti non descritti nel seguente documento, l'utente \`e invitato a contattare la WebShape all'indirizzo \href{mailto:webshape.contact@gmail.com}{webshape.contact@gmail.com} riportando i seguenti dati:\\
\begin{elencopuntato}[\normindent]
	\item[-] Versione del prodotto
	\item[-] Sistema operativo in uso
	\item[-] Browser in uso
	\item[-] Configurazione hardware in uso
	\item[-] Fase di utilizzo in cui si \`e manifestato l'errore
	\item[-] Descrizione dell'errore
\end{elencopuntato}

\sezione{Descrizione Generale}

\subsezione{Caratteristiche del Prodotto}
Il prodotto \`e stato sviluppato come un'applicazione web. \`E possibile utilizzarlo, una volta installato sul server, semplicemente accedendovi tramite un browser. Una volta effettuato l'accesso alla pagina \`e possibile accedere a tutte le
funzionalit\`a. Il sistema permette un salvataggio dei disegni che potranno essere ricaricati e aggiornati.

\subsezione{Requisiti tecnici per il funzionamento del programma}
\subsubsezione{lato client}
\begin{elencopuntato}[\normindent]
    \item[-] Requisiti software: Firefox 3.2b2 o successivo, Firefox 3.0.6 tutto tranne il disegno di testo, Opera 9.6 tutto tranne il disegno di testo, Chrome (versione attuale), Safari (versione attuale), IE8: tutto tranne il testo e la selezione che risulta non essere ottimale.\\
    Risoluzione minima consigliata: 1024x768.\\
N.B. Dove non \`e supportato il testo, c'\`e un supporto minimo: un solo font, colore di riempimento ignorato, set di caratteri minimo. 
    \item[-] Requisiti hardware:  Processore Intel 2GHz o compatibili; 512MB memoria RAM 
\end{elencopuntato}
\subsubsezione{lato server}
\begin{elencopuntato}[\normindent]
    \item[-] Requisiti software:  JVM 1.5 o superiore, Tomcat (Bosco..)
    \item[-] Requisiti hardware:  Processore Intel 1GHz; 1GB memoria RAM
\end{elencopuntato}

\sezione{Istruzioni per l'uso}

\subsezione{Installazione del programma}
\subsubsezione{lato client}
Il programma verr\`a fornito pacchettizzato con tutte le librerie necessarie al suo funzionamento. \`E sufficiente avviare lo script per avviare l'installazione del sistema sul proprio computer.

\subsubsezione{lato server}
Rimandiamo ai seguenti link per l'installazione del server Tomcat e della JVM
 
\subsezione{Avvio del programma}
Con il programma viene consegnato anche un file index.html per l'esecuzione su browser. 
Per avviare il programma \`e sufficente cliccare sul file.

\subsezione{Descrizione funzionale}
vari screenshot con relativa descrizione, da fare alla fine

\sezione{Appendice}

\subsezione{Messaggi di errore e loro significato}
descrizione degli errori, dividendoli tra server e client\\
Errore di parsing durante il caricamento di un file SVG.\\
Errore di comunicazione col server. \\
Errore interno del server (e.g. non c'e' piu' spazio su disco).\\

\end{document}
