\input{../TeX/base} %BASE!!!
\usepackage{multirow}
\title{\TITOLODOC}
\author{Marco Cunico}

\begin{document}

\renewcommand{\insertversion}{0.0} %INSERIRE LA VERSIONE QUI DENTRO STILE x.x.xx
\renewcommand{\TITOLODOC}{Manuale Utente} %INSERIRE IL TITOLO DEL DOCUMENTO DA FAR COMPARIRE A PIE PAGINA
\renewcommand{\glosspath}{.\glossario} %INSERIRE PERCORSO RELATIVO

%%%%%%%%%%%%%%%%%%%%%%PARTE DA NON MODIFICARE%%%%%%%%%%%%%%%%%
\begin{titlepage}
\begin{center}
	\begin{Large}	\today \end{Large}
\end{center}

\vspace{20pt}

\begin{center}
	\begin{Huge}
				\textbf{\ajax}
	\end{Huge}
\end{center}			

\begin{center}
	\begin{large}
				\textbf{Software per il Disegno Grafico\\ in Tecnologie Web}
	\end{large}
\end{center}			

\vspace{20pt}

\begin{center}
\includegraphics[width=150pt]{../logo/logo}
\end{center}

\vspace{170pt}
\begin{center} %INSERIRE ALL'INTERNO IL TITOLO DOCUMENTO CHE COMPARIRA NELLA PAGINA INIZIALE				
	\begin{Huge}
				\textbf{\TITOLODOC}
	\end{Huge}
			\\
\end{center}
\vspace{190pt}
\begin{center}
Versione: \insertversion
\end{center}
\end{titlepage}

\newpage
%%%%%%%%%%%%%%%%%%%%%%FINE PARTE DA NON MODIFICARE%%%%%%%%%%%%%%%%%

\begin{center} %INSERIRE ALL'INTERNO IL TITOLO DOCUMENTO CHE COMPARIRA NELLA PAGINA INIZIALE
	\begin{Huge}	
				\textbf{\TITOLODOC}
			\\
	\end{Huge}
\end{center}

%\setlength{\parindent}{18pt} %settato indentazione di default 
\section*{\LARGE Sommario:}
Il presente documento descrive l'analisi dei costi e delle risorse effettuata dall'azienda WebShape per lo sviluppo del Capitolato C04.

 %SEZIONE SOMMARIO
\indent \indent

\section*{\LARGE Stato del documento:}
\indent \indent
	Formale Esterno

\section*{\LARGE Redazione:}
	\begin{table}[!h]
		\begin{center}
			\begin{tabular}
				{|c|c|}
				\hline
				%%%%%%%%%%%%%%INTESTAZIONE COLONNE%%%%%%%%%%%%%%%%%%%%%%%%%%%%%%%%
				\multicolumn{2}{|c|}{ \textbf{Redazione} } \\
				\hline
				\textbf{Fase} & \textbf{Redattori} \\
				%%%%%%%%%%%%%%FINE INTESTAZIONE COLONNE%%%%%%%%%%%%%%%%%%%%%%%%%%%%%%%%%%%%%%
				\hline
				%%%%%%%%%%% PARTE DA MODIFICARE %%%%%%%%%%%%%%%%%%%%%%%%%%%%%%%%%%%%%%%%%%		
				\multirow{2}{*}{PPP-RQ} & \\
										& \\
				\hline
				%%%%%%%%%%% FINE PARTE DA MODIFICARE %%%%%%%%%%%%%%%%%%%%%%%%%%%
			\end{tabular}
			\caption{Lista Redattori} %INSERIRE DIDASCALIA - SE NECESSARIA - 
			\label{tabredazione}
		\end{center}
	\end{table}
	
	
\section*{\LARGE Approvazione:}
\begin{table}[!h]
	\begin{center}
		\begin{tabular}
			{|c|c|}
			\hline
			%%%%%INTESTAZIONE COLONNE%%%%%%%%%%%%%%%%%%%%%%%%%%%%%%%
			\multicolumn{2}{|c|}{ \textbf{Approvazione} } \\
			\hline
			\textbf{Fase} & \textbf{Approvatori} \\
			%%%%%%%%%%%%%%FINE INTESTAZIONE COLONNE%%%%%%%%%%%%%%%%%%%%%%%%%%%%%%
			\hline
			%%%%%%%%%%% PARTE DA MODIFICARE %%%%%%%%%%%%%%%%%%%%%%%%%%%%%%%%%%%%%%		
			\multirow{1}{*}{RPP-RQ} & \\
									
			\hline
			%%%%%%%%%%% FINE PARTE DA MODIFICARE %%%%%%%%%%%%%%%%%%%%%%%%%%%%%%%%%%%
		\end{tabular}
		\caption{Lista Approvatori} %INSERIRE DIDASCALIA - SE NECESSARIA - 
		\label{tabapprovazione}
	\end{center}
\end{table}

\textbf{}
\newpage
\section*{\LARGE Lista di Distribuzione:}

	\begin{elenconumerato}{\normindent}
		\item WebShape 
		\item I committenti Conte Renato e Vardanega Tullio in rappresentanza \\  dell'azienda proponente Zucchetti SPA
	\end{elenconumerato}




\section*{\Large Registro delle Modifiche:}


\begin{center}
	\begin{table}[h]
		  \begin{tabular*}
			{1\textwidth}%
				{@{\extracolsep{\fill}}|p{0.1\textwidth}|p{0.54\textwidth}|p{0.26\textwidth}|}
			 \hline
%%%%%%%%%%%%%%INTESTAZIONE COLONNE%%%%%%%%%%%%%%%%%%%%%%%%%%%%%%%%%%%%%%%%%%
			\textbf{Versione}  & \textbf{Descrizione} & \textbf{Autore} \\
%%%%%%%%%%%%%%FINE INTESTAZIONE COLONNE%%%%%%%%%%%%%%%%%%%%%%%%%%%%%%%%%%%%%%%
		 \hline
%%%%%%%%%%% PARTE DA MODIFICARE %%%%%%%%%%%%%%%%%%%%%%%%%%%%%%%%%%%%%%%%%%%
    	  \hline
    	  0.0 & 27/02/2008 Prima stesura & \\

		\hline %%FINE RIGA
%%%%%%%%%%% FINE PARTE DA MODIFICARE %%%%%%%%%%%%%%%%%%%%%%%%%%%%%%%%%%%%%
		\end{tabular*}
	\caption{Registro delle modifiche} %INSERIRE DIDASCALIA - SE NECESSARIA - 
	\label{tab:modifiche}
	\end{table}
\end{center}


\newpage
\thispagestyle{fancy}
\tableofcontents
\thispagestyle{fancy}
\newpage

\sezione{Introduzione}

\subsezione{Scopo del documento}
Il presente documento si propone di fornire la documentazione di supporto necessaria per l'utilizzo del prodotto.\\

\subsezione{Definizione dell'utente del prodotto}
Il prodotto \`e pensato per tutti gli utenti che hanno conoscenze minime in ambiente di grafica vettoriale e che possiedono una connessione ad internet. Il sistema potr\`a anche essere usato da chiunque voglia estendere l'applicazione, secondo le norme che regolano la licenza d'utilizzo.\\

\subsezione{Come leggere il manuale}
Il presente Manuale Utente \`e suddiviso in varie sezioni. Nella sezione ''Introduzione'' il lettore pu\`o trovare riferimenti a documenti utili per l'uso del prodotto e informazioni riguardo eventuali segnalazioni di problemi riscontrati nel suo utilizzo.
Nella sezione ''Descrizione generale'' viene descritto il prodotto illustrandone lo scopo, le modalit\`a di utilizzo, le caratteristiche salienti, i requisiti tecnici. Nella sezione ''Istruzioni d'uso'' sono presenti le operazioni da effettuare per utilizzare il software, a partire dall'installazione fino al termine di una sessione di utilizzo.\\ %integrate anche mediante vari screenshot


\subsezione{Documenti Utili}
\begin{elencopuntato}[\normindent]
	\item[-] \textit{Glossario.pdf}
\end{elencopuntato}

\subsezione{Come riportare problemi e malfunzionamenti}
Nel caso durante l'utilizzo del prodotto sorgano problemi o malfunzionamenti non descritti nel seguente documento, l'utente \`e invitato a contattare la WebShape all'indirizzo \href{mailto:webshape.contact@gmail.com}{webshape.contact@gmail.com} riportando i seguenti dati:\\
\begin{elencopuntato}[\normindent]
	\item[-] Versione del prodotto
	\item[-] Sistema operativo in uso
	\item[-] Browser in uso
	\item[-] Configurazione hardware in uso
	\item[-] Fase di utilizzo in cui si \`e manifestato l'errore
	\item[-] Descrizione dell'errore
\end{elencopuntato}

\sezione{Descrizione Generale}

\subsezione{Caratteristiche del Prodotto}
Il prodotto \`e stato sviluppato come un'applicazione web. \`E possibile utilizzarlo, una volta installato sul server, semplicemente accedendovi tramite un browser. Una volta effettuato l'accesso alla pagina \`e possibile accedere a tutte le
funzionalit\`a. Il sistema permette un salvataggio dei disegni che potranno essere ricaricati e aggiornati.

\subsezione{Requisiti tecnici per il funzionamento del programma}
\subsubsezione{lato client}
\begin{elencopuntato}[\normindent]
    \item[-] Requisiti software: Firefox 3.2b2 o successivo, Firefox 3.0.6 tutto tranne il disegno di testo, Opera 9.6 tutto tranne il disegno di testo, Chrome (versione attuale), Safari (versione attuale), IE8: tutto tranne il testo e la selezione che risulta non essere ottimale.\\
    Risoluzione minima consigliata: 1024x768.\\
N.B. Dove non \`e supportato il testo, c'\`e un supporto minimo: un solo font, colore di riempimento ignorato, set di caratteri minimo. 
    \item[-] Requisiti hardware:  Processore Intel 2GHz o compatibili; 512MB memoria RAM 
\end{elencopuntato}
\subsubsezione{lato server}
\begin{elencopuntato}[\normindent]
    \item[-] Requisiti software:  JVM 1.5 o superiore, Tomcat (Bosco..)
    \item[-] Requisiti hardware:  Processore Intel 1GHz; 1GB memoria RAM
\end{elencopuntato}

\sezione{Istruzioni per l'uso}

\subsezione{Installazione del programma}
\subsubsezione{lato client}
Il programma verr\`a fornito pacchettizzato con tutte le librerie necessarie al suo funzionamento. \`E sufficiente avviare lo script per avviare l'installazione del sistema sul proprio computer.

\subsubsezione{lato server}
Rimandiamo ai seguenti link per l'installazione del server Tomcat e della JVM
 
\subsezione{Avvio del programma}
Con il programma viene consegnato anche un file index.html per l'esecuzione su browser. 
Per avviare il programma \`e sufficente cliccare sul file.

\subsezione{Descrizione funzionale}
\subsezione{Il Menu}
Il programma dispone di un menu a tendina nella parte alta dell'interfaccia grafica.

\begin{figure}[!ht]
\centering
\includegraphics[scale=4]{images/menu.png}
\caption{menu a tendina}
\end{figure} 

Nella sezione \textit{File} \`e possibile salvare il lavoro effettuato su proprio computer tramite file svg; oppure caricare un lavoro realizzato in precedenza. 
L'utente dovr\`a rispettivamente scegliere un nome per il file da salvare o scegliere il file da caricare.
Nella sezione \textit{Visualizza} l'utente pu\`o visualizzare i dialog delle propriet\`a nel caso fossero stati chiusi  tramite il tasto di chiusura in alto a destra (la classica X). \textit{Colori} si riferisce alla finestra che modifica i colori di una figura, \textit{propriet} invece gestisce le posizioni e dimensioni.
Infine la sezione \textit{About} visualizza informazioni sul prodotto e sulla'azienda sviluppatrice.


\subsezione{La Toolbar}
Le funzioni di disegno del programma sono presenti nei bottoni della parte sinistra dell'interfaccia grafica. L'utente deve cliccare sul pulsante della funzione desiderata. Il putatore del mouse cambia a seconda della funzione di disegno scelta.

\begin{figure}[!ht]
\centering
\includegraphics[scale=4]{images/selezione.png}
\caption{Selezione}
\end{figure} 

Qui \`e stata usata la funzione \textit{seleziona} per selezionare una figura e tutti i suoi parametri sono visualizzati nelle finestre grafiche a destra. Se una figura  selezionata i suoi vertici sono evidenziati. In questo caso l'utente, tenendo premuto il puntatore mouse dentro alla figura pu\`o spostarla all'interno dell'area di disegno. Allo stesso modo, tenendo premuto sui vertici della figura, puo modificarne manualmente le dimensioni.

\begin{figure}[!ht]
\centering
\includegraphics[scale=4]{images/ellisse.png}
\caption{Disegno di un Ellisse}
\end{figure} 

\vspace{100pt}
Qui \`e stata selezionata la funzione \textit{Ellisse}  cliccando sull'apposito bottone (evidenziato in grigio). L'utente deve fare click nella pozione in cui vuole disegnare la figura (vertice alto a sinistra della figura), poi col puntatore del mouse premuto decide la dimensione della figura. La stessa procedura vale per il disegno di un \textit{Quadrato-rettangolo}.

\begin{figure}[!ht]
\centering
\includegraphics[scale=4]{images/poligono.png}
\caption{Disegno di un Poligono regolare}
\end{figure} 

\vspace{300pt}
Qui \`e stato disegnato un \textit{poligono regolare}  cliccando sull'apposito bottone (evidenziato in grigio). L'unica differenze rispetto a ellissi e quadrati-rettangoli sta nella finestra di propriet\`a a destra dell'interfaccia grafica). \`E possibile stabilire il numero di lati del poligono tramite l'etichettta N. di lati. Basta cliccare sul bottone cambia per memorizzare le modifche sul numero di lati.

\begin{figure}[!ht]
\centering
\includegraphics[scale=4]{images/mano.png}
\caption{Strumento Mano per spostare, trascinare le figure}
\end{figure} 

\vspace{300pt}
Qui \`e stata selezionata la funzione \textit{Mano} cliccando sull'apposito bottone (evidenziato in grigio). L'utente deve fare click dentro alla figura e poi, tenendo premuto il puntatore, puo' spostare la figura dove desidera. Tale azione \`e realizzabile anche con il pulsante Selezione, descritto in precedenza. La medesima funzione pu\`o essere svolta a partire da una figura selezionata sul piano di disegno e spostata semplicemente con le frecce direzionali della propria tastiera.

\begin{figure}[!ht]
\centering
\includegraphics[scale=4]{images/matita.png}
\caption{Disegno a mano libera}
\end{figure} 

\vspace{300pt}
Qui \`e stata selezionata la funzione \textit{Disegno a mano libera} cliccando sull'apposito bottone (evidenziato in grigio). L'utente deve semplicemente tenere premuto il puntatore del mouse e disegnare.

\begin{figure}[!ht]
\centering
\includegraphics[scale=4]{images/linea.png}
\caption{Disegno di una retta}
\end{figure} 

\vspace{300pt}
Qui \`e stata selezionata la funzione \textit{Disegno di una retta} cliccando sull'apposito bottone (evidenziato in grigio). L'utente deve semplicemente tenere premuto il puntatore del mouse e usare il cursore per disegnare e direzionare la linea retta. La fine della retta \`e determinata dal rilascio del mouse.

\begin{figure}[!ht]
\centering
\includegraphics[scale=4]{images/label.png}
\caption{Etichetta di testo}
\end{figure} 

\vspace{300pt}
Qui \`e stata selezionata la funzione \textit{Etichetta di testo} cliccando sull'apposito bottone (evidenziato in grigio). L'utente deve semplicemente cliccare sull'area di disegno per scegliere la posizione del testo e aiutarsi con la finestra di propriet\`a a destra per settare il testo dell'etichetta e il font.
 
\begin{figure}[!ht]
\centering
\includegraphics[scale=4]{}
\caption{Uso dello Zoom}
\end{figure} 

\vspace{100pt}
Qui \`e stata selezionata la funzione \textit{Zoom} L'utente ha possibilit\`a di impostare manualmente la percentuale di zoom agendo sulla combobox posta in basso a sinistra dell'applicazione con un click sulla combobox e un click sulla percentuale desiderata. Un altro modo per effettuare le modifiche di zoom sul piano di disegno prevede l'tuilizzo dei pulsanti lente (evidenziati in grigio) per aumentare o diminuire il grado di zoom.
 
\begin{figure}[!ht]
\centering
\includegraphics[scale=4]{images/ellisse.png}
\caption{Cancellazione totale del piano di disegno}
\end{figure} 

\vspace{100pt}
Facendo click sul bottone \textit{Svuota} in basso a sinistra, l'utente pu\`o cancellare tutto ci\`o che \`e stato di segnato sul piano di disegno fino a quel punto e ripartire da zero con un nuovo disegno: ATTENZIONE: si consiglia agli utenti di salvare prima di cancellare per evitare di perdere l'eventuale lavoro utile svolto fino a quel momento.

\begin{figure}[!ht]
\centering
\includegraphics[scale=4]{images/canella_elemento_prima.png}
\caption{Cancellazione di un elemento di disegno  - PRIMA}
\end{figure} 

\begin{figure}[!ht]
\centering
\includegraphics[scale=4]{images/cancella_elemento_dopo.png}
\caption{Cancellazione di un elemento di disegno  - DOPO}
\end{figure}

\vspace{100pt}
Il bottone \textit{Cancella} in basso a sinistra (sotto la toolbar di disegno) permette di cancellare una figura selezionata, 
basta un click sul pulsante \textit{seleziona} di selezione immagine e successivamente fare click sull'immagine che si intende cancellare; a questo punto \`e possibile cancellare la figura selezionata con un click sul pulsante suddetto. La medesima funzione può essere svolta a partire da una figura selezionata premendo il tasto canc da tastiera.

\begin{figure}[!ht]
\centering
\includegraphics[scale=4]{images/ellisse.png}
\caption{Disegno di una curva di bezier}
\end{figure} 

\vspace{100pt}
Qui \`e stata selezionata la funzione \textit{Curva di Bezier} consentendo all'utente di disegnare la curva in modo semplice e veloce.
Click sull'apposito bottone (evidenziato in grigio)  e successivamente fare click sul piano di disegno dove si desidera disegnare la curva; il punto selezionato sar\`a il punto P0 di partenza della curva dopodich\`e click sul piano di disegno per descrivere i punti P1, P2 e P3 che determineranno rispettivamente: P1 e P2 i punti di direzione e traiettoria della curva mentre P3 rappresenta il punto di arrivo e termine della curva stessa.

\begin{figure}[!ht]
\centering
\includegraphics[scale=4]{images/colore_bordo_prima.png}
\caption{cambio colore del bordo  - PRIMA}
\end{figure} 

\begin{figure}[!ht]
\centering
\includegraphics[scale=4]{images/colore_bordo_dopo.png}
\caption{cambio colore del bordo  - DOPO}
\end{figure} 


\vspace{100pt}
Qui \`e stata selezionata la funzione di cambiamento del colore del bordo della figura selezionata con l'ausilio del dialog \textit{Modifica Colori} posto in alto a destra. Per iniziare \`e necesssario fare click sul pulsante \textit{Cambia Colore Bordo}: 
\begin{elencopuntato}[\normindent]
\item[-] Appare sul dialog la ruota dei colori.
\item[-] Selezionare un colore a piacere sulla ruota e impostare il grado di opacit\`a dal rettangolo iscritto all'interno della ruota. 
\item[-]Quando il colore e l'opacit\`a sono stati scelti basta un click sul pulsante \textit{Cambia} per vedere il colore del contorno modificato con le tonalit\`a scelte. 
\end{elencopuntato}
Questa funzionalit\`a prevede la possibilit\`a per l'utente di cambiare il colore scegliendolo dalla \textit{Palette} di colori posti sotto il piano di disegno: per prima cosa bisogna selezionare \textit{bordo} sulla combobox a destra della \textit{Palette}, un click sul colore scelto per cambiare il colore del bordo sul dialog. A questo punto basta un click sul pulsante \textit{Cambia Colore Bordo} e un click su \textit{Cambia}.

\begin{figure}[!ht]
\centering
\includegraphics[scale=4]{images/colore_riempimento_prima.png}
\caption{cambio colore del riempimento  - PRIMA}
\end{figure} 

\begin{figure}[!ht]
\centering
\includegraphics[scale=4]{images/colore_riempimento_dopo.png}
\caption{cambio colore del riempimento  - DOPO}
\end{figure} 


\vspace{100pt}
Qui \`e stata selezionata la funzione di cambiamento del colore di riempimento della figura selezionata con l'ausilio del dialog \textit{Modifica Colori} posto in alto a destra. Per iniziare \`e necesssario fare click sul pulsante \textit{Cambia Colore Riemp.}: 
\begin{elencopuntato}[\normindent]
\item[-] Appare sul dialog la ruota dei colori.
\item[-] Selezionare un colore a piacere sulla ruota e impostare il grado di opacit\`a dal rettangolo iscritto all'interno della ruota. 
\item[-]Quando il colore e l'opacit\`a sono stati scelti basta un click sul pulsante \textit{Cambia} per vedere il colore del riempimento modificato con le tonalit\`a scelte. 
\end{elencopuntato}
Questa funzionalit\`a prevede la possibilit\`a per l'utente di cambiare il colore scegliendolo dalla \textit{Palette} di colori posti sotto il piano di disegno: per prima cosa bisogna selezionare \textit{bordo} sulla combobox a destra della \textit{Palette}, un click per il colore scelto per cambiare il riempimento sul dialog. A questo punto basta un click sul pulsante \textit{Cambia Colore Riemp.} e un click su \textit{Cambia}.


\begin{figure}[!ht]
\centering
\includegraphics[scale=4]{images/cambio_#lati_prima.png}
\caption{cambio numero lati di un poligono  - PRIMA}
\end{figure} 

\begin{figure}[!ht]
\centering
\includegraphics[scale=4]{images/cambio_#lati_dopo.png}
\caption{cambio numero lati di un poligono  - DOPO}
\end{figure} 

\vspace{100pt}
Qui \`e stata selezionata la funzione di cambiamento del numero di lati per il disegno di poligoni. Tale funzione viene gestita con l'ausilio del dialog \textit{Modifica Attributi} posto in alto a destra e precisamente nella parte riguardante il numero di lati. Impostando a piacere il numero di lati sul dialog si rende disponibile la possibilit\`a di selezionare il pulsante apposito \textit{Crea Poligoni} (evidenziato in grigio). A questo punto disponete il puntatore del mouse sulla zona del piano di disegno desiderata e, tenendo prmuto il tasto sinistro, disegnare un poligono con il numero di lati scelto.  

\sezione{Appendice}

\subsezione{Messaggi di errore e loro significato}
descrizione degli errori, dividendoli tra server e client\\
Errore di parsing durante il caricamento di un file SVG.\\
Errore di comunicazione col server. \\
Errore interno del server (e.g. non c'e' piu' spazio su disco).\\

\end{document}
