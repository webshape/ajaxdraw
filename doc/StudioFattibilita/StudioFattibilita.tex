	\input{../TeX/base} %BASE!!!

\title{\TITOLODOC}
\author{Piero Bizzotto}

\begin{document}

\renewcommand{\insertversion}{0.1} %INSERIRE LA VERSIONE QUI DENTRO STILE x.x.xx
\renewcommand{\TITOLODOC}{Studio di Fattibilit\`{a}} %INSERIRE IL TITOLO DEL DOCUMENTO DA FAR COMPARIRE A PIE PAGINA
\renewcommand{\glosspath}{.\glossario} %INSERIRE PERCORSO RELATIVO

%%%%%%%%%%%%%%%%%%%%%%PARTE DA NON MODIFICARE%%%%%%%%%%%%%%%%%
\begin{titlepage}
\begin{center}
	\begin{Large}	\today \end{Large}
\end{center}

\vspace{20pt}

\begin{center}
	\begin{Huge}
				\textbf{AjaxDraw}
	\end{Huge}
\end{center}			

\begin{center}
	\begin{large}
				\textbf{Software per il Disegno Grafico in Tecnologie Web}
	\end{large}
\end{center}			

\vspace{20pt}

\begin{center}
\includegraphics[width=150pt]{../logo/logo}
\end{center}

\vspace{170pt}
\begin{center} %INSERIRE ALL'INTERNO IL TITOLO DOCUMENTO CHE COMPARIRA NELLA PAGINA INIZIALE				
	\begin{Huge}
				\textbf{\TITOLODOC}
	\end{Huge}
			\\
\end{center}
\vspace{210pt}
\begin{center}
Versione: \insertversion
\end{center}
\end{titlepage}

\newpage
%%%%%%%%%%%%%%%%%%%%%%FINE PARTE DA NON MODIFICARE%%%%%%%%%%%%%%%%%

\begin{center} %INSERIRE ALL'INTERNO IL TITOLO DOCUMENTO CHE COMPARIRA NELLA PAGINA INIZIALE
	\begin{Huge}	
				\textbf{\TITOLODOC}
			\\
	\end{Huge}
\end{center}

%\setlength{\parindent}{18pt} %settato indentazione di default 
\section*{\Large Sommario:} %SEZIONE SOMMARIO
\indent \indent
Considerazioni ricavate dallo studio di fattibilit\`{a} del capitolato d\'{}appalto C04, 
Software per il Disegno Grafico in Tecnologie Web (in sigla AjaxDraw).


\section*{\Large Stato del documento:}
\indent \indent
	Formale Interno

\section*{\Large Redazione:}
	\begin{elencopuntato}[\normindent]
		\item[-] Mirco Geremia
	
	\end{elencopuntato}

\section*{\Large Approvazione:}
	\begin{elencopuntato}[\normindent]
		\item Approvatore N. 1
		\item Approvatore N. 2
		\item Approvatore N. 3
	\end{elencopuntato}

\section*{\LARGE Lista di Distribuzione:}

	\begin{elenconumerato}{\normindent}
		\item WebShape \footnote{Il termine WebShape designa una collettivit\`a di individui come da organigramma contenuto nel piano di progetto fornito in allegato al presente documento}
		\item I committenti Vardanega Tullio e Conte Renato in rappresentanza \\  dell'azienda proponente Zucchetti SPA
		%\item Il committente Conte Renato
	\end{elenconumerato}

\newpage



\section*{\Large Registro delle Modifiche:}


\begin{center}
	\begin{table}[h]
		  \begin{tabular*}
			{1\textwidth}%
				{@{\extracolsep{\fill}}|p{0.1\textwidth}|p{0.54\textwidth}|p{0.26\textwidth}|}
			 \hline
%%%%%%%%%%%%%%INTESTAZIONE COLONNE%%%%%%%%%%%%%%%%%%%%%%%%%%%%%%%%%%%%%%%%%%%%%%%%%%%%%%%%%%%%%%
			\textbf{Versione}  & \textbf{Descrizione} & \textbf{Autore} \\
%%%%%%%%%%%%%%FINE INTESTAZIONE COLONNE%%%%%%%%%%%%%%%%%%%%%%%%%%%%%%%%%%%%%%%%%%%%%%%%%%%%%%%%%%%%%%
		 \hline
%%%%%%%%%%% PARTE DA MODIFICARE %%%%%%%%%%%%%%%%%%%%%%%%%%%%%%%%%%%%%%%%%%%%%%%%%%%%%%%%%%%%%%%%%		
    	 	\insertversion & 	 21$\slash$11$\slash$2008 Aggiornato Fattiblit\`a tecnologica & Mirco Geremia \\
    	 	0.0 & 	 19$\slash$11$\slash$2008 Prima bozza & Mirco Geremia \\

		\hline %%FINE RIGA
%%%%%%%%%%% FINE PARTE DA MODIFICARE %%%%%%%%%%%%%%%%%%%%%%%%%%%%%%%%%%%%%%%%%%%%%%%%%%%%%%%%%%%
		\end{tabular*}
	\caption{didascalia tabella 	MODIFICHE} %INSERIRE DIDASCALIA - SE NECESSARIA - 
	\label{tab:modifiche}
	\end{table}
\end{center}


\newpage
\thispagestyle{fancy}
%\tableofcontents
\thispagestyle{fancy}
\newpage

\section{\Large Analisi Fattibilit\`{a}:}
\indent \indent
Dopo aver analizzato i cinque capitolati d'appalto presentati in data 12 Novembre 2008 
dai docenti Conte Renato e Vardanega Tullio, il gruppo WebShape ha deciso di proporsi come 
fornitore per il capitolato C04 (\H{}AJAXDRAW\H{}), che consiste nella creazione di un software 
per il disegno grafico in tecnologie web.
Tale scelta si basa prevalentemente sui seguenti punti:
	\begin{elenconumerato}{\normindent}
		\item interesse e capacit\`{a}: i membri del gruppo condividono lo stesso interesse nei confronti delle tecnologie web e delle applicazioni grafiche; inoltre alcuni componenti sono in possesso delle conoscenze necessarie per comprendere a fondo il dominio del problema, in quanto hanno gi\`{a} superato l'esame di Tecnologie Web della Laurea Triennale di Informatica dell'Universit\`{a} di Padova, mentre per gli altri tale esame rappresenta ulteriore stimolo di interesse nei riguardi del progetto.
		\item aspetto commerciale: il capitolato in oggetto rappresenta secondo il gruppo un buon prodotto 
		poich\`{e} l'utilizzo del software risulta pi\`{u}  semplice e immediato rispetto alla stessa applicazione desktop	(ad esempio esenta l'utente da oneri come l'installazione e l'aggiornamento del prodotto).
		E' da notare inoltre l'attuale orientamento del mercato informatico verso le tecnologie web, 
		che si stanno indubbiamente rivelando sempre pi\`{u} interessanti.
		\item fattibilit\`{a} tecnologica: i componenti del gruppo hanno accertato l'esistenza di strumenti software adatti alla realizzazione del progetto: 
				\begin{elencopuntato}[\normindent]
					\item l'oggetto "canvas" che su richiesta del cliente si andrà ad utilizzare rende disponibili le funzionalità di base del prodotto.
					\item il formato SVG richiesto per rappresentare le immagini, essendo un linguaggio derivato dall'XML, è facilmente utilizzabile senza dover ricorrere a librerie grafiche specifiche.
					\item il JavaScript è un linguaggio di programmazione semplice e completo, adatto a scrivere anche applicazioni di un certo rilievo, e non solo semplici script web; queste sue caratteristiche facilitano la realizzazione del progetto.
				\end{elencopuntato}
\end{elenconumerato}

Pertanto il gruppo WebShape si dichiara disponibile per la creazione e lo sviluppo del capitolato d\'{}appalto C04.

\end{document}
