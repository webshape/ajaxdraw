\input{../TeX/base} %BASE!!!

\title{\TITOLODOC}
\author{Mirco Geremia}

\begin{document}

\renewcommand{\insertversion}{1.0} %INSERIRE LA VERSIONE QUI DENTRO STILE x.x.xx
\renewcommand{\TITOLODOC}{Glossario} %INSERIRE IL TITOLO DEL DOCUMENTO DA FAR COMPARIRE A PIE PAGINA
\renewcommand{\glosspath}{.\glossario} %INSERIRE PERCORSO RELATIVO

%%%%%%%%%%%%%%%%%%%%%%PARTE DA NON MODIFICARE%%%%%%%%%%%%%%%%%
\begin{titlepage}
\begin{center}
	\begin{Large}	\today \end{Large}
\end{center}

\vspace{20pt}

\begin{center}
	\begin{Huge}
				\textbf{AJAXDRAW}
	\end{Huge}
\end{center}			

\begin{center}
	\begin{large}
				\textbf{Software per il Disegno Grafico\\ in Tecnologie Web}
	\end{large}
\end{center}			

\vspace{20pt}

\begin{center}
\includegraphics[width=150pt]{../logo/logo}
\end{center}

\vspace{170pt}
\begin{center} %INSERIRE ALL'INTERNO IL TITOLO DOCUMENTO CHE COMPARIRA NELLA PAGINA INIZIALE				
	\begin{Huge}
				\textbf{\TITOLODOC}
	\end{Huge}
			\\
\end{center}
\vspace{210pt}
\begin{center}
Versione: \insertversion
\end{center}
\end{titlepage}

\newpage
%%%%%%%%%%%%%%%%%%%%%%FINE PARTE DA NON MODIFICARE%%%%%%%%%%%%%%%%%

\begin{center} %INSERIRE ALL'INTERNO IL TITOLO DOCUMENTO CHE COMPARIRA NELLA PAGINA INIZIALE
	\begin{Huge}	
				\textbf{\TITOLODOC}
			\\
	\end{Huge}
\end{center}

%\setlength{\parindent}{18pt} %settato indentazione di default 
\section*{\LARGE Sommario:} %SEZIONE SOMMARIO
\indent \indent
Glossario

\section*{\LARGE Stato del documento:}
\indent \indent
	Formale Esterno

\section*{\LARGE Redazione:}
	\begin{table}[!h]
		\begin{center}
			\begin{tabular}
				{|c|c|}
				\hline
				%%%%%%%%%%%%%%INTESTAZIONE COLONNE%%%%%%%%%%%%%%%%%%%%%%%%%%%%%
				\multicolumn{2}{|c|}{ \textbf{Redazione} } \\
				\hline
				\textbf{Fase} & \textbf{Redattori} \\
				%%%%%%%%%%%%%%FINE INTESTAZIONE COLONNE%%%%%%%%%%%%%%%%%%%%%%%%%%%%%%%
				\hline
				%%%%%%%%%%% PARTE DA MODIFICARE %%%%%%%%%%%%%%%%%%%%%%%%%%%		
				{Pre-RR} &  Bizzotto Piero \\ & Geremia Mirco \\
				\hline
				{RR-RPP} & \\
				\hline
				%%%%%%%%%%% FINE PARTE DA MODIFICARE %%%%%%%%%%%%%%%%%%%%%%%%%
			\end{tabular}
			\caption{Lista Redattori} %INSERIRE DIDASCALIA - SE NECESSARIA - 
			\label{tabredazione}
		\end{center}
	\end{table}
	
\section*{\LARGE Approvazione:}

\begin{table}[!h]
	\begin{center}
		\begin{tabular}
			{|c|c|}
			\hline
			%%%%%%%%%%%%%%INTESTAZIONE COLONNE%%%%%%%%%%%%%%%%%%%%%%%%%%%%%%%%%%%
			\multicolumn{2}{|c|}{ \textbf{Approvazione} } \\
			\hline
			\textbf{Fase} & \textbf{Approvatori} \\
			%%%%%%%%%%%%%%FINE INTESTAZIONE COLONNE%%%%%%%%%%%%%%%%%%%%%%%
			\hline
			%%%%%%%%%%% PARTE DA MODIFICARE %%%%%%%%%%%%%%%%%%%%%%%%%%%%%%%%%%%%%%%		
			{Pre-RR} &  Dissegna Stefano \\
			\hline
			{RR-RPP} & \\
			\hline
			%%%%%%%%%%% FINE PARTE DA MODIFICARE %%%%%%%%%%%%%%%%%%%%%%%%%%%%%%%%%%%
		\end{tabular}
		\caption{Lista Approvatori} %INSERIRE DIDASCALIA - SE NECESSARIA - 
		\label{tabapprovazione}
	\end{center}
\end{table}
\textbf{}

\section*{\LARGE Lista di Distribuzione:}

	\begin{elenconumerato}{\normindent}
		\item WebShape 
     	\item I committenti Conte Renato e Vardanega Tullio in rappresentanza \\  dell'azienda proponente Zucchetti SPA
	\end{elenconumerato}

\newpage


\section*{\LARGE Registro delle Modifiche:}

\begin{center}
	\begin{table}[h]
		  \begin{tabular*}
			{1\textwidth}%
				{@{\extracolsep{\fill}}|p{0.1\textwidth}|p{0.54\textwidth}|p{0.26\textwidth}|}
			 \hline
%%%%%%%%%%%%%%INTESTAZIONE COLONNE%%%%%%%%%%%%%%%%%%%%%%%%%%%%%%%%%%%%%%%%%%%%%%%%%%%%%%%%%%%%%%
			\textbf{Versione}  & \textbf{Descrizione} & \textbf{Autore} \\
%%%%%%%%%%%%%%FINE INTESTAZIONE COLONNE%%%%%%%%%%%%%%%%%%%%%%%%%%%%%%%%%%%%%%%%%%%%%%%%%%%%%%%%%%%%%%
		 \hline
%%%%%%%%%%% PARTE DA MODIFICARE %%%%%%%%%%%%%%%%%%%%%%%%%%%%%%%%%%%%%%%%%%%%%%%%%%%%%%%%%%%%%%%%%		
			\hline
			1.0 & 09$\slash$12$\slash$2008  Verifica finale in preparazione al rilascio & Bizzotto Piero \\
			\hline
			0.3  &    08$\slash$12$\slash$2008 Aggiunta altri termini & Dal Bosco Davide \\
			\hline
			0.2&    06$\slash$12$\slash$2008 Aggiunta altri termini & Dissegna Stefano \\
			\hline
    	 	0.1 &	 05$\slash$12$\slash$2008 Aggiunta termini da AnalisiDeiRequisiti & Bizzotto Piero\\
    	 	\hline
    	 	0.0 & 	 04$\slash$12$\slash$2008 Struttura e aggiunta termini da NormeDiProgetto & Geremia Mirco \\

		\hline %%FINE RIGA
%%%%%%%%%%% FINE PARTE DA MODIFICARE %%%%%%%%%%%%%%%%%%%%%%%%%%%%%%%%%%%%%%%%%%%%%%%%%%%%%%%%%%%
		\end{tabular*}
	\caption{didascalia tabella 	MODIFICHE} %INSERIRE DIDASCALIA - SE NECESSARIA - 
	\label{tab:modifiche}
	\end{table}
\end{center}

\newpage

\section*{C}
\hypertarget{canvas}{}
\textbf{Canvas:}
Canvas \` e una estensione dell'HTML standard che permette il rendering dinamico di immagini bitmap gestibili attraverso un linguaggio di scripting.  Consiste in una regione disegnabile, definita in codice HTML con gli attributi \textit{height} and \textit{width}. Il codice \hyperlink{javascript}{\underline{JavaScript}} pu\`o accedere all'area con un set completo di funzioni per il disegno, simili a quelle comuni ad altre API 2D, permettendo cos\`i la generazione dinamica di disegni.\\

\section*{E}
\hypertarget{ecmascript}{}
\textbf{ECMAScript:}
\text Si veda \hyperlink{javascript}{\underline{JavaScript}}.\\

\section*{J}
\hypertarget{javascript}{}
\textbf{JavaScript:}
JavaScript (o ECMAScript) \`e un linguaggio di scripting orientato agli oggetti comunemente usato nei siti web. \`E  uno standard ISO.\\

\section*{L}
\hypertarget{latex}{}
\textbf{\LaTeX:}
\LaTeX \ \`e un linguaggio di markup usato per la redazione di documentii che permette di gestire in maniera semplice ed automatica la formattazione del testo.\\

\textbf{Linguaggio di markup:}
\hypertarget{markup}{}
Un linguaggio di markup \`e un linguaggio che stabilisce delle convenzioni standard e descrive dei meccanismi per la rappresentazione del testo.\\

\section*{O}
\hypertarget{opensource}{}
\textbf{OpenSource:}
Con il termine OpenSource si indica un tipo di licenza software in cui il codice sorgente viene reso disponibile; lo scopo dell'OpenSource \`e quello di favorire la collaborazione tra gli sviluppatori e il progresso del software tramite la condivisione di idee.\\

\section*{P}
\hypertarget{png}{}
\textbf{PNG:}
Il PNG (Portable Network Graphics) \`e un formato di file che permette di memorizzare immagini senza perdere informazioni.\\

\section*{R}
\hypertarget{rgb}{}
\textbf{RGB:}
RGB (Red, Green, Blue) \`e  il nome di un modello di colori di tipo additivo che si basa sui tre colori rosso, verde e blu, da cui appunto il nome RGB, da non confondere con i colori primari sottrattivi giallo, ciano e magenta (chiamati anche giallo, rosso e blu).
Un'immagine pu\`o  infatti essere scomposta, attraverso filtri o altre tecniche, in questi colori base che, miscelati tra loro, danno quasi tutto lo spettro dei colori visibili, con l'eccezione delle porpore.

\section*{S}
\hypertarget{svg}{}
\textbf{SVG:}
L' SVG (Scalable Vector Graphics) \` indica una tecnologia in grado di visualizzare oggetti di grafica vettoriale e, per questo motivo, di gestire immagini scalabili dimensionalmente.\\

\section*{W}
\hypertarget{w3c}{}
\textbf{W3C:}
Il W3C \`e un'associazione internazionale fondata nel 1994 con lo scopo di migliorare i protocolli ed i linguaggi del Web. Pur non essendo un organismo di standardizzazione, le raccomandazioni fornite dal W3C sono spesso trattate come standard.\\

\hypertarget{wiki}{}
\textbf{Wiki:}
Un wiki \`e un sistema che permette di gestire pagine web in maniera collaborativa favorendo la codivisione di informazioni tra gli utenti; in un wiki ognuno pu\`o aggiungere o modificare i contenuti di una pagina attraverso il browser.\\

\section*{X}
\hypertarget{xml}{}
\textbf{XML}
\text Formato estensibile raccomandato dal \hyperlink{w3c}{\underline{W3C}} per la rappresentazione di dati semistrutturati.\\

\end{document}
