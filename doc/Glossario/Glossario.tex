\input{../TeX/base} %BASE!!!
\usepackage{multirow}

\title{\TITOLODOC}
\author{Mirco Geremia}

\begin{document}

\renewcommand{\insertversion}{3.0} %INSERIRE LA VERSIONE QUI DENTRO STILE x.x.xx
\renewcommand{\TITOLODOC}{Glossario} %INSERIRE IL TITOLO DEL DOCUMENTO DA FAR COMPARIRE A PIE PAGINA
\renewcommand{\glosspath}{.\glossario} %INSERIRE PERCORSO RELATIVO

%%%%%%%%%%%%%%%%%%%%%%PARTE DA NON MODIFICARE%%%%%%%%%%%%%%%%%
\begin{titlepage}
\begin{center}
	\begin{Large}	\today \end{Large}
\end{center}

\vspace{20pt}

\begin{center}
	\begin{Huge}
				\textbf{AJAXDRAW}
	\end{Huge}
\end{center}			

\begin{center}
	\begin{large}
				\textbf{Software per il Disegno Grafico\\ in Tecnologie Web}
	\end{large}
\end{center}			

\vspace{20pt}

\begin{center}
\includegraphics[width=150pt]{../logo/logo}
\end{center}

\vspace{170pt}
\begin{center} %INSERIRE ALL'INTERNO IL TITOLO DOCUMENTO CHE COMPARIRA NELLA PAGINA INIZIALE				
	\begin{Huge}
				\textbf{\TITOLODOC}
	\end{Huge}
			\\
\end{center}
\vspace{200pt}
\begin{center}
Versione: \insertversion
\end{center}
\end{titlepage}

\newpage
%%%%%%%%%%%%%%%%%%%%%%FINE PARTE DA NON MODIFICARE%%%%%%%%%%%%%%%%%

\begin{center} %INSERIRE ALL'INTERNO IL TITOLO DOCUMENTO CHE COMPARIRA NELLA PAGINA INIZIALE
	\begin{Huge}	
				\textbf{\TITOLODOC}
			\\
	\end{Huge}
\end{center}

%\setlength{\parindent}{18pt} %settato indentazione di default 
%\section*{\LARGE Sommario:} %SEZIONE SOMMARIO
%\indent \indent
%Glossario

\section*{\LARGE Stato del documento:}
\indent \indent
	Formale Esterno

\section*{\LARGE Redazione:}
	\begin{table}[!h]
		\begin{center}
			\begin{tabular}
				{|c|c|}
				\hline
				%%%%%%%%%%%%%%INTESTAZIONE COLONNE%%%%%%%%%%%%%%%%%%%%%%%%%%%%%
				\multicolumn{2}{|c|}{ \textbf{Redazione} } \\
				\hline
				\textbf{Fase} & \textbf{Redattori} \\
				%%%%%%%%%%%%%%FINE INTESTAZIONE COLONNE%%%%%%%%%%%%%%%%%%%%%%%%%%%%%%%
				\hline
				%%%%%%%%%%% PARTE DA MODIFICARE %%%%%%%%%%%%%%%%%%%%%%%%%%%		
				\multirow{2}{*}{Pre-RR} &  Bizzotto Piero \\ & Geremia Mirco \\
				\hline
				\multirow{2}{*}{RR-RPP} &  Cunico Marco \\ & Dissegna Stefano \\
				\hline
				{RPP-RQ} &  Cunico Marco \\ 
				\hline
				%%%%%%%%%%% FINE PARTE DA MODIFICARE %%%%%%%%%%%%%%%%%%%%%%%%%
			\end{tabular}
			\caption{Lista Redattori} %INSERIRE DIDASCALIA - SE NECESSARIA - 
			\label{tabredazione}
		\end{center}
	\end{table}

\section*{\LARGE Verifica:}

\begin{table}[!h]
	\begin{center}
		\begin{tabular}
			{|c|c|}
			\hline
			%%%%%%%%%%%%%%INTESTAZIONE COLONNE%%%%%%%%%%%%%%%%%%%%%%%%%%%%%%%%%%%
			\multicolumn{2}{|c|}{ \textbf{Verifica} } \\
			\hline
			\textbf{Fase} & \textbf{Verificatori} \\
			%%%%%%%%%%%%%%FINE INTESTAZIONE COLONNE%%%%%%%%%%%%%%%%%%%%%%%
			\hline
			%%%%%%%%%%% PARTE DA MODIFICARE %%%%%%%%%%%%%%%%%%%%%%%%%%%%%%%%%%%%%%%		
			{Pre-RR} &  Dissegna Stefano \\
			\hline
			{RR-RPP} & Geremia Mirco \\
			\hline
			{RPP-RQ} & Carollo Mirko\\
			\hline
			%%%%%%%%%%% FINE PARTE DA MODIFICARE %%%%%%%%%%%%%%%%%%%%%%%%%%%%%%%%%%%
		\end{tabular}
		\caption{Lista Verificatori} %INSERIRE DIDASCALIA - SE NECESSARIA - 
		\label{tabverifica}
	\end{center}
\end{table}

\newpage 
	
\section*{\LARGE Approvazione:}

\begin{table}[!h]
	\begin{center}
		\begin{tabular}
			{|c|c|}
			\hline
			%%%%%%%%%%%%%%INTESTAZIONE COLONNE%%%%%%%%%%%%%%%%%%%%%%%%%%%%%%%%%%%
			\multicolumn{2}{|c|}{ \textbf{Approvazione} } \\
			\hline
			\textbf{Fase} & \textbf{Approvatori} \\
			%%%%%%%%%%%%%%FINE INTESTAZIONE COLONNE%%%%%%%%%%%%%%%%%%%%%%%
			\hline
			%%%%%%%%%%% PARTE DA MODIFICARE %%%%%%%%%%%%%%%%%%%%%%%%%%%%%%%%%%%%%%%		
			{Pre-RR} &  Dal Bosco Davide \\
			\hline
			{RR-RPP} & Bizzotto Piero \\
			\hline
			{RPP-RQ} & Rizzo Maurizio\\
			\hline
			%%%%%%%%%%% FINE PARTE DA MODIFICARE %%%%%%%%%%%%%%%%%%%%%%%%%%%%%%%%%%%
		\end{tabular}
		\caption{Lista Approvatori} %INSERIRE DIDASCALIA - SE NECESSARIA - 
		\label{tabapprovazione}
	\end{center}
\end{table}
\textbf{}

\section*{\LARGE Lista di Distribuzione:}

	\begin{elenconumerato}{\normindent}
		\item WebShape 
     	\item I committenti Conte Renato e Vardanega Tullio in rappresentanza \\  dell'azienda proponente Zucchetti SPA
	\end{elenconumerato}

\newpage


\section*{\LARGE Registro delle Modifiche:}

\begin{center}
	\begin{table}[h]
		  \begin{tabular*}
			{1\textwidth}%
				{@{\extracolsep{\fill}}|p{0.1\textwidth}|p{0.54\textwidth}|p{0.26\textwidth}|}
			 \hline
%%%%%%%%%%%%%%INTESTAZIONE COLONNE%%%%%%%%%%%%%%%%%%%%%%%%%%%%%%%%%%%%%%%%%%%%%%%%%%%%%%%%%%%%%%
			\textbf{Versione}  & \textbf{Descrizione} & \textbf{Autore} \\
%%%%%%%%%%%%%%FINE INTESTAZIONE COLONNE%%%%%%%%%%%%%%%%%%%%%%%%%%%%%%%%%%%%%%%%%%%%%%%%%%%%%%%%%%%%%%
		 \hline
%%%%%%%%%%% PARTE DA MODIFICARE %%%%%%%%%%%%%%%%%%%%%%%%%%%%%%%%%%%%%%%%%%%%%%%%%%%%%%%%%%%%%%%%%
			3.0 & 07/03/2009 Rilascio per RQ & Cunico Marco\\
			\hline
			2.2 & 06$\slash$03$\slash$2009 Aggiunta altri termini & Cunico Marco \\
			\hline
			2.1 & 10$\slash$02$\slash$2009 Aggiunta altri termini & Cunico Marco \\
			\hline
            2.0 & 24$\slash$01$\slash$2009 Aggiunto termine, rilascio per RPP & Dissegna Stefano \\
			\hline
		    1.2 & 22$\slash$01$\slash$2009  Aggiunta altri termini & Cunico Marco \\
			\hline
		    1.1 & 20$\slash$01$\slash$2009  Aggiunta altri termini & Cunico Marco \\
			\hline
			1.0 & 09$\slash$12$\slash$2008  Verifica finale in preparazione al rilascio & Bizzotto Piero \\
			\hline
			0.3  &    08$\slash$12$\slash$2008 Aggiunta altri termini & Geremia Mirco \\
			\hline
			0.2&    06$\slash$12$\slash$2008 Aggiunta altri termini & Geremia Mirco \\
			\hline
    	 	0.1 &	 05$\slash$12$\slash$2008 Aggiunta termini da AnalisiDeiRequisiti & Bizzotto Piero\\
    	 	\hline
    	 	0.0 & 	 04$\slash$12$\slash$2008 Struttura e aggiunta termini da NormeDiProgetto & Geremia Mirco \\

		\hline %%FINE RIGA
%%%%%%%%%%% FINE PARTE DA MODIFICARE %%%%%%%%%%%%%%%%%%%%%%%%%%%%%%%%%%%%%%%%%%%%%%%%%%%%%%%%%%%
		\end{tabular*}
	\caption{didascalia tabella 	MODIFICHE} %INSERIRE DIDASCALIA - SE NECESSARIA - 
	\label{tab:modifiche}
	\end{table}
\end{center}

\newpage

\section*{A}
\hypertarget{ajax}{}
\textbf{Ajax:}
Ajax \` e un acronimo di Asynchronous \hyperlink{javascript}{\underline{JavaScript}} and \hyperlink{xml}{\underline{XML}}, \` e uno strumento di sviluppo per la realizzazione di applicazioni web interattive. La tecnologia AJAX si basa su uno scambio di dati in background fra web browser e server, che consente l'aggiornamento dinamico di una pagina web senza esplicito ricaricamento da parte dell'utente. Normalmente le funzioni richiamate sono scritte con il linguaggio JavaScript.\\

\section*{B}
\hypertarget{browser}{}
\textbf{Browser:}
Un browser web \` e un programma che consente agli utenti di visualizzare e interagire con testi, immagini e altre informazioni, tipicamente contenute in una pagina web di un sito (o all'interno di una rete locale).
Il browser \` e in grado di interpretare il codice \hyperlink{html}{\underline{HTML}} e visualizzarlo in forma di ipertesto. Il browser consente perci\` o la navigazione nel web.\\

\hypertarget{bytecode}{}
\textbf{Bytecode:}
Il Bytecode \` e un linguaggio intermedio pi\` u astratto del linguaggio macchina, usato per descrivere le operazioni che costituiscono un programma. \` E chiamato cos\` i perch\` e spesso le operazioni hanno un codice che occupa un solo byte, anche se la lunghezza dell'intera istruzione pu\` o variare perch\` e ogni operazione ha un numero specifico di parametri su cui operare.\\

\section*{C}
\hypertarget{canvas}{}
\textbf{Canvas:}
Canvas \` e una estensione dell'\hyperlink{html}{\underline{HTML}} standard che permette il rendering dinamico di immagini bitmap gestibili attraverso un linguaggio di scripting.  Consiste in una regione disegnabile, definita in codice HTML con gli attributi \textit{height} and \textit{width}. Il codice \hyperlink{javascript}{\underline{JavaScript}} pu\`o accedere all'area con un set completo di funzioni per il disegno, simili a quelle comuni ad altre API 2D, permettendo cos\`i la generazione dinamica di disegni.\\

\hypertarget{client}{}
\textbf{Client:}
Client indica una componente che accede ai servizi o alle risorse di un'altra componente, detta \hyperlink{server}{\underline{server}}.\\

\section*{D}
\hypertarget{dom}{}
\textbf{DOM:}
DOM(Document Object Model) \` e una forma di rappresentazione dei documenti strutturati come modello orientato agli oggetti.\\

\section*{E}
\hypertarget{ecmascript}{}
\textbf{ECMAScript:}
\text Si veda \hyperlink{javascript}{\underline{JavaScript}}.\\

\section*{G}
\hypertarget{gantt}{}
\textbf{Gantt:}
Il Diagramma di Gantt \` e uno strumento di supporto alla gestione dei progetti a rappresentazione temporale delle mansioni o attivit\`a che costituiscono il progetto. \\

\hypertarget{grmfrm}{}
\textbf{Grammatica formale:}
Una Grammatica formale \` e un sistema di regole che delineano matematicamente un insieme (di solito infinito) di sequenze finite di simboli (stringhe) appartenenti ad un alfabeto anch'esso finito.\\

\section*{H}
\hypertarget{html}{}
\textbf{HTML:}
HTML \` e un linguaggio usato per descrivere la struttura dei documenti ipertestuali disponibili nel \hyperlink{www}{\underline{WWW}}.
L'HTML non \` e un linguaggio di programmazione, ma un \hyperlink{linguaggio di markup}{\underline{linguaggio di markup}}, ossia descrive il contenuto, testuale e non, di una pagina web.\\

\hypertarget{html5}{}
\textbf{HTML 5:}
HTML 5 \` e un \hyperlink{linguaggio di markup}{\underline{linguaggio di markup}} per la progettazione delle pagine web attualmente in fase di definizione, presso il \hyperlink{w3c}{\underline{W3C}}.\\ 

\section*{J}
\hypertarget{java}{}
\textbf{Java:}
Java \`e un linguaggio di programmazione orientato agli oggetti. La piattaforma di programmazione Java \`e fondata sul linguaggio stesso, sulla \hyperlink{jvm}{\underline{JVM}} e sulle API Java.\\

\hypertarget{javascript}{}
\textbf{JavaScript:}
JavaScript (o ECMAScript) \`e un linguaggio di scripting orientato agli oggetti comunemente usato nei siti web. \`E  uno standard ISO.\\

\hypertarget{JIT}{}
\textbf{JIT:}
Acronimo di Just In Time: si riferisce alla compilazione al volo di un programma un attimo prima della sua esecuzione.\\

\hypertarget{jsp}{}
\textbf{JSP:}
JSP(JavaServer Pages) \`e una tecnologia \hyperlink{java}{\underline{Java}} per lo sviluppo di applicazioni \hyperlink{web}{\underline{Web}} che forniscono contenuti dinamici in formato \hyperlink{html}{\underline{HTML}} o \hyperlink{xml}{\underline{XML}}.\\

\hypertarget{jvm}{}
\textbf{JVM:}
JVM(Java Virtual Machine) \`e la macchina virtuale che esegue i programmi in linguaggio Java \hyperlink{bytecode}{\underline{bytecode}}, ovvero i prodotti della compilazione dei sorgenti Java.\\

\section*{L}
\hypertarget{latex}{}
\textbf{\LaTeX:}
\LaTeX \ \`e un linguaggio di markup usato per la redazione di documentii che permette di gestire in maniera semplice ed automatica la formattazione del testo.\\

\textbf{Linguaggio di markup:}
\hypertarget{markup}{}
Un linguaggio di markup \`e un linguaggio che stabilisce delle convenzioni standard e descrive dei meccanismi per la rappresentazione del testo.\\

\section*{O}
\hypertarget{opensource}{}
\textbf{Open Source:}
Con il termine Open Source si indica un tipo di licenza software in cui il codice sorgente viene reso disponibile; lo scopo dell'Open Source \`e quello di favorire la collaborazione tra gli sviluppatori e il progresso del software tramite la condivisione di idee.\\

\section*{P}
\hypertarget{parser}{}
\textbf{Parser:}
Il Parser \` e un programma che esegue il \hyperlink{parsing}{\underline{parsing}} o analisi sintattica.\\

\hypertarget{parsing}{}
\textbf{Parsing:}
Il Parsing \` e il processo atto ad analizzare uno stream continuo in input in modo da determinare la sua struttura grammaticale grazie ad una data \hyperlink{grmfrm}{\underline{grammatica formale}}.\\

\hypertarget{png}{}
\textbf{PNG:}
Il PNG (Portable Network Graphics) \`e un formato di file che permette di memorizzare immagini senza perdere informazioni.\\

\section*{R}
\hypertarget{rgb}{}
\textbf{RGB:}
RGB (Red, Green, Blue) \`e  il nome di un modello di colori di tipo additivo che si basa sui tre colori rosso, verde e blu, da cui appunto il nome RGB, da non confondere con i colori primari sottrattivi giallo, ciano e magenta (chiamati anche giallo, rosso e blu).
Un'immagine pu\`o  infatti essere scomposta, attraverso filtri o altre tecniche, in questi colori base che, miscelati tra loro, danno quasi tutto lo spettro dei colori visibili, con l'eccezione delle porpore.\\

\section*{S}
\hypertarget{server}{}
\textbf{Server:}
Il server \`e una componente informatica software o hardware che fornisce servizi ad altre componenti chiamate \hyperlink{client}{\underline{client}} attraverso una rete. \\

\hypertarget{serverhttp}{}
\textbf{Server HTTP:}
Si veda \hyperlink{serverweb}{\underline{Server Web}}.\\

\hypertarget{serverweb}{}
\textbf{Server Web:}
Un Server Web (o Server HTTP) \` e un processo, e per estensione il computer su cui \` e in esecuzione, che si occupa di fornire, su richiesta del browser, una pagina web. Le informazioni inviate dal web server viaggiano in rete trasportate dal protocollo HTTP. L'insieme di server web d\` a vita al \hyperlink{www}{\underline{WWW}}.\\ 

\hypertarget{servlet}{}
\textbf{Servlet:}
Le servlet sono oggetti che operano all'interno di un server per applicazioni e potenziano le sue funzionalit\` a.\\

\hypertarget{svg}{}
\textbf{SVG:}
L' SVG (Scalable Vector Graphics) indica una tecnologia in grado di visualizzare oggetti di grafica vettoriale e, per questo motivo, di gestire immagini scalabili dimensionalmente.\\

\section*{T}
\hypertarget{tabhash}{}
\textbf{Tabella di hash:}
Una Tabella di hash \` e una struttura dati usata per mettere in corrispondenza una data chiave con un dato valore. Pu\` o usare qualsiasi tipo di dato come indice.\\

\hypertarget{tomcat}{}
\textbf{Tomcat:}
Tomcat \` e un web container \hyperlink{opensource}{\underline{open source}} sviluppato dalla Apache Software Foundation. Implementa le specifiche \hyperlink{jsp}{\underline{JSP}} e \hyperlink{servlet}{\underline{Servlet}} di Sun Microsystems, fornendo quindi una piattaforma per l'esecuzione di applicazioni Web sviluppate nel linguaggio \hyperlink{java}{\underline{Java}}. La sua distribuzione standard include anche le funzionalit\` a di \hyperlink{serverweb}{\underline{web server}} tradizionale.\\

\hypertarget{toolbar}{}
\textbf{Toolbar:}
La toolbar (lett. barra degli strumenti) \` e un componente (\hyperlink{widget}{\underline{widget}}) molto diffuso nelle interfacce utente. \` E una barra orizzontale o verticale, o un box, che raccoglie, sotto forma di icone o pulsanti, i collegamenti alle funzioni pi\` u usate di un software.\\


\section*{W}
\hypertarget{w3c}{}
\textbf{W3C:}
Il W3C \`e un'associazione internazionale fondata nel 1994 con lo scopo di migliorare i protocolli ed i linguaggi del Web. Pur non essendo un organismo di standardizzazione, le raccomandazioni fornite dal W3C sono spesso trattate come standard.\\

\hypertarget{web}{}
\textbf{Web:}
Si veda \hyperlink{www}{\underline{WWW}}.\\

\hypertarget{widget}{}
\textbf{Widget:}
Un widget \` e un elemento (tipicamente grafico) di una interfaccia utente di un programma che facilita all'utente l'interazione con il programma stesso. In italiano \`  e detto  elemento grafico; pu\` o essere una vera e propria miniapplicazione.\\

\hypertarget{wiki}{}
\textbf{Wiki:}
Un wiki \`e un sistema che permette di gestire pagine web in maniera collaborativa favorendo la codivisione di informazioni tra gli utenti; in un wiki ognuno pu\`o aggiungere o modificare i contenuti di una pagina attraverso il \hyperlink{browser}{\underline{browser}}.\\

\hypertarget{www}{}
\textbf{WWW:}
WWW (World Wide Web) \`e uno dei servizi di Internet, la pi\`u grande rete di computer mondiale e ad accesso pubblico attualmente esistente.
Creata da Tim Berners-Lee, il servizio mette a disposizione degli utenti uno spazio elettronico e digitale per la pubblicazione di contenuti multimediali oltre che un mezzo per la distribuzione di software e la fornitura di servizi particolari sviluppati dagli stessi utenti. \\

\section*{X}
\hypertarget{xml}{}
\textbf{XML}
\text Formato estensibile raccomandato dal \hyperlink{w3c}{\underline{W3C}} per la rappresentazione di dati semistrutturati.\\

\end{document}
