	\input{../TeX/base} %BASE!!!
\usepackage{multirow}
\title{\TITOLODOC}
\author{Dal Bosco Davide}

\begin{document}

\renewcommand{\insertversion}{0.0} %INSERIRE LA VERSIONE QUI DENTRO STILE x.x.xx
\renewcommand{\TITOLODOC}{Specifica Tecnica} %INSERIRE IL TITOLO DEL DOCUMENTO DA FAR COMPARIRE A PIE PAGINA
\renewcommand{\glosspath}{.\glossario} %INSERIRE PERCORSO RELATIVO

%%%%%%%%%%%%%%%%%%%%%%PARTE DA NON MODIFICARE%%%%%%%%%%%%%%%%%
\begin{titlepage}
\begin{center}
	\begin{Large}	\today \end{Large}
\end{center}

\vspace{20pt}

\begin{center}
	\begin{Huge}
				\textbf{\ajax}
	\end{Huge}
\end{center}			

\begin{center}
	\begin{large}
				\textbf{Software per il Disegno Grafico\\ in Tecnologie Web}
	\end{large}
\end{center}			

\vspace{20pt}

\begin{center}
\includegraphics[width=150pt]{../logo/logo}
\end{center}

\vspace{170pt}
\begin{center} %INSERIRE ALL'INTERNO IL TITOLO DOCUMENTO CHE COMPARIRA NELLA PAGINA INIZIALE				
	\begin{Huge}
				\textbf{\TITOLODOC}
	\end{Huge}
			\\
\end{center}
\vspace{210pt}
\begin{center}
Versione: \insertversion
\end{center}
\end{titlepage}

\newpage
%%%%%%%%%%%%%%%%%%%%%%FINE PARTE DA NON MODIFICARE%%%%%%%%%%%%%%%%%

\begin{center} %INSERIRE ALL'INTERNO IL TITOLO DOCUMENTO CHE COMPARIRA NELLA PAGINA INIZIALE
	\begin{Huge}	
				\textbf{\TITOLODOC}
			\\
	\end{Huge}
\end{center}

%\setlength{\parindent}{18pt} %settato indentazione di default 
\section*{\LARGE Sommario:}
Il presente documento descrive le scelte tecniche, architetturali e tecnologiche effettuate dall'azienda WebShape per lo sviluppo del Capitolato C04.

 %SEZIONE SOMMARIO
\indent \indent

\section*{\LARGE Stato del documento:}
\indent \indent
	Formale Esterno

\section*{\LARGE Redazione:}
	\begin{table}[!h]
		\begin{center}
			\begin{tabular}
				{|c|c|}
				\hline
				%%%%%%%%%%%%%%INTESTAZIONE COLONNE%%%%%%%%%%%%%%%%%%%%%%%%%%%%%%%%
				\multicolumn{2}{|c|}{ \textbf{Redazione} } \\
				\hline
				\textbf{Fase} & \textbf{Redattori} \\
				%%%%%%%%%%%%%%FINE INTESTAZIONE COLONNE%%%%%%%%%%%%%%%%%%%%%%%%%%%%%%%%%%%%%%
				\hline
				%%%%%%%%%%% PARTE DA MODIFICARE %%%%%%%%%%%%%%%%%%%%%%%%%%%%%%%%%%%%%%%%%%		
				\multirow{3}{*}{RR-RPP} & Bizzotto Piero\\
										& Dal Bosco Davide\\
										& Dissegna Stefano\\
				\hline
				%%%%%%%%%%% FINE PARTE DA MODIFICARE %%%%%%%%%%%%%%%%%%%%%%%%%%%
			\end{tabular}
			\caption{Lista Redattori} %INSERIRE DIDASCALIA - SE NECESSARIA - 
			\label{tabredazione}
		\end{center}
	\end{table}
	
\newpage	
	
\section*{\LARGE Approvazione:}
\begin{table}[!h]
	\begin{center}
		\begin{tabular}
			{|c|c|}
			\hline
			%%%%%INTESTAZIONE COLONNE%%%%%%%%%%%%%%%%%%%%%%%%%%%%%%%
			\multicolumn{2}{|c|}{ \textbf{Approvazione} } \\
			\hline
			\textbf{Fase} & \textbf{Approvatori} \\
			%%%%%%%%%%%%%%FINE INTESTAZIONE COLONNE%%%%%%%%%%%%%%%%%%%%%%%%%%%%%%
			\hline
			%%%%%%%%%%% PARTE DA MODIFICARE %%%%%%%%%%%%%%%%%%%%%%%%%%%%%%%%%%%%%%		
			\multirow{2}{*}{RR-RPP} & \\
									& \\
			\hline
			%%%%%%%%%%% FINE PARTE DA MODIFICARE %%%%%%%%%%%%%%%%%%%%%%%%%%%%%%%%%%%
		\end{tabular}
		\caption{Lista Approvatori} %INSERIRE DIDASCALIA - SE NECESSARIA - 
		\label{tabapprovazione}
	\end{center}
\end{table}

\textbf{}

\section*{\LARGE Lista di Distribuzione:}

	\begin{elenconumerato}{\normindent}
		\item WebShape 
		\item I committenti Conte Renato e Vardanega Tullio in rappresentanza \\  dell'azienda proponente Zucchetti SPA
	\end{elenconumerato}

\newpage



\section*{\Large Registro delle Modifiche:}


\begin{center}
	\begin{table}[h]
		  \begin{tabular*}
			{1\textwidth}%
				{@{\extracolsep{\fill}}|p{0.1\textwidth}|p{0.54\textwidth}|p{0.26\textwidth}|}
			 \hline
%%%%%%%%%%%%%%INTESTAZIONE COLONNE%%%%%%%%%%%%%%%%%%%%%%%%%%%%%%%%%%%%%%%%%%
			\textbf{Versione}  & \textbf{Descrizione} & \textbf{Autore} \\
%%%%%%%%%%%%%%FINE INTESTAZIONE COLONNE%%%%%%%%%%%%%%%%%%%%%%%%%%%%%%%%%%%%%%%
		 \hline
%%%%%%%%%%% PARTE DA MODIFICARE %%%%%%%%%%%%%%%%%%%%%%%%%%%%%%%%%%%%%%%%%%%
    	  0.0 & 15/01/2009 Bozza iniziale & Dal Bosco Davide \\
		\hline %%FINE RIGA
%%%%%%%%%%% FINE PARTE DA MODIFICARE %%%%%%%%%%%%%%%%%%%%%%%%%%%%%%%%%%%%%
		\end{tabular*}
	\caption{Registro delle modifiche} %INSERIRE DIDASCALIA - SE NECESSARIA - 
	\label{tab:modifiche}
	\end{table}
\end{center}


\newpage
\thispagestyle{fancy}
\tableofcontents
\thispagestyle{fancy}
\newpage

\sezione{Introduzione}

\subsezione{Scopo del documento}
Il documento si propone di presentare ai Committenti le scelte riguardanti la progettazione architetturale del capitolato d'appalto C04. In particolare viene presentata la struttura dei package e le loro relazioni, le unit\`a che compongono i package stessi e uno schema generale dell'interfaccia grafica.

\sezione{Scopo del prodotto}
Il software AJAXDRAW \`e proposto per verificare e dimostrare la fattibilit\`a di realizzazione di un'applicazione di disegno grafico, in grado di poter elaborare figure vettoriali primitive e complesse utilizzando le tecnologie web.

\subsezione{Glossario}
All'interno del documento \textit{Glossario}, sono presentati i termini tecnici utilizzati in tutti i documenti. Il glossario \`e fornito in allegato al presente documento.
\subsubsezione{Definizioni}
\subsubsezione{Acronimi}
\subsubsezione{Abbreviazioni}
\subsezione{Riferimenti}
Il presente documento \`e redatto utilizzando le convenzioni inserite nel documento \textit{NormeDiProgetto}, consultabile dal repository pubblico al quale WebShape si appoggia per i suoi progetti.
\subsubsezione{Normativi}
\subsubsezione{Informativi}
\sezione{Definizione del prodotto}
\subsezione{Metodo e formalismo di specifica}
L'architettura del sistema \`e stata sviluppata secondo un paradigma orientato agli oggetti. Le componenti del sistema e le loro relazioni verranno presentate in questo documento tramite diagrammi standard UML. Verranno presentate prima le componenti ad alto livello, le quali saranno poi descritte in maggiore dettaglio. Saranno inoltre descritti i pattern progettuali utilizzati.
\subsezione{Presentazione dell'architettura generale del sistema}
\subsezione{Identificazione dei componenti architetturali di alto livello}
\subsubsezione{Logica interna}
Questo sottosistema si divide in tre ulteriori sottosistemi
\subsubsubsezione{Gestione figure}
[inserire diagramma delle classi] \\
Si occupa della gestione interna delle figure disegnabili e manipolabili dall'utente e delle relative propriet\`a. Il termine \textit{figura} in questo contesto include anche i testi disegnati sul canvas. La classe \textit{Figure} rappresenta una figura qualsiasi. Ogni figura contiene un \textit{BoundingRectangle} che rappresenta la cornice entro cui la figura \`e completamente contenuta. Una figura pu\`o essere selezionata oppure no. Le classi che implementano \textit{Figure} sono responsabili di disegnarsi sul canvas implementando il metodo \textit{draw}. In questo metodo, non si dovr\`a tener conto dell'area correntemente visualizzata o del livello di zoom, in quanto gestiti in modo trasparente dalla classe \textit{Visualization}. Ad ogni figura \`e associato un insieme di punti significativi, come ad esempio i vertici in un quadrato, ottenibili tramite il metodo \textit{getMainPoints}. Ogni figura mantiene un colore del bordo, rappresentato dalla class \textit{Colour}. Le classi \textit{Circle} e \textit{Polygon} aggiungono un ulteriore colore per il riempimento, rappresentato dalla classe \textit{FillColour}. La classe \textit{Text} rappresenta un testo disegnato dall'utente, e mantiene delle propriet\`a aggregate dalla classe \textit{TextStyle}. A fini implementativi, le propriet\`a del testo e del colore andranno rappresentate secondo lo standard \textit{CSS}, quindi le suddette propriet\`a dovranno implementare l'interfaccia \textit{CSSProperty} che offre il metodo \textit{toCSS} responsabile della conversione dalla rappresentazione interna a quella \textit{CSS}. La classe \textit{FigureSet} mantiene l'insieme delle istanze di \textit{Figure} correntemente disegnate dall'utente e offre la possibilit\`a di selezionarne una a partire da un punto sul canvas.
\subsubsubsezione{Conversione in \textit{SVG}}
[inserire diagramma delle classi] \\
Si occupa di trasformare il \textit{FigureSet} corrente in una stringa 
contenente un documento SVG valido e viceversa. 
\paragraph{Da \textit{FigureSet} a SVG}
Sia \textit{FigureSet} sia \textit{Figure} implementano l'interfaccia \textit{SVGWritable}, che rappresenta un qualsiasi elemento, o insieme di elementi, convertibile in SVG. La conversione avviene tramite il metodo \textit{toSVG}. La conversione dipende dalla classe coinvolta. La classe \textit{FigureSet} dovr\`a creare un documento SVG vuoto e richiamare il metodo \textit{toSVG} su tutte le figure contenute, le quali genereranno gli elementi SVG necessari alla propria rappresentazione. Per creare un documento SVG si istanzia la classe \textit{SVGGenerator}, la quale offre dei metodi per generare comandi SVG. Quando tutte le figure hanno avuto la possibilit\`a di inserirsi nel documento SVG, il metodo \textit{flush} chiude il documento e lo ritorna sotto forma di stringa.
\paragraph{Da SVG a \textit{FigureSet}}

\sezione{Descrizione dei singoli componenti}
\subsezione{Tipo, obiettivo e funzione del componente}
\subsezione{Relazioni d'uso di altre componenti}
\subsezione{Interfacce e relazioni di uso da altre componenti}
\subsezione{Attivit\`a svolte e dati trattati}
\sezione{Stime di fattibilit\`a e di bisogno di risorse}
\sezione{Tracciamento della relazione componenti-requisiti}
			

\end{document}
