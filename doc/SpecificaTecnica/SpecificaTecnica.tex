	\input{../TeX/base} %BASE!!!
\usepackage{multirow}
\title{\TITOLODOC}
\author{Dal Bosco Davide}

\begin{document}

\renewcommand{\insertversion}{0.0} %INSERIRE LA VERSIONE QUI DENTRO STILE x.x.xx
\renewcommand{\TITOLODOC}{Specifica Tecnica} %INSERIRE IL TITOLO DEL DOCUMENTO DA FAR COMPARIRE A PIE PAGINA
\renewcommand{\glosspath}{.\glossario} %INSERIRE PERCORSO RELATIVO

%%%%%%%%%%%%%%%%%%%%%%PARTE DA NON MODIFICARE%%%%%%%%%%%%%%%%%
\begin{titlepage}
\begin{center}
	\begin{Large}	\today \end{Large}
\end{center}

\vspace{20pt}

\begin{center}
	\begin{Huge}
				\textbf{\ajax}
	\end{Huge}
\end{center}			

\begin{center}
	\begin{large}
				\textbf{Software per il Disegno Grafico\\ in Tecnologie Web}
	\end{large}
\end{center}			

\vspace{20pt}

\begin{center}
\includegraphics[width=150pt]{../logo/logo}
\end{center}

\vspace{170pt}
\begin{center} %INSERIRE ALL'INTERNO IL TITOLO DOCUMENTO CHE COMPARIRA NELLA PAGINA INIZIALE				
	\begin{Huge}
				\textbf{\TITOLODOC}
	\end{Huge}
			\\
\end{center}
\vspace{190pt}
\begin{center}
Versione: \insertversion
\end{center}
\end{titlepage}

\newpage
%%%%%%%%%%%%%%%%%%%%%%FINE PARTE DA NON MODIFICARE%%%%%%%%%%%%%%%%%

\begin{center} %INSERIRE ALL'INTERNO IL TITOLO DOCUMENTO CHE COMPARIRA NELLA PAGINA INIZIALE
	\begin{Huge}	
				\textbf{\TITOLODOC}
			\\
	\end{Huge}
\end{center}

%\setlength{\parindent}{18pt} %settato indentazione di default 
\section*{\LARGE Sommario:}
Il presente documento descrive le scelte tecniche, architetturali e tecnologiche effettuate dall'azienda WebShape per lo sviluppo del Capitolato C04.

 %SEZIONE SOMMARIO
\indent \indent

\section*{\LARGE Stato del documento:}
\indent \indent
	Formale Esterno

\section*{\LARGE Redazione:}
	\begin{table}[!h]
		\begin{center}
			\begin{tabular}
				{|c|c|}
				\hline
				%%%%%%%%%%%%%%INTESTAZIONE COLONNE%%%%%%%%%%%%%%%%%%%%%%%%%%%%%%%%
				\multicolumn{2}{|c|}{ \textbf{Redazione} } \\
				\hline
				\textbf{Fase} & \textbf{Redattori} \\
				%%%%%%%%%%%%%%FINE INTESTAZIONE COLONNE%%%%%%%%%%%%%%%%%%%%%%%%%%%%%%%%%%%%%%
				\hline
				%%%%%%%%%%% PARTE DA MODIFICARE %%%%%%%%%%%%%%%%%%%%%%%%%%%%%%%%%%%%%%%%%%		
				\multirow{3}{*}{RR-RPP} & Bizzotto Piero\\
										& Dal Bosco Davide\\
										& Dissegna Stefano\\
				\hline
				%%%%%%%%%%% FINE PARTE DA MODIFICARE %%%%%%%%%%%%%%%%%%%%%%%%%%%
			\end{tabular}
			\caption{Lista Redattori} %INSERIRE DIDASCALIA - SE NECESSARIA - 
			\label{tabredazione}
		\end{center}
	\end{table}
	
\newpage	
	
\section*{\LARGE Approvazione:}
\begin{table}[!h]
	\begin{center}
		\begin{tabular}
			{|c|c|}
			\hline
			%%%%%INTESTAZIONE COLONNE%%%%%%%%%%%%%%%%%%%%%%%%%%%%%%%
			\multicolumn{2}{|c|}{ \textbf{Approvazione} } \\
			\hline
			\textbf{Fase} & \textbf{Approvatori} \\
			%%%%%%%%%%%%%%FINE INTESTAZIONE COLONNE%%%%%%%%%%%%%%%%%%%%%%%%%%%%%%
			\hline
			%%%%%%%%%%% PARTE DA MODIFICARE %%%%%%%%%%%%%%%%%%%%%%%%%%%%%%%%%%%%%%		
			\multirow{2}{*}{RR-RPP} & \\
									& \\
			\hline
			%%%%%%%%%%% FINE PARTE DA MODIFICARE %%%%%%%%%%%%%%%%%%%%%%%%%%%%%%%%%%%
		\end{tabular}
		\caption{Lista Approvatori} %INSERIRE DIDASCALIA - SE NECESSARIA - 
		\label{tabapprovazione}
	\end{center}
\end{table}

\textbf{}

\section*{\LARGE Lista di Distribuzione:}

	\begin{elenconumerato}{\normindent}
		\item WebShape 
		\item I committenti Conte Renato e Vardanega Tullio in rappresentanza \\  dell'azienda proponente Zucchetti SPA
	\end{elenconumerato}

\newpage



\section*{\Large Registro delle Modifiche:}


\begin{center}
	\begin{table}[h]
		  \begin{tabular*}
			{1\textwidth}%
				{@{\extracolsep{\fill}}|p{0.1\textwidth}|p{0.54\textwidth}|p{0.26\textwidth}|}
			 \hline
%%%%%%%%%%%%%%INTESTAZIONE COLONNE%%%%%%%%%%%%%%%%%%%%%%%%%%%%%%%%%%%%%%%%%%
			\textbf{Versione}  & \textbf{Descrizione} & \textbf{Autore} \\
%%%%%%%%%%%%%%FINE INTESTAZIONE COLONNE%%%%%%%%%%%%%%%%%%%%%%%%%%%%%%%%%%%%%%%
		 \hline
%%%%%%%%%%% PARTE DA MODIFICARE %%%%%%%%%%%%%%%%%%%%%%%%%%%%%%%%%%%%%%%%%%%
    	  0.0 & 15/01/2009 Bozza iniziale & Dal Bosco Davide \\
		\hline %%FINE RIGA
%%%%%%%%%%% FINE PARTE DA MODIFICARE %%%%%%%%%%%%%%%%%%%%%%%%%%%%%%%%%%%%%
		\end{tabular*}
	\caption{Registro delle modifiche} %INSERIRE DIDASCALIA - SE NECESSARIA - 
	\label{tab:modifiche}
	\end{table}
\end{center}


\newpage
\thispagestyle{fancy}
\tableofcontents
\thispagestyle{fancy}
\newpage

\sezione{Introduzione}

\subsezione{Scopo del documento}
Il documento si propone di presentare ai Committenti le scelte riguardanti la progettazione architetturale del capitolato d'appalto C04. In particolare viene presentata la struttura dei package e le loro relazioni, le unit\`a che compongono i package stessi e uno schema generale dell'interfaccia grafica.

\sezione{Scopo del prodotto}
Il software AJAXDRAW \`e proposto per verificare e dimostrare la fattibilit\`a di realizzazione di un'applicazione di disegno grafico, in grado di poter elaborare figure vettoriali primitive e complesse utilizzando le tecnologie web.

\subsezione{Glossario}
All'interno del documento \textit{Glossario}, sono presentati i termini tecnici utilizzati in tutti i documenti. Il glossario \`e fornito in allegato al presente documento.
\subsubsezione{Definizioni}
\subsubsezione{Acronimi}
\subsubsezione{Abbreviazioni}
\subsezione{Riferimenti}
Il presente documento \`e redatto utilizzando le convenzioni inserite nel documento \textit{NormeDiProgetto}, consultabile dal repository pubblico al quale WebShape si appoggia per i suoi progetti.
\subsubsezione{Normativi}
\subsubsezione{Informativi}
\sezione{Definizione del prodotto}
\subsezione{Metodo e formalismo di specifica}
L'architettura del sistema \`e stata sviluppata secondo un paradigma orientato agli oggetti. Le componenti del sistema e le loro relazioni verranno presentate in questo documento tramite diagrammi standard UML. Verranno presentate prima le componenti ad alto livello, le quali saranno poi descritte in maggiore dettaglio. Saranno inoltre descritti i pattern progettuali utilizzati.
\subsezione{Presentazione dell'architettura generale del sistema}
\subsezione{Identificazione dei componenti architetturali di alto livello}
\subsubsezione{Logica interna}
Questo sottosistema si divide in tre ulteriori sottosistemi
\subsubsezione{Gestione figure}
[inserire diagramma delle classi] \\
Si occupa della gestione interna delle figure disegnabili e manipolabili dall'utente e delle relative propriet\`a. Il termine \textit{figura} in questo contesto include anche i testi disegnati sul canvas. La classe \textit{Figure} rappresenta una figura qualsiasi. Una figura \`e caratterizzata da delle propriet\`a derivate dalla classe \textit{Property}. La suddetta classe contiene un metodo \textit{createWidget} per fabbricare un elemento grafico che permette all'utente di modificare la propriet\`a che lo ha generato. Ogni figura contiene un \textit{BoundingRectangle} che rappresenta la cornice entro cui la figura \`e completamente contenuta. Una figura pu\`o essere selezionata oppure no. Le classi che implementano \textit{Figure} sono responsabili di disegnarsi sul canvas implementando il metodo \textit{draw}. In questo metodo, non si dovr\`a tener conto dell'area correntemente visualizzata o del livello di zoom, in quanto gestiti in modo trasparente dalla classe \textit{Visualization}. Ad ogni figura \`e associato un insieme di punti significativi, come ad esempio i vertici in un quadrato, ottenibili tramite il metodo \textit{getMainPoints}. Ogni figura mantiene un colore del bordo, rappresentato dalla classe \textit{Colour}. Le classi \textit{Circle}, \textit{Polygon} e \textit{Rectangle} aggiungono un ulteriore colore per il riempimento, rappresentato dalla classe \textit{FillColour}. La classe \textit{Text} rappresenta un testo disegnato dall'utente, e mantiene delle propriet\`a derivate dalla classe \textit{TextProperty}. A fini implementativi, le propriet\`a del testo e del colore andranno rappresentate secondo lo standard \textit{CSS}, quindi le suddette propriet\`a dovranno implementare l'interfaccia \textit{CSSProperty} che offre il metodo \textit{toCSS} responsabile della conversione dalla rappresentazione interna a quella \textit{CSS}. La classe \textit{FigureSet} mantiene l'insieme delle istanze di \textit{Figure} correntemente disegnate dall'utente e offre la possibilit\`a di selezionarne una a partire da un punto sul canvas.
\subsubsezione{Conversione in \textit{SVG}}
[inserire diagramma delle classi] \\
Si occupa di trasformare il \textit{FigureSet} corrente in una stringa 
contenente un documento SVG valido e viceversa. 
\paragraph{Da \textit{FigureSet} a SVG}
[inserire diagramma di stato]
Le classi derivanti da \textit{Figure} implementano l'interfaccia \textit{SVGWritable}, che rappresenta un qualsiasi elemento convertibile in SVG. La conversione avviene tramite il metodo \textit{toSVG}. La classe \textit{SVGWriter} dovr\`a, nel metodo \textit{write}, creare un documento SVG vuoto e richiamare il metodo \textit{toSVG} su tutte le figure contenute, le quali genereranno gli elementi SVG necessari alla propria rappresentazione. Per creare un documento SVG si istanzia la classe \textit{SVGGenerator}, la quale offre dei metodi per generare comandi SVG. Quando tutte le figure hanno avuto la possibilit\`a di inserirsi nel documento SVG, il metodo \textit{flush} chiude il documento e lo ritorna sotto forma di stringa.
\paragraph{Da SVG a \textit{FigureSet}}
[inserire diagramma di stato]
L'importazione di un file SVG viene gestita dalla classe \textit{SVGReader} attraverso il metodo \textit{read}, che riceve un documento SVG valido sotto forma di stringa e restituisce un'istanza di \textit{FigureSet} corrispondente. Viene fatto il parsing del documento come \textit{XML} attraverso la classe \textit{XMLParser}, e per ogni comando SVG viene creato un'istanza di \textit{SVGTag} che lo rappresenta. Il nome del tag verr\`a usato come indice dal metodo \textit{makeFigureClassFromTag} della classe \textit{SVGElementRegistry}, che costruir\`a un'istanza della classe corrispondente. Su quest'istanza verr\`a invocato il metodo \textit{fromSVG}, il quale \`e responsabile di inizializzare la classe a partire dal tag. Per associare una classe al nome di un tag si usa il metodo \textit{register}. La classe registrata deve implementare l'interfaccia \textit{SVGReadable}. 

\sezione{Descrizione dei singoli componenti}

\subsezione{Figure}
\subsubsezione{Tipo, obiettivo e funzione del componente}
Classe astratta che rappresenta una figura generica nel sistema, ovvero un oggetto disegnabile nel canvas.
\subsubsezione{Relazioni d'uso di altre componenti}
Mantiene un'istanza della classe \textit{BoundingRectangle} per rappresentare i limiti della figura. Contiene anche un'istanza di \textit{Colour} che rappresenta il colore del bordo. Deriva, ma non implementa, l'interfaccia \textit{SVGReadable} e l'interfaccia \textit{SVGWritable}, in quanto le sue sottoclassi dovranno essere serializzabili e deserializabili in file SVG.
\subsubsezione{Interfacce e relazioni di uso da altre componenti}
Viene aggregata dalla classe \textit{FigureSet}.
\subsubsezione{Attivit\`a svolte e dati trattati}
\begin{elencopuntato}[\normindent]
\item[-]  \textit{draw} metodo astratto per disegnare la figura sul canvas
\item[-]  \textit{getMainPoints} metodo astratto per ottenere i punti della figura manipolabili dall'utente.
\item[-]  \textit{setSelection} permette di selezionare/deselezionare la figura.
\item[-]  \textit{isSelected} dice se la figura \`e correntemente selezionata.
\item[-]  \textit{eachProperty} applica la funzione fn ad ogni propriet\`a della figura.
\end{elencopuntato}

\subsezione{FigureSet}
\subsubsezione{Tipo, obiettivo e funzione del componente}
Rappresenta una collezione di figure.
\subsubsezione{Relazioni d'uso di altre componenti}
Contiene le istanze di \textit{Figure} correntemente attive.
\subsubsezione{Interfacce e relazioni di uso da altre componenti}
Usata dalla classe \textit{SVGWriter} per serializzare in SVG le figure correnti a dall'interfaccia utente per accedere alle figure. Pu\`o venire generata dalla classe \textit{SVGReader} dopo la deserializzazione di un file SVG.
\subsubsezione{Attivit\`a svolte e dati trattati}
\begin{elencopuntato}[\normindent]
\item[-]  \textit{selectFigure} permette di selezionare una figura contenente il punto passato come argomento, se tale figura esiste.
\item[-]  \textit{each} permette di applicare una funzione ad ogni figura.
\item[-]  \textit{add} aggiunge una figura all'insieme
\item[-]  \textit{rem} rimuove una figura dall'insieme se presente
\end{elencopuntato}

\subsezione{Property}
\subsubsezione{Tipo, obiettivo e funzione del componente}
Rappresenta una propriet\`a generica di una figura.
\subsubsezione{Relazioni d'uso di altre componenti}
Nessuna.
\subsubsezione{Interfacce e relazioni di uso da altre componenti}
Implementata da tutte le propriet\`a concrete delle diverse figure.
\subsubsezione{Attivit\`a svolte e dati trattati}
\begin{elencopuntato}[\normindent]
\item[-] \textit{createWidget} crea un widget grafico che permetter\`a di modificare la propriet\`a all'utente.
\end{elencopuntato}

\subsezione{CSSProperty}
\subsubsezione{Tipo, obiettivo e funzione del componente}
Rappresenta una propriet\`a di una figura che necessita di una rappresentazione conforme ai CSS.
\subsubsezione{Relazioni d'uso di altre componenti}
Nessuna.
\subsubsezione{Interfacce e relazioni di uso da altre componenti}
Implementata da \textit{Colour}, \textit{TextFont} e \textit{TextSize}. 
\subsubsezione{Attivit\`a svolte e dati trattati}
\begin{elencopuntato}[\normindent]
\item[-] \textit{toCSS} genera una stringa che rappresenta la propriet\`a come propriet\`a CSS.
\end{elencopuntato}

\subsezione{Opacity}
\subsubsezione{Tipo, obiettivo e funzione del componente}
Rappresenta l'opacit\`a di un colore.
\subsubsezione{Relazioni d'uso di altre componenti}
\subsubsezione{Interfacce e relazioni di uso da altre componenti}
\textit{Colour} ne mantiene un'istanza.
\subsubsezione{Attivit\`a svolte e dati trattati}
Nessuna a questo livello.

\subsezione{EdgeNumber}
\subsubsezione{Tipo, obiettivo e funzione del componente}
Rappresenta il numero degli spigoli di un poligono.
\subsubsezione{Relazioni d'uso di altre componenti}
Implementa \textit{Property}.
\subsubsezione{Interfacce e relazioni di uso da altre componenti}
\textit{Polygon} ne mantiene un'istanza.
\subsubsezione{Attivit\`a svolte e dati trattati}
Nessuna a questo livello.

\subsezione{Colour}
\subsubsezione{Tipo, obiettivo e funzione del componente}
Rappresenta un colore.
\subsubsezione{Relazioni d'uso di altre componenti}
Implementa \textit{Property} e \textit{CSSProperty}. Mantiene un'istanza di \textit{Opacity} che rappresenta l'opacit\`a del colore.
\subsubsezione{Interfacce e relazioni di uso da altre componenti}
\textit{Figure} ne mantiene un'istanza per rappresentare il colore del bordo.
\subsubsezione{Attivit\`a svolte e dati trattati}
Nessuna a questo livello.

\subsezione{FillColour}
\subsubsezione{Tipo, obiettivo e funzione del componente}
Rappresenta un colore di riempimento.
\subsubsezione{Relazioni d'uso di altre componenti}
Deriva da \textit{Colour}.
\subsubsezione{Interfacce e relazioni di uso da altre componenti}
\textit{Circle}, \textit{Polygon} e \textit{Rectangle} ne mantengono un'istanza.
\subsubsezione{Attivit\`a svolte e dati trattati}
Nessuna a questo livello.

\subsezione{TextProperty}
\subsubsezione{Tipo, obiettivo e funzione del componente}
Propriet\`a di un testo grafico.
\subsubsezione{Relazioni d'uso di altre componenti}
Deriva da \textit{Property}.
\subsubsezione{Interfacce e relazioni di uso da altre componenti}
\textit{Text} ne mantiene delle istanze.
\subsubsezione{Attivit\`a svolte e dati trattati}
Nessuna a questo livello.

\subsezione{TextColour}
\subsubsezione{Tipo, obiettivo e funzione del componente}
Rappresenta il colore di un testo.
\subsubsezione{Relazioni d'uso di altre componenti}
Deriva da \textit{Colour} e da \textit{TextProperty}.
\subsubsezione{Interfacce e relazioni di uso da altre componenti}
\subsubsezione{Attivit\`a svolte e dati trattati}
Nessuna a questo livello.

\subsezione{TextFont}
\subsubsezione{Tipo, obiettivo e funzione del componente}
Rappresenta il font di un testo.
\subsubsezione{Relazioni d'uso di altre componenti}
Deriva da \textit{TextProperty} e implementa \textit{CSSProperty}.
\subsubsezione{Interfacce e relazioni di uso da altre componenti}
\subsubsezione{Attivit\`a svolte e dati trattati}
Nessuna a questo livello.

\subsezione{TextSize}
\subsubsezione{Tipo, obiettivo e funzione del componente}
Rappresenta la grandezza di un testo.
\subsubsezione{Relazioni d'uso di altre componenti}
Deriva da \textit{TextProperty} e implementa \textit{CSSProperty}.
\subsubsezione{Interfacce e relazioni di uso da altre componenti}
\subsubsezione{Attivit\`a svolte e dati trattati}
Nessuna a questo livello.

\subsezione{BoundingRectangle}
\subsubsezione{Tipo, obiettivo e funzione del componente}
Rappresenta la posizione e le dimensioni massime di una figura.
\subsubsezione{Relazioni d'uso di altre componenti}
Implementa \textit{Property}. Ogni istanza di \textit{Figure} ne mantiene un'istanza.
\subsubsezione{Interfacce e relazioni di uso da altre componenti}
\subsubsezione{Attivit\`a svolte e dati trattati}
Nessuna a questo livello.

\subsezione{Circle}
\subsubsezione{Tipo, obiettivo e funzione del componente}
Rappresenta un cerchio o un'ellisse.
\subsubsezione{Relazioni d'uso di altre componenti}
Deriva da \textit{Figure} e ne implementa i metodi astratti. Mantiene un'istanza di \textit{FillColour}.
\subsubsezione{Interfacce e relazioni di uso da altre componenti}
\subsubsezione{Attivit\`a svolte e dati trattati}
Nessuna a questo livello.

\subsezione{Rectangle}
\subsubsezione{Tipo, obiettivo e funzione del componente}
Rappresenta un rettangolo o un quadrato.
\subsubsezione{Relazioni d'uso di altre componenti}
Deriva da \textit{Figure} e ne implementa i metodi astratti. Mantiene un'istanza di \textit{FillColour}.
\subsubsezione{Interfacce e relazioni di uso da altre componenti}
\subsubsezione{Attivit\`a svolte e dati trattati}
Nessuna a questo livello.

\subsezione{Line}
\subsubsezione{Tipo, obiettivo e funzione del componente}
Rappresenta una linea generica.
\subsubsezione{Relazioni d'uso di altre componenti}
Deriva da \textit{Figure}.
\subsubsezione{Interfacce e relazioni di uso da altre componenti}
\subsubsezione{Attivit\`a svolte e dati trattati}
Nessuna a questo livello.

\subsezione{StraightLine}
\subsubsezione{Tipo, obiettivo e funzione del componente}
Rappresenta una linea diritta.
\subsubsezione{Relazioni d'uso di altre componenti}
Deriva da \textit{Line} e ne implementa i metodi astratti.
\subsubsezione{Interfacce e relazioni di uso da altre componenti}
\subsubsezione{Attivit\`a svolte e dati trattati}
Nessuna a questo livello.

\subsezione{FreeLine}
\subsubsezione{Tipo, obiettivo e funzione del componente}
Rappresenta una linea a mano libera.
\subsubsezione{Relazioni d'uso di altre componenti}
Deriva da \textit{Line} e ne implementa i metodi astratti. Mantiene una lista di punti.
\subsubsezione{Interfacce e relazioni di uso da altre componenti}
Usata come base da \textit{BezierCurve}.
\subsubsezione{Attivit\`a svolte e dati trattati}
\begin{elencopuntato}[\normindent]
\item[-] \textit{eachPoint} applica una funzione ad ogni nodo della linea.
\end{elencopuntato}

\subsezione{BezierCurve}
\subsubsezione{Tipo, obiettivo e funzione del componente}
Rappresenta una linea curva.
\subsubsezione{Relazioni d'uso di altre componenti}
Deriva da \textit{FreeLine}.
\subsubsezione{Interfacce e relazioni di uso da altre componenti}
\subsubsezione{Attivit\`a svolte e dati trattati}
Nessuna a questo livello.

\subsezione{Polygon}
\subsubsezione{Tipo, obiettivo e funzione del componente}
Rappresenta un poligono regolare.
\subsubsezione{Relazioni d'uso di altre componenti}
Deriva da \textit{Figure} e ne implementa i metodi astratti. Mantiene un'istanza di \textit{FillColour} e di \textit{EdgeNumber} che ne identifica il numero di lati.
\subsubsezione{Interfacce e relazioni di uso da altre componenti}
\subsubsezione{Attivit\`a svolte e dati trattati}
Nessuna a questo livello.

\subsezione{SVGWriter}
\subsubsezione{Tipo, obiettivo e funzione del componente}
Permette di creare un documento SVG.
\subsubsezione{Relazioni d'uso di altre componenti}
Accede alle figure correnti attraverso un'istanza di \textit{FigureSet}.
\subsubsezione{Interfacce e relazioni di uso da altre componenti}
\subsubsezione{Attivit\`a svolte e dati trattati}
\begin{elencopuntato}[\textbf{}]{\subsubsecindent}
\item \textit{write} crea un documento SVG che contiene le figure correntemente disegnate.
\end{elencopuntato}

\subsezione{SVGWritable}
\subsubsezione{Tipo, obiettivo e funzione del componente}
Rappresenta un oggetto convertibile in SVG.
\subsubsezione{Relazioni d'uso di altre componenti}
Usa \textit{SVGGenerator} per generare i comandi SVG.
\subsubsezione{Interfacce e relazioni di uso da altre componenti}
Implementata dalle classi che rappresentano figure.
\subsubsezione{Attivit\`a svolte e dati trattati}
\begin{elencopuntato}[\textbf{}]{\subsubsecindent}
\item \textit{toSVG} converte in SVG.
\end{elencopuntato}

\subsezione{SVGGenerator}
\subsubsezione{Tipo, obiettivo e funzione del componente}
Rappresenta un documento SVG in costruzione.
\subsubsezione{Relazioni d'uso di altre componenti}
\subsubsezione{Interfacce e relazioni di uso da altre componenti}
Usato da \textit{SVGWriter} e dalle classi che implementano \textit{SVGWritable} per realizzare un documento SVG.
\subsubsezione{Attivit\`a svolte e dati trattati}
\begin{elencopuntato}[\textbf{}]{\subsubsecindent}
\item \textit{startCommand} apre un comando con il nome dato.
\item \textit{attr} aggiunge un attributo con il nome e il valore dato al comando corrente.
\item \textit{endCommand} chiude il comando corrente.
\end{elencopuntato}

\subsezione{XMLParser}
\subsubsezione{Tipo, obiettivo e funzione del componente}
Un parser SVG.
\subsubsezione{Relazioni d'uso di altre componenti}
\subsubsezione{Interfacce e relazioni di uso da altre componenti}
Usato da \textit{SVGReader}.
\subsubsezione{Attivit\`a svolte e dati trattati}

\subsezione{SVGTag}
\subsubsezione{Tipo, obiettivo e funzione del componente}
Rappresenta un tag (ovvero un comando) SVG.
\subsubsezione{Relazioni d'uso di altre componenti}
\subsubsezione{Interfacce e relazioni di uso da altre componenti}
Usato da \textit{SVGReader} e dalle classi che implementano \textit{SVGReadable}.
\subsubsezione{Attivit\`a svolte e dati trattati}

\subsezione{SVGReadable}
\subsubsezione{Tipo, obiettivo e funzione del componente}
Rappresenta un oggetto caricabile da un documento SVG.
\subsubsezione{Relazioni d'uso di altre componenti}
\subsubsezione{Interfacce e relazioni di uso da altre componenti}
Implementata dalle classi che rappresentano figure.
\subsubsezione{Attivit\`a svolte e dati trattati}
\begin{elencopuntato}[\textbf{}]{\subsubsecindent}
\item \textit{fromSVG} inizializza l'oggetto a partire da un tag SVG.
\end{elencopuntato}

\subsezione{SVGReadableClass}
\subsubsezione{Tipo, obiettivo e funzione del componente}
Metaclasse di \textit{SVGReadable}.
\subsubsezione{Relazioni d'uso di altre componenti}
\subsubsezione{Interfacce e relazioni di uso da altre componenti}
Usata da \textit{SVGElementRegistry} per associare il nome di un tag a una classe che implementa \textit{SVGReadable}.
\subsubsezione{Attivit\`a svolte e dati trattati}

\subsezione{SVGReader}
\subsubsezione{Tipo, obiettivo e funzione del componente}
Permette di caricare un documento SVG.
\subsubsezione{Relazioni d'uso di altre componenti}
Usa \textit{XMLParser} per fare il parsing del documento testuale, crea \textit{SVGTag} e usa \textit{SVGElementRegistry} per creare una figura a partire da un tag SVG.
\subsubsezione{Interfacce e relazioni di uso da altre componenti}
\subsubsezione{Attivit\`a svolte e dati trattati}
\begin{elencopuntato}[\textbf{}]{\subsubsecindent}
\item \textit{read} prende un documento SVG e ritorna un'istanza di \textit{FigureSet} che lo rappresenta.
\end{elencopuntato}

\subsezione{SVGElementRegistry}
\subsubsezione{Tipo, obiettivo e funzione del componente}
Associa nomi di tag SVG a classi (non istanze) che implementano \textit{SVGReadable} e ne permette l'istanziazione. 
\subsubsezione{Relazioni d'uso di altre componenti}
Usa \textit{SVGReadableClass} per mantenere un riferimento ad una classe per poterla poi istanziare.
\subsubsezione{Interfacce e relazioni di uso da altre componenti}
Usata da \textit{SVGReader} durante la conversione di un documento SVG. Ogni classe che implementa \textit{SVGReadable} deve registrarsi.
\subsubsezione{Attivit\`a svolte e dati trattati}
\begin{elencopuntato}[\textbf{}]{\subsubsecindent}
\item \textit{register} associa il nome di un tag SVG ad una classe che implementa \textit{SVGReadable}.
\item \textit{makeFigureClassFromTag} crea un'istanza della classe associata al nome di tag passato come parametro.
\end{elencopuntato}

\subsezione{Tipo, obiettivo e funzione del componente}
\subsezione{Relazioni d'uso di altre componenti}
\subsezione{Interfacce e relazioni di uso da altre componenti}
\subsezione{Attivit\`a svolte e dati trattati}
\sezione{Stime di fattibilit\`a e di bisogno di risorse}
\sezione{Tracciamento della relazione componenti-requisiti}
			

\end{document}
