	\input{../TeX/base} %BASE!!!
\usepackage{multirow}
\title{\TITOLODOC}
\author{Dal Bosco Davide}

\begin{document}

\renewcommand{\insertversion}{2.0} %INSERIRE LA VERSIONE QUI DENTRO STILE x.x.xx
\renewcommand{\TITOLODOC}{Specifica Tecnica} %INSERIRE IL TITOLO DEL DOCUMENTO DA FAR COMPARIRE A PIE PAGINA
\renewcommand{\glosspath}{.\glossario} %INSERIRE PERCORSO RELATIVO

%%%%%%%%%%%%%%%%%%%%%%PARTE DA NON MODIFICARE%%%%%%%%%%%%%%%%%
\begin{titlepage}
\begin{center}
	\begin{Large}	\today \end{Large}
\end{center}

\vspace{20pt}

\begin{center}
	\begin{Huge}
				\textbf{\ajax}
	\end{Huge}
\end{center}			

\begin{center}
	\begin{large}
				\textbf{Software per il Disegno Grafico\\ in Tecnologie Web}
	\end{large}
\end{center}			

\vspace{20pt}

\begin{center}
\includegraphics[width=150pt]{../logo/logo}
\end{center}

\vspace{170pt}
\begin{center} %INSERIRE ALL'INTERNO IL TITOLO DOCUMENTO CHE COMPARIRA NELLA PAGINA INIZIALE				
	\begin{Huge}
				\textbf{\TITOLODOC}
	\end{Huge}
			\\
\end{center}
\vspace{190pt}
\begin{center}
Versione: \insertversion
\end{center}
\end{titlepage}

\newpage
%%%%%%%%%%%%%%%%%%%%%%FINE PARTE DA NON MODIFICARE%%%%%%%%%%%%%%%%%

\begin{center} %INSERIRE ALL'INTERNO IL TITOLO DOCUMENTO CHE COMPARIRA NELLA PAGINA INIZIALE
	\begin{Huge}	
				\textbf{\TITOLODOC}
			\\
	\end{Huge}
\end{center}

%\setlength{\parindent}{18pt} %settato indentazione di default 
\section*{\LARGE Sommario:}
Il presente documento descrive le scelte tecniche, architetturali e tecnologiche effettuate dall'azienda WebShape per lo sviluppo del Capitolato C04.

 %SEZIONE SOMMARIO
\indent \indent

\section*{\LARGE Stato del documento:}
\indent \indent
	Formale Esterno

\section*{\LARGE Redazione:}
	\begin{table}[!h]
		\begin{center}
			\begin{tabular}
				{|c|c|}
				\hline
				%%%%%%%%%%%%%%INTESTAZIONE COLONNE%%%%%%%%%%%%%%%%%%%%%%%%%%%%%%%%
				\multicolumn{2}{|c|}{ \textbf{Redazione} } \\
				\hline
				\textbf{Fase} & \textbf{Redattori} \\
				%%%%%%%%%%%%%%FINE INTESTAZIONE COLONNE%%%%%%%%%%%%%%%%%%%%%%%%%%%%%%%%%%%%%%
				\hline
				%%%%%%%%%%% PARTE DA MODIFICARE %%%%%%%%%%%%%%%%%%%%%%%%%%%%%%%%%%%%%%%%%%		
				\multirow{4}{*}{RR-RPP} & Bizzotto Piero\\
										& Dal Bosco Davide\\
										& Dissegna Stefano\\
										& Rizzo Maurizio\\
				\hline
				\multirow{2}{*}{RPP-RQ} & Dissegna Stefano\\
										& Rizzo Maurizio\\
				\hline
				%%%%%%%%%%% FINE PARTE DA MODIFICARE %%%%%%%%%%%%%%%%%%%%%%%%%%%
			\end{tabular}
			\caption{Lista Redattori} %INSERIRE DIDASCALIA - SE NECESSARIA - 
			\label{tabredazione}
		\end{center}
	\end{table}

\section*{\LARGE Verifica:}
\begin{table}[!h]
	\begin{center}
		\begin{tabular}
			{|c|c|}
			\hline
			%%%%%INTESTAZIONE COLONNE%%%%%%%%%%%%%%%%%%%%%%%%%%%%%%%
			\multicolumn{2}{|c|}{ \textbf{Verifica} } \\
			\hline
			\textbf{Fase} & \textbf{Verificatori} \\
			%%%%%%%%%%%%%%FINE INTESTAZIONE COLONNE%%%%%%%%%%%%%%%%%%%%%%%%%%%%%%
			\hline
			%%%%%%%%%%% PARTE DA MODIFICARE %%%%%%%%%%%%%%%%%%%%%%%%%%%%%%%%%%%%%%		
			\multirow{1}{*}{RR-RPP} & Geremia Mirco\\
									
			\hline
			\multirow{1}{*}{RPP-RQ} & Bizzotto Piero\\
									
			\hline
			%%%%%%%%%%% FINE PARTE DA MODIFICARE %%%%%%%%%%%%%%%%%%%%%%%%%%%%%%%%%%%
		\end{tabular}
		\caption{Lista Verificatori} %INSERIRE DIDASCALIA - SE NECESSARIA - 
		\label{tabverifica}
	\end{center}
\end{table}
	
\section*{\LARGE Approvazione:}
\begin{table}[!h]
	\begin{center}
		\begin{tabular}
			{|c|c|}
			\hline
			%%%%%INTESTAZIONE COLONNE%%%%%%%%%%%%%%%%%%%%%%%%%%%%%%%
			\multicolumn{2}{|c|}{ \textbf{Approvazione} } \\
			\hline
			\textbf{Fase} & \textbf{Approvatori} \\
			%%%%%%%%%%%%%%FINE INTESTAZIONE COLONNE%%%%%%%%%%%%%%%%%%%%%%%%%%%%%%
			\hline
			%%%%%%%%%%% PARTE DA MODIFICARE %%%%%%%%%%%%%%%%%%%%%%%%%%%%%%%%%%%%%%		
			\multirow{1}{*}{RR-RPP} & Carollo Mirko\\
									
			\hline
			\multirow{1}{*}{RPP-RQ} & Cunico Marco\\
									
			\hline
			%%%%%%%%%%% FINE PARTE DA MODIFICARE %%%%%%%%%%%%%%%%%%%%%%%%%%%%%%%%%%%
		\end{tabular}
		\caption{Lista Approvatori} %INSERIRE DIDASCALIA - SE NECESSARIA - 
		\label{tabapprovazione}
	\end{center}
\end{table}

\textbf{}
\newpage
\section*{\LARGE Lista di Distribuzione:}

	\begin{elenconumerato}{\normindent}
		\item WebShape 
		\item I committenti Conte Renato e Vardanega Tullio in rappresentanza \\  dell'azienda proponente Zucchetti SPA
	\end{elenconumerato}





\section*{\Large Registro delle Modifiche:}


\begin{center}
	\begin{table}[h]
		  \begin{tabular*}
			{1\textwidth}%
				{@{\extracolsep{\fill}}|p{0.1\textwidth}|p{0.54\textwidth}|p{0.26\textwidth}|}
			 \hline
%%%%%%%%%%%%%%INTESTAZIONE COLONNE%%%%%%%%%%%%%%%%%%%%%%%%%%%%%%%%%%%%%%%%%%
			\textbf{Versione}  & \textbf{Descrizione} & \textbf{Autore} \\
%%%%%%%%%%%%%%FINE INTESTAZIONE COLONNE%%%%%%%%%%%%%%%%%%%%%%%%%%%%%%%%%%%%%%%
		 \hline
%%%%%%%%%%% PARTE DA MODIFICARE %%%%%%%%%%%%%%%%%%%%%%%%%%%%%%%%%%%%%%%%%%%
                  2.0 & 7$\slash$03$\slash$2009 Modifiche per rilascio RQ  & Dissegna Stefano \\
                                \hline 
                  1.7 & 13$\slash$02$\slash$2009 Aggiunta e descrizione classi di integrazione & Rizzo Maurizio \\
                                \hline 
                  1.6 & 7$\slash$02$\slash$2009 Descritto in dettaglio il ruolo dei punti in \textit{FreeLine} & Dissegna Stefano \\
				\hline
                  1.5 & 6$\slash$02$\slash$2009 Giustificato JavaScript, descritto problema supporto diversi browser  & Dissegna Stefano \\
				\hline
                  1.4 & 5$\slash$02$\slash$2009 Maggiore dettaglio nella descrizione di alcune classi di ApplicationLogic & Dissegna Stefano \\
				\hline
                  1.3 & 4$\slash$02$\slash$2009 Suddivisione del diagramma ApplicationLogic in tre parti & Dissegna Stefano \\
				\hline
                  1.2 & 4$\slash$02$\slash$2009 Aggiunti diagrammi per design pattern & Dissegna Stefano \\
				\hline
		  1.1 & 4$\slash$02$\slash$2009 Sistemati design pattern & Dissegna Stefano \\
				\hline
		  1.0 & 23$\slash$01$\slash$2009  Verifica finale in preparazione al rilascio per RPP & Bizzotto Piero \\		
		  		\hline
		  0.9 & 23/01/2009 Correzioni in seguito a verifica & Dal Bosco Davide \\
		  		\hline
		  0.8 & 23/01/2009 Correzioni in seguito a verifica & Bizzotto Piero \\
		  		\hline
		  0.7 & 22/01/2009 Correzioni in seguito a verifica & Dissegna Stefano \\
		  		\hline
		  0.6 & 22/01/2009 Descrizione e diagramma delle classi ServerLogic & Dal Bosco Davide\\
		  		\hline
          0.5 & 22/01/2009 Descrizione e diagramma delle classi GUI & Bizzotto Piero \\
          		\hline
          0.4 & 22/01/2009 Descrizione design pattern & Dissegna Stefano \\
          		\hline
          0.3 & 21/01/2009 Descrizione generale e delle classi per ApplicationLogic & Dissegna Stefano \\
          		\hline
          0.2 & 20/01/2009 Diagrammi di sequenza per ApplicationLogic & Dissegna Stefano \\
          		\hline
          0.1 & 19/01/2009 Diagramma delle classi per ApplicationLogic & Dissegna Stefano \\
          		\hline
    	  0.0 & 15/01/2009 Bozza iniziale & Dal Bosco Davide \\
		\hline %%FINE RIGA
%%%%%%%%%%% FINE PARTE DA MODIFICARE %%%%%%%%%%%%%%%%%%%%%%%%%%%%%%%%%%%%%
		\end{tabular*}
	\caption{Registro delle modifiche} %INSERIRE DIDASCALIA - SE NECESSARIA - 
	\label{tab:modifiche}
	\end{table}
\end{center}


\newpage
\thispagestyle{fancy}
\tableofcontents
\thispagestyle{fancy}
\newpage

\sezione{Introduzione}

\subsezione{Scopo del documento}
Il documento si propone di presentare ai Committenti le scelte riguardanti la progettazione architetturale del capitolato d'appalto C04. In particolare viene presentata la struttura dei package e le loro relazioni, le unit\`a che compongono i package stessi e uno schema generale dell'interfaccia grafica.

\sezione{Scopo del prodotto}
Il software AJAXDRAW \`e proposto per verificare e dimostrare la fattibilit\`a di realizzazione di un'applicazione di disegno grafico, in grado di poter elaborare figure vettoriali primitive e complesse utilizzando le tecnologie web.

\subsezione{Glossario}
All'interno del documento \textit{Glossario}, sono presentati i termini tecnici utilizzati in tutti i documenti. Il glossario \`e fornito in allegato al presente documento.
\subsezione{Riferimenti}
Il presente documento \`e redatto utilizzando le convenzioni inserite nel documento \textit{NormeDiProgetto}, consultabile dal repository pubblico al quale WebShape si appoggia per i suoi progetti.
\subsubsezione{Normativi}
\begin{elencopuntato}[\subsubsecindent]
\item[-] \textit{NormeDiProgetto.pdf}
\item[-] Raccomandazione SVG: http://www.w3.org/TR/SVG11/
\item[-] Raccomandazione CSS: http://www.w3.org/TR/CSS21/
\item[-] Working Draft HTML 5: http://www.w3.org/TR/html5/
\item[-] Specifiche UML: http://www.omg.org/spec/UML/2.1.2/
\end{elencopuntato}

\sezione{Definizione del prodotto}
\subsezione{Metodo e formalismo di specifica}
L'architettura del sistema \`e stata sviluppata secondo un paradigma orientato agli oggetti. Le componenti del sistema e le loro relazioni verranno presentate in questo documento tramite diagrammi standard UML. Verranno presentate prima le componenti ad alto livello, le quali saranno poi descritte in maggiore dettaglio. Saranno inoltre descritti i pattern progettuali utilizzati.
\subsezione{Presentazione dell'architettura generale del sistema}

\begin{figure}[!ht]
\centering
\includegraphics{images/MainView.jpg}
\caption{Package principali}
\end{figure}

Il sistema si compone, al massimo livello di astrazione, di tre componenti: il primo, contenuto nel package \textit{GUI}, si occupa dell'interfaccia utente, il secondo, contenuto nel package \textit{ApplicationLogic}, gestisce la logica interna, ovvero la gestione in memoria di quanto disegnato e della conversione tra rappresentazione interna e rappresentazione in SVG, e il terzo, contenuto nel package \textit{DocBackend}, si occupa del salvataggio e del caricamento di file testuali e dell'accesso a file statici, come ad esempio il manuale.
\newpage
\subsubsezione{Design Pattern usati}

\paragrafo{Singleton:}
Esiste una sola istanza della classe \textit{Visualization}. \`E possibile accedere a questa istanza attraverso una variabile globale.

\paragrafo{Template Method:}
\begin{figure}[!ht]
\centering
\includegraphics{images/templatemethod.jpg}
\caption{Template method}
\end{figure}
\subparagrafo{Algoritmo parametrizzato:}
Il metodo \textit{selectFigure} di \textit{FigureSet} implementa la logica di selezione di una figura a partire da un punto.\\ \\
\textbf{Metodo astratto da implementare:}
Implementando il metodo astratto \textit{draw} di \textit{Figure} \`e possibile parametrizzare l'algoritmo generale adattandolo alla figura specifica. L'algoritmo infatti permette di selezionara una figura a seconda di dove essa viene disegnata dal metodo \textit{draw} il quale non ricever\`a il canvas visualizzato dall'utente, ma un canvas apposito.
\newpage
\paragrafo{Abstract Factory:} 
\begin{figure}[!ht]
\centering
\includegraphics[scale=0.5]{images/absfactory.png}
\caption{Abstract factory}
\end{figure}
\subparagrafo{Factory:} 
\textit{Property} genera un elemento grafico per modificare una propriet\`a specifica.\\ \\
\textbf{Concrete Factory: }\\
tutte le classi che derivano da \textit{Property}. \\ \\
\textbf{Abstract Product: }\\
\textit{Widget.}\\ \\
\textbf{Concrete Product: }\\
un particolare elemento grafico a seconda della propriet\`a coinvolta. \\
\textbf{Client:}\\
\textit{PropertiesDialog.}

\newpage

\subsezione{Identificazione dei componenti architetturali di alto livello}
\subsubsezione{GUI - Intefaccia Grafica}
\begin{figure}[!ht]
\centering
\includegraphics[scale=0.5]{images/GUIlogic.png}
\caption{GUILogic}
\end{figure}

\newpage

\begin{figure}[!ht]
\centering
\includegraphics[scale=0.8]{images/ActivityGUI.png}
\caption{Attivit\` a: rapporto utente/canvas}
\end{figure}
\newpage

La GUI \` e  il client che verr\` a utilizzato nel sistema AJAXDRAW, e verr\` a descritta brevemente, in quanto rappresenta solo un'unit\` a che andr\`  a a lavorare sulle altre componenti di sistema. \` E costruita nella sua totalit\` a nella pagina web caricata dal browser, rappresentata dalla classe \textit{Page}, e composta da tutti i widgets con i quali interagire (che implementano tutti l'interfaccia generica \textit{Widget}).  I widgets si suddividono in \textit{Toolbar} (barra degli strumenti), \textit{Button} (pulsanti generici, che compongono anche la toolbar stessa), \textit{Palette} e \textit{ColourDialog}(per la gestione del colore), e \textit{PropertiesDialog}(per la modifica delle propriet\` a, che aggrega gli specifici \textit{Setter}). La pagina \` e composta inoltre dalla classe \textit{Visualization}, la quale gestisce il \textit{Canvas} e tutte le operazioni su di esso eseguite, tra cui la conversione tra coordinate utente e coordinate interne e lo zoom. Le classi che rappresentano oggetti con cui \`e necessario interagire tramite mouse implementano, tutte l'interfaccia \textit{UserClickEvent}, e tra queste vi \` e ovviamente la classe \textit{Canvas}, che oltre a questa implementa anche \textit{MouseMovePressed} e \textit{MouseMoveReleased}, necessarie al disegno tramite il mouse.  \` E infine presente la possibilit\` a per il programma di avvertire l'utente in caso di errore interno all'applicazione, tramite l'oggetto \textit{Alert}. \\ \\ \\

\begin{figure}[!ht]
\centering
\includegraphics[scale=0.5]{images/bozzaGUI.png}
\caption{Bozza della GUI}
\end{figure}

\newpage
\subsubsezione{ApplicationLogic - Logica interna}

\begin{figure}[!ht]
\centering
\includegraphics[scale=0.4]{images/applogic.png}
\caption{ApplicationLogic}
\end{figure}

Questo sottosistema si divide in tre ulteriori sottosistemi: la gestione delle
figure, la loro conversione in SVG e la lettura da file SVG.

\subsubsezione{Gestione figure}
\begin{figure}[!ht]
\centering
\includegraphics[scale=0.5]{images/gestionefigure.png}
\caption{Gestione figure}
\end{figure}
Si occupa della gestione interna delle figure disegnabili e manipolabili dall'utente e delle relative propriet\`a. Il termine \textit{figura} in questo contesto include anche i testi disegnati sul canvas. La classe \textit{Figure} rappresenta una figura qualsiasi. Una figura \`e caratterizzata da delle propriet\`a derivate dalla classe \textit{Property}. La suddetta classe contiene un metodo \textit{createWidget} per fabbricare un elemento grafico, il quale permette all'utente di modificare la propriet\`a che ha generato il widget stesso. Ogni figura contiene un \textit{BoundingRectangle} che rappresenta la cornice entro cui la figura \`e completamente contenuta. Una figura pu\`o essere selezionata oppure no. Le classi che implementano \textit{Figure} sono responsabili di disegnarsi sul canvas implementando il metodo \textit{draw}. In questo metodo, non si dovr\`a tener conto dell'area correntemente visualizzata o del livello di zoom, in quanto gestiti in modo trasparente dalla classe \textit{Visualization}. Ad ogni figura \`e associato un insieme di punti significativi, come ad esempio i vertici in un quadrato, ottenibili tramite il metodo \textit{getMainPoints}. Ogni figura mantiene un colore del bordo, rappresentato dalla classe \textit{Colour}. Le classi \textit{Circle}, \textit{Polygon} e \textit{Rectangle} aggiungono un ulteriore colore per il riempimento, rappresentato dalla classe \textit{FillColour}. La classe \textit{Text} rappresenta un testo disegnato dall'utente, e mantiene delle propriet\`a derivate dalla classe \textit{TextProperty}. A fini implementativi, le propriet\`a del testo e del colore andranno rappresentate secondo lo standard \textit{CSS}, quindi le suddette propriet\`a dovranno implementare l'interfaccia \textit{CSSProperty} che offre il metodo \textit{toCSS} responsabile della conversione dalla rappresentazione interna a quella \textit{CSS}. La classe \textit{FigureSet} mantiene l'insieme delle istanze di \textit{Figure} correntemente disegnate dall'utente e offre la possibilit\`a di selezionarne una a partire da un punto sul canvas.

%Parte di conversione SVG
\newpage
\subsubsezione{Conversione in \textit{SVG}}
Si occupa di trasformare il \textit{FigureSet} corrente in una stringa 
contenente un documento SVG valido e viceversa.  \\


\textbf{Da \textit{FigureSet} a SVG}

\begin{figure}[!ht]
\centering
\includegraphics[width=12cm]{images/ClassiScritturaSVG.png}
\caption{ApplicationLogic - Creazione documento SVG, diagramma delle classi}
\end{figure}

Le classi derivanti da \textit{Figure} implementano l'interfaccia \textit{SVGWritable}, che rappresenta un qualsiasi elemento convertibile in SVG. La conversione avviene tramite il metodo \textit{toSVG}. La classe \textit{SVGWriter} dovr\`a, nel metodo \textit{write}, creare un documento SVG vuoto e richiamare il metodo \textit{toSVG} su tutte le figure contenute, le quali genereranno gli elementi SVG necessari alla propria rappresentazione. Per creare un documento SVG si istanzia la classe \textit{SVGGenerator}, la quale offre dei metodi per generare comandi SVG. Quando tutte le figure hanno avuto la possibilit\`a di inserirsi nel documento SVG, il metodo \textit{flush} chiude il documento e lo ritorna sotto forma di stringa.
\newpage
\begin{figure}[!ht]
\centering
\includegraphics{images/ScritturaSVG.jpg}
\caption{ApplicationLogic - Creazione documento SVG, diagramma di sequenza}
\end{figure}

\newpage
\textbf{Da SVG a \textit{FigureSet}}

\begin{figure}[!ht]
\centering
\includegraphics[width=13cm]{images/ClassiLetturaSVG.png}
\caption{ApplicationLogic - Caricamento documento SVG, diagramma delle classi}
\end{figure}

L'importazione di un file SVG viene gestita dalla classe \textit{SVGReader} attraverso il metodo \textit{read}, che riceve un documento SVG valido sotto forma di stringa e restituisce un'istanza di \textit{FigureSet} corrispondente. Viene fatto il parsing del documento come \textit{XML} attraverso la classe \textit{XMLParser}, e per ogni comando SVG viene creato un'istanza di \textit{SVGTag} che lo rappresenta. Il nome del tag verr\`a usato come indice dal metodo \textit{makeFigureClassFromTag} della classe \textit{SVGElementRegistry}, che costruir\`a un'istanza della classe corrispondente. Su quest'istanza verr\`a invocato il metodo \textit{fromSVG}, il quale \`e responsabile di inizializzare la classe a partire dal tag. Per associare una classe al nome di un tag si usa il metodo \textit{register}. La classe registrata deve implementare l'interfaccia \textit{SVGReadable}.


\begin{figure}[!ht]
\centering
\includegraphics[width=15cm]{images/LetturaSVG.png}
\caption{ApplicationLogic - Caricamento documento SVG, diagramma di sequenza}
\end{figure}





\newpage
\subsubsezione{DocBackend - Lato Server}
\begin{figure}[!ht]
\centering
\includegraphics{images/ServerLogic.jpg}\\
\caption{ServerLogic}
\end{figure}

In AJAXDRAW \`e necessaria la presenza di una parte server, visto il modello client-server dell'applicazione. La parte server ha funzionalit\`a minimali, ma fondamentali per permettere il funzionamento dell'intero sistema. Le sue funzioni, oltre a quella di essere un webserver e fornire la pagina HTML al client, sono esclusivamente quelle di permettere un salvataggio in formato SVG del disegno effettuato sul client, effettuare il caricamento di un file residente sul disco del client e fornire l'accesso al manuale utente. Come si vede nella figura 6, questo sottosistema \`e composto lato client da una classe \textit{Document} che si occupa delle comunicazioni con il server, nella quale sono presenti i metodi \textit{save} e \textit{load}. Questi metodi permettono l'invio di una richiesta al server, rappresentata dalla classe \textit{Request}, la quale, lato server, fornisce il metodo \textit{acceptRequest} per accettare una richiesta. La classe \textit{Server} contiene tre metodi: \textit{getSave} che riceve un documento e lo salva su disco, \textit{getLoad} che permette di caricare un file presente sul disco del client e \textit{getHTML} che fornisce le pagine HTML ed i file statici, compreso il manuale utente. La classe \textit{File} rappresenta un file che risiede sul disco del server, il quale pu\`o essere prelevato tramite \textit{getFile} o cancellato tramite \textit{deleteFile}. La classe \textit{Clean} contiene il metodo \textit{sweeper} il quale elimina i file generati dal client che sono gi\`a stati scaricati oppure che non verranno usati, in modo da evitare che questi file presenti temporaneamente sul server non lo intasino. La classe lato client \textit{StaticDocument}, tramite il metodo \textit{get} permette di inviare una richiesta al server per ottenere il manuale utente. Sono presenti inoltre delle classi che permettono la gestione di eccezioni: pu\`o verificarsi un errore di comunicazione (rappresentato dall'interfaccia \textit{CommunicationError} e dalle classi \textit{ConnectionError} e \textit{ServerError} con relativi metodi che la implementano), o un errore di file non trovato. Questo tipo di eccezione \`e rappresentato dalla classe \textit{FileNotFound} con il relativo metodo \textit{getFileError}. 


\sezione{Descrizione dei singoli componenti}
%Parte Piero
\subsezione{Widget}
\subsubsezione{Tipo, obiettivo e funzione del componente}
Rappresenta un generico oggetto grafico presente nella GUI e visibile all'utente.
\subsubsezione{Relazioni d'uso di altre componenti}
Nessuna.
\subsubsezione{Interfacce e relazioni di uso da altre componenti}
Implementata da tutti gli oggetti grafici concreti presenti nella pagina.
\subsubsezione{Attivit\`a svolte e dati trattati}
\leftskip=36pt{Nessuna a questo livello. }

\subsezione{Toolbar}
\subsubsezione{Tipo, obiettivo e funzione del componente}
Classe concreta che rappresenta la toolbar dalla quale l'utente pu\` o scegliere la funzione desiderata, posizionata in maniera da essere facilmente accessibile e sempre pronta all'uso.
\subsubsezione{Relazioni d'uso di altre componenti}
\` E formata da pi\` u istanze aggregate della classe \textit{Button}, per rappresentare ciascun tool selezionabile. Oltre a \textit{Widget}, implementa l'interfaccia \textit{UserClickEvent} per la gestione dei click del mouse. 
\subsubsezione{Interfacce e relazioni di uso da altre componenti}
\textit{Page} ne mantiene un'istanza.
\subsubsezione{Attivit\`a svolte e dati trattati}
\leftskip=36pt{Nessuna a questo livello.}

\subsezione{Button}
\subsubsezione{Tipo, obiettivo e funzione del componente}
Classe concreta che rappresenta un generico pulsante tramite il quale effettuare l'operazione desiderata.
\subsubsezione{Relazioni d'uso di altre componenti}
Oltre a \textit{Widget}, implementa l'interfaccia \textit{UserClickEvent} per la gestione dei click del mouse. 
\subsubsezione{Interfacce e relazioni di uso da altre componenti}
\textit{Page} e \textit{Toolbar }ne mantengono una o pi\` u istanze, \` e inoltre implementata dalla classe \textit{CloneButton}.
\subsubsezione{Attivit\`a svolte e dati trattati}
\begin{elencopuntato}[\normindent]
\item[-]  \textit{setPressed} metodo che valuta la pressione del bottone per decidere le operazioni da effettuare.
\end{elencopuntato}

\subsezione{CloneButton}
\subsubsezione{Tipo, obiettivo e funzione del componente}
Classe concreta che permette a un utente di clonare un oggetto presente sul canvas.
\subsubsezione{Relazioni d'uso di altre componenti}
Eredita dalla classe \textit{Button}.
\subsubsezione{Interfacce e relazioni di uso da altre componenti}
nessuna.
\subsubsezione{Attivit\`a svolte e dati trattati}
\leftskip=36pt{Nessuna a questo livello.}

\subsezione{Palette}
\subsubsezione{Tipo, obiettivo e funzione del componente}
Classe concreta che rappresenta una tavolozza composta dai colori di base pi\`u  generalmente utilizzati, nel caso l'utente non abbia bisogno di un colore specifico, la quale imposta il colore desiderato da applicare alle figure da disegnare e gi\` a disegnate semplicemente tramite la pressione della casella apposita.
\subsubsezione{Relazioni d'uso di altre componenti}
Oltre a \textit{Widget}, implementa l'interfaccia \textit{UserClickEvent} per la gestione dei click del mouse. Usa il metodo setColor della classe \textit{ColourDialog} per effettuare il cambiamento di colore alla figura.
\subsubsezione{Interfacce e relazioni di uso da altre componenti}
\textit{Page} ne mantiene un'istanza.
\subsubsezione{Attivit\`a svolte e dati trattati}
\leftskip=36pt{Nessuna a questo livello.}

\subsezione{ColourDialog}
\subsubsezione{Tipo, obiettivo e funzione del componente}
Classe concreta che rappresenta il menu di modifica dei colori di riempimento e dei bordi delle figure disegnate e da disegnare, il quale integra una ruota dei colori funzionale alla scelta di qualsiasi colore desiderato.
\subsubsezione{Relazioni d'uso di altre componenti}
Implementa \textit{Widget}.
\subsubsezione{Interfacce e relazioni di uso da altre componenti}
\textit{Page} ne mantiene un'istanza.
\subsubsezione{Attivit\`a svolte e dati trattati}
\begin{elencopuntato}[\normindent]
\item[-]  \textit{setColour} metodo che setta il colore scelto sulla figura da disegnare o selezionata sul canvas.
\end{elencopuntato}

\subsezione{PropertiesDialog}
\subsubsezione{Tipo, obiettivo e funzione del componente}
Classe concreta che rappresenta  il menu di modifica delle propriet\` a di un oggetto selezionato o da disegnare sul canvas a seconda del tipo dello stesso, che sia linea, poligono o casella di testo, ognuno con le sue caratteristiche generiche e specifiche.
\subsubsezione{Relazioni d'uso di altre componenti}
\`E formata da pi\`u istanze aggregate tra cui \textit{BoundingRectagleSetter}, \textit{EdgeNumberSetter} e \textit{FontSetter}. Implementa \textit{Widget}.
\subsubsezione{Interfacce e relazioni di uso da altre componenti}
\textit{Page} ne mantiene un'istanza.
\subsubsezione{Attivit\`a svolte e dati trattati}
\leftskip=36pt{Nessuna a questo livello.}

\subsezione{BoundingRectagleSetter}
\subsubsezione{Tipo, obiettivo e funzione del componente}
Classe concreta che rappresenta l'oggetto che si occupa di impostare e modificare le coordinate del rettangolo contenente la figura disegnata sul canvas.
\subsubsezione{Relazioni d'uso di altre componenti}
Nessuna.
\subsubsezione{Interfacce e relazioni di uso da altre componenti}
\textit{PropertiesDialog} ne mantiene un'istanza.
\subsubsezione{Attivit\`a svolte e dati trattati}
\leftskip=36pt{Nessuna a questo livello.}

\subsezione{EdgeNumberSetter}
\subsubsezione{Tipo, obiettivo e funzione del componente}
Classe concreta che rappresenta l'oggetto che si occupa di modificare il numero di lati della figura da disegnare o selezionata.
\subsubsezione{Relazioni d'uso di altre componenti}
Nessuna.
\subsubsezione{Interfacce e relazioni di uso da altre componenti}
\textit{PropertiesDialog} ne mantiene un'istanza.
\subsubsezione{Attivit\`a svolte e dati trattati}
\leftskip=36pt{Nessuna a questo livello.}

\subsezione{RotationSetter}
\subsubsezione{Tipo, obiettivo e funzione del componente}
Classe concreta che permette a un utente di ruotare un oggetto presente sul canvas
\subsubsezione{Relazioni d'uso di altre componenti}
nessuna. 
\subsubsezione{Interfacce e relazioni di uso da altre componenti}
\textit{PropertiesDialog} ne mantiene un' istanza.
\subsubsezione{Attivit\`a svolte e dati trattati}
\leftskip=36pt{Nessuna a questo livello.}

\subsezione{FontSetter}
\subsubsezione{Tipo, obiettivo e funzione del componente}
Classe concreta che rappresenta l'oggetto che si occupa di modificare il font di una casella di testo da disegnare o selezionata.
\subsubsezione{Relazioni d'uso di altre componenti}
\` E formata dalle istanze aggregate di \textit{FontSizeSetter} e \textit{FontTypeSetter}.
\subsubsezione{Interfacce e relazioni di uso da altre componenti}
\textit{PropertiesDialog} ne mantiene un'istanza.
\subsubsezione{Attivit\`a svolte e dati trattati}
\leftskip=36pt{Nessuna a questo livello.}

\subsezione{FontSizeSetter}
\subsubsezione{Tipo, obiettivo e funzione del componente}
Classe concreta che rappresenta l'oggetto che si occupa di modificare la dimensione del carattere di una casella di testo da disegnare o selezionata.
\subsubsezione{Relazioni d'uso di altre componenti}
Nessuna.
\subsubsezione{Interfacce e relazioni di uso da altre componenti}
\textit{FontSetter} ne mantiene un'istanza.
\subsubsezione{Attivit\`a svolte e dati trattati}
\leftskip=36pt{Nessuna a questo livello.}

\subsezione{FontTypeSetter}
\subsubsezione{Tipo, obiettivo e funzione del componente}
Classe concreta che rappresenta l'oggetto che si occupa di modificare il tipo di carattere di una casella di testo da disegnare o selezionata, scegliendo tra alcune famiglie predefinite.
\subsubsezione{Relazioni d'uso di altre componenti}
Nessuna.
\subsubsezione{Interfacce e relazioni di uso da altre componenti}
\textit{FontSetter} ne mantiene un'istanza.
\subsubsezione{Attivit\`a svolte e dati trattati}
\leftskip=36pt{Nessuna a questo livello.}

\subsezione{UserClickEvent}
\subsubsezione{Tipo, obiettivo e funzione del componente}
Rappresenta la risposta ad un click dell'utente su un oggetto che implementa tale intefaccia.
\subsubsezione{Relazioni d'uso di altre componenti}
Utilizza un generico evento, istanza della classe \textit{Event}, per la gestione del click del mouse.
\subsubsezione{Interfacce e relazioni di uso da altre componenti}
\`E implementata dalle classi \textit{Toolbar}, \textit{Palette}, \textit{Canvas}, e \textit{Button}. 
\subsubsezione{Attivit\`a svolte e dati trattati}
\begin{elencopuntato}[\normindent]
\item[-]  \textit{onClick} gestisce la pressione del pulsante del mouse su un oggetto che implementa l'interfaccia.
\end{elencopuntato}

\subsezione{Canvas}
\subsubsezione{Tipo, obiettivo e funzione del componente}
Classe che rappresenta il canvas, oggetto sul quale verrano visualizzate le figure disegnate, modificate e spostate dall'utente, tramite l'utilizzo del mouse. 
\subsubsezione{Relazioni d'uso di altre componenti}
Implementa le interfacce \textit{MouseMovePressed} e \textit{MouseMoveReleased}, per tenere traccia della pressione del pulsante sinistro del mouse e del suo trascinamento, oltre a \textit{Widget}.
\subsubsezione{Interfacce e relazioni di uso da altre componenti}
Compone un'istanza di \textit{Visualization}, che si occupa della sua visualizzazione su schermo.
\subsubsezione{Attivit\`a svolte e dati trattati}
\begin{elencopuntato}[\normindent]
\item[-]  \textit{clear} imposta il canvas alla sua condizione iniziale.
\end{elencopuntato}

\subsezione{OnMouseMove}
\subsubsezione{Tipo, obiettivo e funzione del componente}
Rappresenta un generico stato di utilizzo del mouse, specificato meglio dalle interfacce che la estendono.
\subsubsezione{Relazioni d'uso di altre componenti}
Utilizza un'istanza della classe \textit{Event}.
\subsubsezione{Interfacce e relazioni di uso da altre componenti}
\` E esteso dalle interfacce \textit{MouseMovePressed} e \textit{MouseMoveReleased}, per la gestione pi\` u specifica della pressione del pulsante sinistro del mouse.
\subsubsezione{Attivit\`a svolte e dati trattati}
\begin{elencopuntato}[\normindent]
\item[-]  \textit{onMove} gestisce il generico movimento del mouse.
\end{elencopuntato}

\subsezione{MouseMovePressed}
\subsubsezione{Tipo, obiettivo e funzione del componente}
Rappresenta lo stato di pressione del pulsante sinistro del mouse, utile in genere per il dimensionamento di una figura in fase di creazione, del suo ridimensionamento o del suo spostamento sul canvas. 
\subsubsezione{Relazioni d'uso di altre componenti}
Estende \textit{OnMouseMove}.
\subsubsezione{Interfacce e relazioni di uso da altre componenti}
Viene implementato dall'istanza della classe \textit{Canvas}, per gestire il caso del pulsante del mouse premuto.
\subsubsezione{Attivit\`a svolte e dati trattati}
\leftskip=36pt{Nessuna a questo livello.}

\subsezione{MouseMoveReleased}
\subsubsezione{Tipo, obiettivo e funzione del componente}
Rappresenta lo stato di rilascio del pulsante sinistro del mouse a seguito della sua pressione, che in genere sta ad indicare il termine di un'azione effettuata sul canvas.
\subsubsezione{Relazioni d'uso di altre componenti}
Estende \textit{OnMouseMovePressed}.
\subsubsezione{Interfacce e relazioni di uso da altre componenti}
Viene implementato dall'istanza della classe \textit{Canvas}, per gestire il caso del pulsante del mouse premuto.
\subsubsezione{Attivit\`a svolte e dati trattati}
\leftskip=36pt{Nessuna a questo livello.}

\subsezione{Event}
\subsubsezione{Tipo, obiettivo e funzione del componente}
Classe che rappresenta un generico evento generato da un'azione prestabilita, quali movimenti e pressione dei tasti del mouse.
\subsubsezione{Relazioni d'uso di altre componenti}
Nessuna.
\subsubsezione{Interfacce e relazioni di uso da altre componenti}
Viene utilizzato dalle interfacce \textit{OnMouseMove} e \textit{UserClickEvent}, a seguito dell'utilizzo del mouse per effettuare qualche operazione.
\subsubsezione{Attivit\`a svolte e dati trattati}
\leftskip=36pt{Nessuna a questo livello.}

\subsezione{Page}
\subsubsezione{Tipo, obiettivo e funzione del componente}
Classe concreta che rappresenta la pagina web visualizzata dall'utente tramite l'internet browser, contenitore di tutti i widgets componenti l'interfaccia.
\subsubsezione{Relazioni d'uso di altre componenti}
\` E composta da istanze delle classi \textit{Toolbar}, \textit{Button}, \textit{Palette}, \textit{ColourDialog}, \textit{PropertiesDialog} e \textit{Visualization}.
\subsubsezione{Interfacce e relazioni di uso da altre componenti}
Nessuna.
\subsubsezione{Attivit\`a svolte e dati trattati}
\leftskip=36pt{Nessuna a questo livello.}

\subsezione{Visualization}
\subsubsezione{Tipo, obiettivo e funzione del componente}
Classe concreta che si occupa delle operazioni di visualizzazione del canvas su schermo, come la riscalatura a seguito di uno zoom e la corretta visualizzazione della parte ingrandita o rimpicciolita.
\subsubsezione{Relazioni d'uso di altre componenti}
\` E composta da un'istanza della classe \textit{Canvas}. Utilizza istanze di \textit{Scale} e \textit{BoundingRectangle}.
\subsubsezione{Interfacce e relazioni di uso da altre componenti}
\textit{Page} ne mantiene un'istanza.
\subsubsezione{Attivit\`a svolte e dati trattati}
\begin{elencopuntato}[\normindent]
\item[-]  \textit{zoom} metodo che si occupa del riscalamento a seguito di uno zoom sul canvas.
\item[-]  \textit{setVisibleRect} metodo usato per settare l'area di canvas visualizzata.
\item[-]  \textit{refresh} si occupa dell'aggiornamento della situazione del canvas visualizzato.
\end{elencopuntato}

\subsezione{Scale}
\subsubsezione{Tipo, obiettivo e funzione del componente}
Classe concreta che si occupa delle operazioni di riscalamento del canvas su schermo. 
\subsubsezione{Relazioni d'uso di altre componenti}
Nessuna.
\subsubsezione{Interfacce e relazioni di uso da altre componenti}
\textit{Visualization} ne utilizza un'istanza.
\subsubsezione{Attivit\`a svolte e dati trattati}
\leftskip=36pt{Nessuna a questo livello.}

\subsezione{Alert}
\subsubsezione{Tipo, obiettivo e funzione del componente}
Classe concreta che si occupa di avvertire l'utente di un errore verificatosi nel programma.
\subsubsezione{Relazioni d'uso di altre componenti}
Nessuna.
\subsubsezione{Interfacce e relazioni di uso da altre componenti}
Nessuna.
\subsubsezione{Attivit\`a svolte e dati trattati}
\begin{elencopuntato}[\normindent]
\item[-]  \textit{showError} metodo che avverte l'utente di un qualche errore tramite pop-up o un avviso di altro genere.
\end{elencopuntato}


%Parte Dissegna
\subsezione{Figure}
\subsubsezione{Tipo, obiettivo e funzione del componente}
Classe astratta che rappresenta una figura generica nel sistema, ovvero un oggetto disegnabile nel canvas.
\subsubsezione{Relazioni d'uso di altre componenti}
Mantiene un'istanza della classe \textit{BoundingRectangle} per rappresentare i limiti della figura. Contiene anche un'istanza di \textit{Colour} che rappresenta il colore del bordo. Deriva, ma non implementa, l'interfaccia \textit{SVGReadable} e l'interfaccia \textit{SVGWritable}, in quanto le sue sottoclassi dovranno essere serializzabili e deserializabili in file SVG.
\subsubsezione{Interfacce e relazioni di uso da altre componenti}
Viene aggregata dalla classe \textit{FigureSet}.
\subsubsezione{Attivit\`a svolte e dati trattati}
\begin{elencopuntato}[\normindent]
\item[-]  \textit{draw} metodo astratto per disegnare la figura sul canvas.
\item[-]  \textit{getMainPoints} metodo astratto per ottenere i punti della figura manipolabili dall'utente.
\item[-]  \textit{setSelection} permette di selezionare/deselezionare la figura.
\item[-]  \textit{isSelected} dice se la figura \`e correntemente selezionata.
\item[-]  \textit{eachProperty} applica la funzione fn ad ogni propriet\`a della figura.
\item[-]  \textit{clone} crea un oggetto clone appena sopra la figura.
\end{elencopuntato}

\subsezione{FigureSet}
\subsubsezione{Tipo, obiettivo e funzione del componente}
Rappresenta una collezione di figure. Tutte le figure disegnate dall'utente sono contenute in questa collezione, la quale ne permette la manipolazione e l'attraversamento. Per cancellare una figura \`e sufficiente rimuoverla dalla collezione.
\subsubsezione{Relazioni d'uso di altre componenti}
Contiene le istanze di \textit{Figure} correntemente attive.
\subsubsezione{Interfacce e relazioni di uso da altre componenti}
Usata dalla classe \textit{SVGWriter} per serializzare in SVG le figure correnti a dall'interfaccia utente per accedere alle figure. Pu\`o venire generata dalla classe \textit{SVGReader} dopo la deserializzazione di un file SVG.
\subsubsezione{Attivit\`a svolte e dati trattati}
\begin{elencopuntato}[\normindent]
\item[-]  \textit{selectFigure} permette di selezionare una figura contenente il punto passato come argomento, se tale figura esiste.
\item[-]  \textit{each} permette di applicare una funzione ad ogni figura. Viene usato per iterare attraverso la collezione.
\item[-]  \textit{add} aggiunge una figura all'insieme.
\item[-]  \textit{rem} rimuove una figura dall'insieme, se presente.
\end{elencopuntato}

\subsezione{Property}
\subsubsezione{Tipo, obiettivo e funzione del componente}
Rappresenta una propriet\`a generica di una figura. Le propriet\`a influenzano 
l'aspetto delle diverse figure, e sono modificabili dall'utente.
\subsubsezione{Relazioni d'uso di altre componenti}
Nessuna.
\subsubsezione{Interfacce e relazioni di uso da altre componenti}
Implementata da tutte le propriet\`a concrete delle diverse figure.
\subsubsezione{Attivit\`a svolte e dati trattati}
\begin{elencopuntato}[\normindent]
\item[-] \textit{createWidget} crea un widget grafico che permetter\`a all'utente di modificare la propriet\`a.
\end{elencopuntato}

\subsezione{CSSProperty}
\subsubsezione{Tipo, obiettivo e funzione del componente}
Rappresenta una propriet\`a di una figura che necessita di una rappresentazione conforme ai CSS. Questa rappresentazione \`e necessaria per applicare alcune propriet\`a (come ad esempio il colore) al canvas e per la conversione in SVG.
\subsubsezione{Relazioni d'uso di altre componenti}
Nessuna.
\subsubsezione{Interfacce e relazioni di uso da altre componenti}
Implementata da \textit{Colour}, \textit{TextFont} e \textit{TextSize}. 
\subsubsezione{Attivit\`a svolte e dati trattati}
\begin{elencopuntato}[\normindent]
\item[-] \textit{toCSS} genera una stringa che rappresenta la propriet\`a come propriet\`a CSS.
\end{elencopuntato}

\subsezione{Opacity}
\subsubsezione{Tipo, obiettivo e funzione del componente}
Rappresenta l'opacit\`a di un colore. Mantiene un unico valore che varia nell'intervallo [0, 1]. Se il valore \`e 0 la figura \`e completamente trasparente, se
il valore \`e 1, la figura \`e completamente opaca, ovvero non permette la visualizzazione delle figure sottostanti.
\subsubsezione{Relazioni d'uso di altre componenti}
Nessuna.
\subsubsezione{Interfacce e relazioni di uso da altre componenti}
\textit{Colour} ne mantiene un'istanza.
\subsubsezione{Attivit\`a svolte e dati trattati}
\leftskip=36pt{Nessuna a questo livello.}

\subsezione{EdgeNumber}
\subsubsezione{Tipo, obiettivo e funzione del componente}
Rappresenta il numero degli spigoli di un poligono. Questo numero non pu\`o essere inferiore a 3.
\subsubsezione{Relazioni d'uso di altre componenti}
Implementa \textit{Property}.
\subsubsezione{Interfacce e relazioni di uso da altre componenti}
\textit{Polygon} ne mantiene un'istanza.
\subsubsezione{Attivit\`a svolte e dati trattati}
\leftskip=36pt{Nessuna a questo livello.}

\subsezione{Colour}
\subsubsezione{Tipo, obiettivo e funzione del componente}
Rappresenta un colore come una terna RGB pi\`u l'opacit\`a del colore stesso. Questa rappresentazione trova una corrispondenza diretta sia nel tag <canvas>, sia nel formato SVG.
\subsubsezione{Relazioni d'uso di altre componenti}
Implementa \textit{Property} e \textit{CSSProperty}. Mantiene un'istanza di \textit{Opacity} che rappresenta l'opacit\`a del colore.
\subsubsezione{Interfacce e relazioni di uso da altre componenti}
\textit{Figure} ne mantiene un'istanza per rappresentare il colore del bordo.
\subsubsezione{Attivit\`a svolte e dati trattati}
\leftskip=36pt{Nessuna a questo livello.}

\subsezione{FillColour}
\subsubsezione{Tipo, obiettivo e funzione del componente}
Rappresenta un colore di riempimento.
\subsubsezione{Relazioni d'uso di altre componenti}
Deriva da \textit{Colour}.
\subsubsezione{Interfacce e relazioni di uso da altre componenti}
\textit{Circle}, \textit{Polygon} e \textit{Rectangle} ne mantengono un'istanza.
\subsubsezione{Attivit\`a svolte e dati trattati}
\leftskip=36pt{Nessuna a questo livello.}

\subsezione{TextProperty}
\subsubsezione{Tipo, obiettivo e funzione del componente}
Propriet\`a di un testo grafico.
\subsubsezione{Relazioni d'uso di altre componenti}
Deriva da \textit{Property}.
\subsubsezione{Interfacce e relazioni di uso da altre componenti}
\textit{Text} ne mantiene delle istanze.
\subsubsezione{Attivit\`a svolte e dati trattati}
\leftskip=36pt{Nessuna a questo livello.}

\subsezione{TextColour}
\subsubsezione{Tipo, obiettivo e funzione del componente}
Rappresenta il colore di un testo.
\subsubsezione{Relazioni d'uso di altre componenti}
Deriva da \textit{Colour} e da \textit{TextProperty}.
\subsubsezione{Interfacce e relazioni di uso da altre componenti}
Nessuna.
\subsubsezione{Attivit\`a svolte e dati trattati}
\leftskip=36pt{Nessuna a questo livello.}

\subsezione{TextFont}
\subsubsezione{Tipo, obiettivo e funzione del componente}
Rappresenta il font di un testo. Mantiene il nome del font.
\subsubsezione{Relazioni d'uso di altre componenti}
Deriva da \textit{TextProperty} e implementa \textit{CSSProperty}.
\subsubsezione{Interfacce e relazioni di uso da altre componenti}
Nessuna.
\subsubsezione{Attivit\`a svolte e dati trattati}
\leftskip=36pt{Nessuna a questo livello.}

\subsezione{TextSize}
\subsubsezione{Tipo, obiettivo e funzione del componente}
Rappresenta la grandezza di un testo. La grandezza viene espressa in pixel.
\subsubsezione{Relazioni d'uso di altre componenti}
Deriva da \textit{TextProperty} e implementa \textit{CSSProperty}.
\subsubsezione{Interfacce e relazioni di uso da altre componenti}
Nessuna.
\subsubsezione{Attivit\`a svolte e dati trattati}
\leftskip=36pt{Nessuna a questo livello.}

\subsezione{BoundingRectangle}
\subsubsezione{Tipo, obiettivo e funzione del componente}
Rappresenta la posizione e le dimensioni massime di una figura. Modificando il
\textit{BoundingRectangle} \`e possibile spostare, ingrandire e deformare una figura.
\`E responsabilit\`a delle singole figure, all'interno del metodo \textit{draw}, di disegnarsi all'interno dei limiti.
\subsubsezione{Relazioni d'uso di altre componenti}
Implementa \textit{Property}. Ogni istanza di \textit{Figure} ne mantiene un'istanza.
\subsubsezione{Interfacce e relazioni di uso da altre componenti}
Nessuna.
\subsubsezione{Attivit\`a svolte e dati trattati}
\leftskip=36pt{Nessuna a questo livello.}

\subsezione{Rotation}
\subsubsezione{Tipo, obiettivo e funzione del componente}
Propriet\`a che rappresenta il grado di rotazione di una figura. La rotazione avviene seguendo il centro della circonferenza che iscrive la figura; pi\`u precisamente, la rotazione viene svolta attraverso l'uso del centro del BoundingRectangle. 
\subsubsezione{Relazioni d'uso di altre componenti}
Deriva da \textit{Property}.
\subsubsezione{Interfacce e relazioni di uso da altre componenti}
\textit{figure} ne mantiene una o pi\'u istanze.
\subsubsezione{Attivit\`a svolte e dati trattati}
\leftskip=36pt{Nessuna a questo livello.}

\subsezione{Circle}
\subsubsezione{Tipo, obiettivo e funzione del componente}
Rappresenta un cerchio o un'ellisse, a seconda che il proprio \textit{BoundingRectangle} sia un quadrato oppure un rettangolo. Viene disegnato in modo da essere inscritto al \textit{BoundingRectangle}.
\subsubsezione{Relazioni d'uso di altre componenti}
Deriva da \textit{Figure} e ne implementa i metodi astratti. Mantiene un'istanza di \textit{FillColour}.
\subsubsezione{Interfacce e relazioni di uso da altre componenti}
Nessuna.
\subsubsezione{Attivit\`a svolte e dati trattati}
\leftskip=36pt{Nessuna a questo livello.}

\subsezione{Rectangle}
\subsubsezione{Tipo, obiettivo e funzione del componente}
Rappresenta un rettangolo od un quadrato.
\subsubsezione{Relazioni d'uso di altre componenti}
Deriva da \textit{Figure} e ne implementa i metodi astratti. Mantiene un'istanza di \textit{FillColour}.
\subsubsezione{Interfacce e relazioni di uso da altre componenti}
Nessuna.
\subsubsezione{Attivit\`a svolte e dati trattati}
\leftskip=36pt{Nessuna a questo livello.}

\subsezione{Line}
\subsubsezione{Tipo, obiettivo e funzione del componente}
Rappresenta una linea generica.
\subsubsezione{Relazioni d'uso di altre componenti}
Deriva da \textit{Figure}.
\subsubsezione{Interfacce e relazioni di uso da altre componenti}
Nessuna.
\subsubsezione{Attivit\`a svolte e dati trattati}
\leftskip=36pt{Nessuna a questo livello.}

\subsezione{StraightLine}
\subsubsezione{Tipo, obiettivo e funzione del componente}
Rappresenta una linea diritta. La linea v\`a dall'angolo in alto a sinistra all'angolo in basso a destra del \textit{BoundingRectangle}.
\subsubsezione{Relazioni d'uso di altre componenti}
Deriva da \textit{Line} e ne implementa i metodi astratti.
\subsubsezione{Interfacce e relazioni di uso da altre componenti}
Nessuna.
\subsubsezione{Attivit\`a svolte e dati trattati}
\leftskip=36pt{Nessuna a questo livello.}

\subsezione{FreeLine}
\subsubsezione{Tipo, obiettivo e funzione del componente}
Rappresenta una linea a mano libera. Viene implementata tramite delle curve di bezier brevi e ravvicinate tra di loro.
\subsubsezione{Relazioni d'uso di altre componenti}
Deriva da \textit{Line} e ne implementa i metodi astratti. Mantiene una lista di punti. I punti prendono come origine l'angolo in alto a sinistra del \textit{BoundingRectangle} e presuppongono un quadrato alto 1 e largo 1, in modo da essere indipendenti dal \textit{BoundingRectangle}. I punti verranno poi convertiti nei loro valori assoluti, a seconda del \textit{BoundingRectangle}, prima di disegnare la linea. Qualora ripetere questa computazione ad ogni refresh fosse troppo dispendioso, si potrebbe ricorrere ad un sistema di caching.
\subsubsezione{Interfacce e relazioni di uso da altre componenti}
Usata come base da \textit{BezierCurve}.
\subsubsezione{Attivit\`a svolte e dati trattati}
\begin{elencopuntato}[\normindent]
\item[-] \textit{eachPoint} applica una funzione ad ogni nodo della linea.
\end{elencopuntato}

\subsezione{BezierCurve}
\subsubsezione{Tipo, obiettivo e funzione del componente}
Rappresenta una linea curva, disegnata tramite curve di bezier.
\subsubsezione{Relazioni d'uso di altre componenti}
Deriva da \textit{FreeLine}. A differenza di \textit{FreeLine} la lista di punti interni contiene un numero ridotto di punti.
\subsubsezione{Interfacce e relazioni di uso da altre componenti}
Nessuna.
\subsubsezione{Attivit\`a svolte e dati trattati}
\leftskip=36pt{Nessuna a questo livello.}

\subsezione{Polygon}
\subsubsezione{Tipo, obiettivo e funzione del componente}
Rappresenta un poligono regolare.
\subsubsezione{Relazioni d'uso di altre componenti}
Deriva da \textit{Figure} e ne implementa i metodi astratti. Mantiene un'istanza di \textit{FillColour} e di \textit{EdgeNumber} che ne identifica il numero di lati.
\subsubsezione{Interfacce e relazioni di uso da altre componenti}
Nessuna.
\subsubsezione{Attivit\`a svolte e dati trattati}
\leftskip=36pt{Nessuna a questo livello.}

\subsezione{SVGWriter}
\subsubsezione{Tipo, obiettivo e funzione del componente}
Permette di creare un documento SVG.
\subsubsezione{Relazioni d'uso di altre componenti}
Accede alle figure correnti attraverso un'istanza di \textit{FigureSet}.
\subsubsezione{Interfacce e relazioni di uso da altre componenti}
Nessuna.
\subsubsezione{Attivit\`a svolte e dati trattati}
\begin{elencopuntato}[\normindent]
\item[-] \textit{write} crea un documento SVG che contiene le figure correntemente disegnate.
\end{elencopuntato}

\subsezione{SVGWritable}
\subsubsezione{Tipo, obiettivo e funzione del componente}
Rappresenta un oggetto convertibile in SVG.
\subsubsezione{Relazioni d'uso di altre componenti}
Usa \textit{SVGGenerator} per generare i comandi SVG.
\subsubsezione{Interfacce e relazioni di uso da altre componenti}
Implementata dalle classi che rappresentano figure.
\subsubsezione{Attivit\`a svolte e dati trattati}
\begin{elencopuntato}[\normindent]
\item[-] \textit{toSVG} converte in SVG. Per trascriversi in un documento SVG l'oggetto potr\`a usare solo i metodi messi a disposizione dall'\textit{SVGGenerator} passato come parametro.
\end{elencopuntato}

\subsezione{SVGGenerator}
\subsubsezione{Tipo, obiettivo e funzione del componente}
Rappresenta un documento SVG in costruzione. Mantiene al suo interno una rappresentazione del documento finora generato e mette a disposizione dei metodi per scrivere nel documento.
\subsubsezione{Relazioni d'uso di altre componenti}
Nessuna.
\subsubsezione{Interfacce e relazioni di uso da altre componenti}
Usato da \textit{SVGWriter} e dalle classi che implementano \textit{SVGWritable} per realizzare un documento SVG.
\subsubsezione{Attivit\`a svolte e dati trattati}
\begin{elencopuntato}[\normindent]
\item[-] \textit{startCommand} apre un comando con il nome dato.
\item[-] \textit{attr} aggiunge un attributo con il nome e il valore dato al comando corrente.
\item[-] \textit{endCommand} chiude il comando corrente.
\item[-] \textit{flush} ritorna il documento finora generato sotto forma di stringa.
\end{elencopuntato}

\subsezione{XMLParser}
\subsubsezione{Tipo, obiettivo e funzione del componente}
Un parser SVG. Costruisce il DOM di un documento XML a partire dalla sua rappresentazione testuale.
\subsubsezione{Relazioni d'uso di altre componenti}
Nessuna.
\subsubsezione{Interfacce e relazioni di uso da altre componenti}
Usato da \textit{SVGReader}.
\subsubsezione{Attivit\`a svolte e dati trattati}
\leftskip=36pt{Nessuna a questo livello.}

\subsezione{SVGTag}
\subsubsezione{Tipo, obiettivo e funzione del componente}
Rappresenta un tag (ovvero un comando) SVG. Un tag \`e caratterizzato dal nome, da un'insieme di coppie attributo-valore, e da eventuali tag figli.
\subsubsezione{Relazioni d'uso di altre componenti}
Nessuna.
\subsubsezione{Interfacce e relazioni di uso da altre componenti}
Usato da \textit{SVGReader} e dalle classi che implementano \textit{SVGReadable}.
\subsubsezione{Attivit\`a svolte e dati trattati}
\leftskip=36pt{Nessuna a questo livello.}

\subsezione{SVGReadable}
\subsubsezione{Tipo, obiettivo e funzione del componente}
Rappresenta un oggetto caricabile da un documento SVG.
\subsubsezione{Relazioni d'uso di altre componenti}
Nessuna.
\subsubsezione{Interfacce e relazioni di uso da altre componenti}
Implementata dalle classi che rappresentano figure.
\subsubsezione{Attivit\`a svolte e dati trattati}
\begin{elencopuntato}[\normindent]
\item[-] \textit{fromSVG} inizializza, modificando il proprio stato interno, se stesso a partire da un tag SVG.
\end{elencopuntato}

\subsezione{SVGReadableClass}
\subsubsezione{Tipo, obiettivo e funzione del componente}
Metaclasse di \textit{SVGReadable}.
\subsubsezione{Relazioni d'uso di altre componenti}
Nessuna.
\subsubsezione{Interfacce e relazioni di uso da altre componenti}
Usata da \textit{SVGElementRegistry} per associare il nome di un tag a una classe che implementa \textit{SVGReadable}.
\subsubsezione{Attivit\`a svolte e dati trattati}
\leftskip=36pt{Nessuna a questo livello.}

\subsezione{SVGReader}
\subsubsezione{Tipo, obiettivo e funzione del componente}
Permette di caricare un documento SVG.
\subsubsezione{Relazioni d'uso di altre componenti}
Usa \textit{XMLParser} per fare il parsing del documento testuale, crea \textit{SVGTag} e usa \textit{SVGElementRegistry} per creare una figura a partire da un tag SVG.
\subsubsezione{Interfacce e relazioni di uso da altre componenti}
Nessuna.
\subsubsezione{Attivit\`a svolte e dati trattati}
\begin{elencopuntato}[\normindent]
\item[-] \textit{read} prende un documento SVG e ritorna un'istanza di \textit{FigureSet} che lo rappresenta.
\end{elencopuntato}

\subsezione{SVGElementRegistry}
\subsubsezione{Tipo, obiettivo e funzione del componente}
Associa nomi di tag SVG a classi (non istanze) che implementano \textit{SVGReadable} e ne permette l'istanziazione. 
\subsubsezione{Relazioni d'uso di altre componenti}
Usa \textit{SVGReadableClass} per mantenere un riferimento ad una classe per poterla poi istanziare.
\subsubsezione{Interfacce e relazioni di uso da altre componenti}
Usata da \textit{SVGReader} durante la conversione di un documento SVG per creare oggetti del tipo corretto a partire da un tag SVG. Gli oggetti creati andranno poi inizializzati. Ogni classe che implementa \textit{SVGReadable} deve registrarsi.
\subsubsezione{Attivit\`a svolte e dati trattati}
\begin{elencopuntato}[\normindent]
\item[-] \textit{register} associa il nome di un tag SVG ad una classe che implementa \textit{SVGReadable}.
\item[-] \textit{makeFigureClassFromTag} crea un'istanza della classe associata al nome di tag passato come parametro.
\end{elencopuntato}

\subsezione{Manual}
\subsubsezione{Tipo, obiettivo e funzione del componente}
Rappresenta il manuale utente.
\subsubsezione{Relazioni d'uso di altre componenti}
Usa \textit{StaticDocument} per ottenere il file contenente il manuale.
\subsubsezione{Interfacce e relazioni di uso da altre componenti}
Nessuna.
\subsubsezione{Attivit\`a svolte e dati trattati}
\begin{elencopuntato}[\normindent]
\item[-] \textit{get} ritorna il manuale.
\end{elencopuntato}

\subsezione{Point}
\subsubsezione{Tipo, obiettivo e funzione del componente}
Un punto sul piano.
\subsubsezione{Relazioni d'uso di altre componenti}
Nessuna.
\subsubsezione{Interfacce e relazioni di uso da altre componenti}
Nessuna.
\subsubsezione{Attivit\`a svolte e dati trattati}
\leftskip=36pt{Nessuna a questo livello.}

\subsezione{PointSeq}
\subsubsezione{Tipo, obiettivo e funzione del componente}
Una sequenza di \textit{Point}.
\subsubsezione{Relazioni d'uso di altre componenti}
Nessuna.
\subsubsezione{Interfacce e relazioni di uso da altre componenti}
Nessuna.
\subsubsezione{Attivit\`a svolte e dati trattati}
\leftskip=36pt{Nessuna a questo livello.}

\subsezione{Object}
\subsubsezione{Tipo, obiettivo e funzione del componente}
Superclasse di ogni classe.
\subsubsezione{Relazioni d'uso di altre componenti}
Nessuna.
\subsubsezione{Interfacce e relazioni di uso da altre componenti}
Ereditata automaticamente da ogni classe.
\subsubsezione{Attivit\`a svolte e dati trattati}
\leftskip=36pt{Nessuna a questo livello.}

\subsezione{Function}
\subsubsezione{Tipo, obiettivo e funzione del componente}
Una funzione di prima classe che accetta un argomento di tipo qualsiasi. La funzione pu\`o essere anonima o no.
\subsubsezione{Relazioni d'uso di altre componenti}
Nessuna.
\subsubsezione{Interfacce e relazioni di uso da altre componenti}
Usata dai metodi di iterazione, come ad esempio \textit{each} di \textit{FigureSet}, per applicare una certa operazione ad ogni elemento di una collezione.
\subsubsezione{Attivit\`a svolte e dati trattati}
\begin{elencopuntato}[\normindent]
\item[-] \textit{call} richiama la funzione.
\end{elencopuntato}

%Parte Dal Bosco
\subsezione{Document}
\subsubsezione{Tipo, obiettivo e funzione del componente}
Permette di salvare e caricare file.
\subsubsezione{Relazioni d'uso di altre componenti}
Usa \textit{Request} per effettuare una richiesta al server.
\subsubsezione{Interfacce e relazioni di uso da altre componenti}
Nessuna.
\subsubsezione{Attivit\`a svolte e dati trattati}
\begin{elencopuntato}[\normindent]
\item[-] \textit{save} chiede il salvataggio di un file.
\item[-] \textit{load} chiede il caricamento di un file.
\end{elencopuntato}

\subsezione{Request}
\subsubsezione{Tipo, obiettivo e funzione del componente}
Contatta il server per una operazione richiesta dal client.
\subsubsezione{Relazioni d'uso di altre componenti}
Usa \textit{Server} per inoltrare una richiesta e usa \textit{ConnectionError} in caso di errore di connessione.
\subsubsezione{Interfacce e relazioni di uso da altre componenti}
Usata da \textit{Document} e da \textit{StaticDocument} per effettuare una richiesta al server.
\subsubsezione{Attivit\`a svolte e dati trattati}
\begin{elencopuntato}[\normindent]
\item[-] \textit{acceptRequest} contatta il server in base alla richiesta ricevuta.
\end{elencopuntato}

\subsezione{Server}
\subsubsezione{Tipo, obiettivo e funzione del componente}
Permette le operazioni di salvataggio, caricamento, fornisce la pagina HTML e il manuale utente.
\subsubsezione{Relazioni d'uso di altre componenti}
Usa \textit{File} per salvare o ricevere un file, usa \textit{Clean} per eliminare i file in eccesso e usa \textit{ServerError} per gestire un errore sul server.
\subsubsezione{Interfacce e relazioni di uso da altre componenti}
Usata da \textit{Request} per inoltrare una richiesta del client.
\subsubsezione{Attivit\`a svolte e dati trattati}
\begin{elencopuntato}[\normindent]
\item[-] \textit{getSave} restitusice un riferimento al file salvato.
\item[-] \textit{getLoad} restitusice un riferimento al file caricato da disco.
\item[-] \textit{getHTML} restitusice la pagina HTML o il manuale utente.
\end{elencopuntato}

\subsezione{File}
\subsubsezione{Tipo, obiettivo e funzione del componente}
Rappresenta un file su disco.
\subsubsezione{Relazioni d'uso di altre componenti}
Usa \textit{FileNotFound} per gestire un errore di file non trovato su disco.
\subsubsezione{Interfacce e relazioni di uso da altre componenti}
Usata da \textit{Server} per restituire un file a seguito di una richiesta e usata da \textit{Clean} per eliminare un file su disco.
\subsubsezione{Attivit\`a svolte e dati trattati}
\begin{elencopuntato}[\normindent]
\item[-] \textit{getFile} restitusice un file su disco.
\item[-] \textit{deleteFile} elimina un file su disco.
\end{elencopuntato}

\subsezione{Clean}
\subsubsezione{Tipo, obiettivo e funzione del componente}
Permette di eliminare i file in eccesso sul server per evitare un suo intasamento.
\subsubsezione{Relazioni d'uso di altre componenti}
Usa \textit{File} per ricevere un file da eliminare.
\subsubsezione{Interfacce e relazioni di uso da altre componenti}
Usata da \textit{Server} quando sono presenti file in eccesso.
\subsubsezione{Attivit\`a svolte e dati trattati}
\begin{elencopuntato}[\normindent]
\item[-] \textit{sweeper} permette di eliminare i file superflui del server.
\end{elencopuntato}

\subsezione{StaticDocument}
\subsubsezione{Tipo, obiettivo e funzione del componente}
Permette di richiedere il manuale utente.
\subsubsezione{Relazioni d'uso di altre componenti}
Usa \textit{Request} per effettuare la richiesta al server.
\subsubsezione{Interfacce e relazioni di uso da altre componenti}
Nessuna.
\subsubsezione{Attivit\`a svolte e dati trattati}
\begin{elencopuntato}[\normindent]
\item[-] \textit{get} permette di restituire il file del manuale utente.
\end{elencopuntato}

\subsezione{CommunicationError}
\subsubsezione{Tipo, obiettivo e funzione del componente}
Rappresenta un errore di comunicazione.
\subsubsezione{Relazioni d'uso di altre componenti}
Nessuna.
\subsubsezione{Interfacce e relazioni di uso da altre componenti}
Implementata da \textit{ConnectionError} e da \textit{ServerError} per la getione di errori di comunicazione.
\subsubsezione{Attivit\`a svolte e dati trattati}
\leftskip=36pt{Nessuna a questo livello.}

\subsezione{ConnectionError}
\subsubsezione{Tipo, obiettivo e funzione del componente}
Rappresenta un errore di connessione.
\subsubsezione{Relazioni d'uso di altre componenti}
Implementa \textit{CommunicationError}.
\subsubsezione{Interfacce e relazioni di uso da altre componenti}
Usata da \textit{Request} nel caso non riesca a gestire una richiesta del client.
\subsubsezione{Attivit\`a svolte e dati trattati}
\begin{elencopuntato}[\normindent]
\item[-] \textit{getConnectionError} permette di restituire un errore di connessione.
\end{elencopuntato}

\subsezione{ServerError}
\subsubsezione{Tipo, obiettivo e funzione del componente}
Rappresenta un errore del server.
\subsubsezione{Relazioni d'uso di altre componenti}
Implementa \textit{CommunicationError}.
\subsubsezione{Interfacce e relazioni di uso da altre componenti}
Usata da \textit{Server} nel caso non riesca ad accettare altre richieste.
\subsubsezione{Attivit\`a svolte e dati trattati}
\begin{elencopuntato}[\normindent]
\item[-] \textit{getServerError} permette di restituire un errore del server.
\end{elencopuntato}

\subsezione{FileNotFound}
\subsubsezione{Tipo, obiettivo e funzione del componente}
Rappresenta un errore di file non trovato su disco.
\subsubsezione{Relazioni d'uso di altre componenti}
Nessuna.
\subsubsezione{Interfacce e relazioni di uso da altre componenti}
Usata da \textit{File} nel caso non venga trovato su disco un file richiesto.
\subsubsezione{Attivit\`a svolte e dati trattati}
\begin{elencopuntato}[\normindent]
\item[-] \textit{getFileError} permette di restituire un errore di file su disco non trovato.
\end{elencopuntato}

\sezione{Stime di fattibilit\`a e di bisogno di risorse}
Il maggiore collo di bottiglia del sistema \`e rappresentato dal server, in quanto una singola macchina pu\`o trovarsi a gestire un numero elevato di richieste. Non dovrebbero comunque esserci problemi in quanto le funzionalit\`a implementate lato server sono minime, e quindi ogni singolo client comporta un dispendio di risorse, in termini di tempo di calcolo, di spazio su disco e di banda, minimo. 
Un secondo problema prestazionale \`e rappresentato dal fatto che l'applicazione sar\`a forzatamente scritta in \underline{JavaScript}, la cui implementazione \`e estremamente scadente dal punto di vista delle prestazioni in quasi tutti i browser, fatta eccezione per Google Chrome, l'unico a fornire un sistema di compilazione \underline{JIT}. Per questo, durante l'implementazione si dovr\`a porre una costante attenzione alle risorse richieste dai diversi metodi. La scelta del linguaggio JavaScript dipende dal fatto che \`e l'unico linguaggio supportato lato client, e una scelta diversa obbligherebbe quindi o a spostare la logica dell'applicazione lato server o ad utilizzare un compilatore dal linguaggio prescelto al JavaScript. La prima opzione aumenterebbe in modo notevole la comunicazione necessaria tra client e server, degradando la reattivit\`a dell'applicazione e aumentando le probabilit\`a di verificarsi di un errore e dei problemi causati dall'errore stesso. La seconda opzione non \`e desidarabile, in quanto la maggiore distanza tra sorgente e codice eseguito renderebbe molto difficile il debugging dell'applicazione.
Un altro problema \`e la necessit\`a di supportare un'ampia gamma di browser. Ogni browser non sempre implementa i vari standard in modo conforme, con il risultato che un'applicazione funzionante correttamente su un browser potrebbe avere problemi in un altro. Per ridurre al minimo questo problema, si utilizzer\`a la liberia \textit{JQuery} che gestisce una buona parte delle discrepanze tra i browser, si seguir\`a lo standard JavaScript versione 1.5 in quanto \`e l'unica ben supportata da tutti e si effettueranno test frequenti su tutti i diversi browser.
\newpage
\sezione{Tracciamento della relazione componenti-requisiti}
\begin{table}[h]
\begin{center}
     \begin{tabular}
           {@{\extracolsep{\fill}}|c|c|}
     \hline
     \multicolumn{2}{|c|}{ \textbf{Componenti GUI} } \\
     \hline
%%%%%%%%%%%%%%INTESTAZIONE COLONNE%%%%%%%%%%%%%%%%%%%%%%%%%%%%%%%%
      \textbf{Componenti} & \textbf{Requisiti} \\
%%%%%%%%%%%%%%FINE INTESTAZIONE COLONNE%%%%%%%%%%%%%%%%%%%%%%%%%%%%%%%%%%%%%
      \hline
     FontSetter & RFO-7 \\
     \hline
     ColourDialog & RFO-9 \\
     \hline
     Palette & RFO-9 \\
     \hline
     Visualization & RFO-13, RFO-14 \\
     \hline
     Toolbar & RFO-11, RIO-1 \\
     \hline
     PropertiesDialog & RFO-12 \\ 
     \hline
     RotationSetter & RFF-3 \\
     \hline
     Canvas & RFO-17 \\
     \hline
     BoundingRectangle & RFO-10 \\
     \hline
     OnMouseMove & RFO-8 \\
     \hline
     UserClickEvent & RFO-8 \\
         
%%%%%%%%%%% PARTE DA MODIFICARE %%%%%%%%%%%%%%%%
    \hline %%FINE RIGA
%%%%%%%%%%% FINE PARTE DA MODIFICARE %%%%%%%%%%%%%%%%%%%%%%%%%%%%%%%%%%%%%%%%
    \end{tabular}
  \caption{Componenti GUI - Requisiti} %INSERIRE DIDASCALIA - SE NECESSARIA -
  \label{tab:requisitiGUI}
  \end{center}
\end{table}


\begin{table}[h]
\begin{center}
     \begin{tabular}
           {@{\extracolsep{\fill}}|c|c|}
      		\hline
           \multicolumn{2}{|c|}{ \textbf{Componenti ApplicationLogic} } \\
     \hline
%%%%%%%%%%%%%%INTESTAZIONE COLONNE%%%%%%%%%%%%%%%%%%%%%%%%%%%%%%%%
      \textbf{Componenti} & \textbf{Requisiti} \\
%%%%%%%%%%%%%%FINE INTESTAZIONE COLONNE%%%%%%%%%%%%%%%%%%%%%%%%%%%%%%%%%%%%%
      \hline
     SVGWriter & RFO-15 \\
     \hline
     SVGWritable & RFO-15\\
     \hline
     SVGReader & RFO-16\\
     \hline
     SVGGenerator & RFO-15, RFO-16\\
     \hline
     SVGElementRegistry & RFO-16\\
     \hline
     SVGReadable & RFO-16\\
     \hline
     SVGReadableClass & RFO-16\\
     \hline
     SVGTag & RFO-16\\
     \hline
     XMLParser & RFO-16\\
     \hline
     FigureSet & RFO-8, RFO-12, RFO-18 \\
     \hline
     Figure & RFO-8, RFO-12, RFD-3 \\
     \hline
     Manual & RIO-2, RIO-3 \\
     \hline
     StraightLine & RFO-1 \\
     \hline
     BezierCurve & RFO-2 \\
     \hline
     Rectangle & RFO-3 \\
     \hline
     Polygon & RFO-4 \\ 
     \hline
     EdgeNumebr & RFO-4 \\
     \hline
     Circle & RFO-5 \\
     \hline
     Text & RFO-6, RFO-7 \\
     \hline
     TextProperty & RFO-7 \\
     \hline
     TextFont & RFO-7 \\
     \hline
     TextSize & RFO-7\\
     \hline
     Colour & RFO-9 \\
     \hline
     FillColour & RFO-9 \\
     \hline
     Rotation & RFF-3 \\
     \hline
     TextColour & RFO-9 \\
     \hline
     Opacity & RFO-9 \\
     \hline
     CssProperty & RFO-9 \\
     \hline
     BoundingRectangle & RFO-10 \\
     \hline
     Property & RFO-12 \\
     \hline
     FreeLine & RFD-1 \\ 
     \hline
     Line & RFO-1, RFD-1 \\
%%%%%%%%%%% PARTE DA MODIFICARE %%%%%%%%%%%%%%%%
    \hline %%FINE RIGA
%%%%%%%%%%% FINE PARTE DA MODIFICARE %%%%%%%%%%%%%%%%%%%%%%%%%%%%%%%%%%%%%%%%
    \end{tabular}
  \caption{Componenti ApplicationLogic - Requisiti} %INSERIRE DIDASCALIA - SE NECESSARIA -
  \label{tab:requisitiAL}
  \end{center}
\end{table}

\begin{table}[h]
\begin{center}
     \begin{tabular}
           {@{\extracolsep{\fill}}|c|c|}
           \hline
           \multicolumn{2}{|c|}{ \textbf{Componenti DocBackend} }\\
     \hline
%%%%%%%%%%%%%%INTESTAZIONE COLONNE%%%%%%%%%%%%%%%%%%%%%%%%%%%%%%%%
      \textbf{Componenti} & \textbf{Requisiti} \\
%%%%%%%%%%%%%%FINE INTESTAZIONE COLONNE%%%%%%%%%%%%%%%%%%%%%%%%%%%%%%%%%%%%%
      \hline
     Document & RFO-15, RFO-16 \\
     \hline
     Request & RFO-15, RFO-16 \\
     \hline
     CommunicationError & RFO-15, RFO-16 \\
     \hline
     ConnectionError & RFO-15, RFO-16 \\
     \hline
     ServerError & RFO-15, RFO-16 \\
     \hline
     Server & RFO-15, RFO-16, RIO-2, RIO-3 \\
     \hline
     StaticDocument & RFO-15, RFO-16 \\
     \hline
     FileNotFound & RFO-15 \\
     \hline
     File & RFO-15 \\
     \hline
     Clean & RFO-15 \\
         
%%%%%%%%%%% PARTE DA MODIFICARE %%%%%%%%%%%%%%%%
    \hline %%FINE RIGA
%%%%%%%%%%% FINE PARTE DA MODIFICARE %%%%%%%%%%%%%%%%%%%%%%%%%%%%%%%%%%%%%%%%
    \end{tabular}
  \caption{Componenti DocBackend - Requisiti} %INSERIRE DIDASCALIA - SE NECESSARIA -
  \label{tab:requisitiDocBackend}
  \end{center}
\end{table}

\end{document}
