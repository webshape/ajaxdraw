\input{../TeX/base} %BASE!!!
 
\title{\TITOLODOC}
\author{Carollo Mirko}
 
\begin{document}
 
\renewcommand{\insertversion}{1.1} %INSERIRE LA VERSIONE QUI DENTRO STILE x.x.xx
\renewcommand{\TITOLODOC}{Piano di qualifica} %INSERIRE IL TITOLO DEL DOCUMENTO DA FAR COMPARIRE A PIE PAGINA
\renewcommand{\glosspath}{.\glossario} %INSERIRE PERCORSO RELATIVO
 
%%%%%%%%%%%%%%%%%%%%%%PARTE DA NON MODIFICARE%%%%%%%%%%%%%%%%%
\begin{titlepage}
\begin{center}
  \begin{Large}  \today \end{Large}
\end{center}
 
\vspace{20pt}
 
\begin{center}
  \begin{Huge}
        \textbf{\ajax}
  \end{Huge}
\end{center}      
 
\begin{center}
  \begin{large}
        \textbf{Software per il Disegno Grafico\\ in Tecnologie Web}
  \end{large}
\end{center}      
 
\vspace{20pt}
 
\begin{center}
\includegraphics[width=150pt]{../logo/logo}
\end{center}
 
\vspace{170pt}
\begin{center} %INSERIRE ALL'INTERNO IL TITOLO DOCUMENTO CHE COMPARIRA NELLA PAGINA INIZIALE        
  \begin{Huge}
        \textbf{\TITOLODOC}
  \end{Huge}
      \\
\end{center}
\vspace{210pt}
\begin{center}
Versione: \insertversion
\end{center}
\end{titlepage}
 
\newpage
%%%%%%%%%%%%%%%%%%%%%%FINE PARTE DA NON MODIFICARE%%%%%%%%%%%%%%%%%
 
\begin{center} %INSERIRE ALL'INTERNO IL TITOLO DOCUMENTO CHE COMPARIRA NELLA PAGINA INIZIALE
  \begin{Huge}  
        \textbf{\TITOLODOC}
      \\
  \end{Huge}
\end{center}
 
%\setlength{\parindent}{18pt} %settato indentazione di default
\section*{\LARGE Sommario:} %SEZIONE SOMMARIO
\indent \indent
Il presente documento descrive le strategie per la gestione della qualit\`a adottate dall'azienda WebShape, e comprende un resoconto delle relative attivit\`a effettivamente svolte.
 
\section*{\LARGE Stato del documento:}
\indent \indent
  Formale Esterno
 
\section*{\LARGE Redazione:}
  \begin{table}[!h]
    \begin{center}
      \begin{tabular}
        {|c|c|}
        \hline
        %%%%%%%%%%%%%%INTESTAZIONE COLONNE%%%%%%%%%%%%%%%%%%%%%%%%%%%%%%%%
        \multicolumn{2}{|c|}{ \textbf{Redazione} } \\
        \hline
        \textbf{Fase} & \textbf{Redattori} \\
        %%%%%%%%%%%%%%FINE INTESTAZIONE COLONNE%%%%%%%%%%%%%%%%%%%%%%%%%%%%%%%%%%%%%%
        \hline
        %%%%%%%%%%% PARTE DA MODIFICARE %%%%%%%%%%%%%%%%%%%%%%%%%%%%%%%%%%%%%%%%%%    
        Pre-RR & Dissegna Stefano\\
        \hline
        RR-RPP & Carollo Mirko\\
                    & \\
        \hline
        %%%%%%%%%%% FINE PARTE DA MODIFICARE %%%%%%%%%%%%%%%%%%%%%%%%%%%
      \end{tabular}
      \caption{Lista Redattori} %INSERIRE DIDASCALIA - SE NECESSARIA -
      \label{tabredazione}
    \end{center}
  \end{table}  
 
\newpage
\section*{\LARGE Approvazione:}
\begin{table}[!h]
  \begin{center}
    \begin{tabular}
      {|c|c|}
      \hline
      %%%%%INTESTAZIONE COLONNE%%%%%%%%%%%%%%%%%%%%%%%%%%%%%%%
      \multicolumn{2}{|c|}{ \textbf{Approvazione} } \\
      \hline
      \textbf{Fase} & \textbf{Approvatori} \\
      %%%%%%%%%%%%%%FINE INTESTAZIONE COLONNE%%%%%%%%%%%%%%%%%%%%%%%%%%%%%%
      \hline
      %%%%%%%%%%% PARTE DA MODIFICARE %%%%%%%%%%%%%%%%%%%%%%%%%%%%%%%%%%%%%%    
      Pre-RR & Dal Bosco Davide\\
                  & \\
      \hline
      RR-RPP & Geremia Mirco \\
                  & \\
      \hline
      %%%%%%%%%%% FINE PARTE DA MODIFICARE %%%%%%%%%%%%%%%%%%%%%%%%%%%%%%%%%%%
    \end{tabular}
    \caption{Lista Approvatori} %INSERIRE DIDASCALIA - SE NECESSARIA -
    \label{tabapprovazione}
  \end{center}
\end{table}
 
 
\section*{\LARGE Lista di Distribuzione:}
 
  \begin{elenconumerato}{\normindent}
    \item WebShape
    \item I committenti Conte Renato e Vardanega Tullio in rappresentanza \\ dell'azienda proponente Zucchetti SPA
  \end{elenconumerato}
 
\newpage
 
 
 
\section*{\LARGE Registro delle Modifiche:}
 
 
\begin{center}
  \begin{table}[h]
     \begin{tabular*}
      {1\textwidth}%
        {@{\extracolsep{\fill}}|p{0.1\textwidth}|p{0.54\textwidth}|p{0.26\textwidth}|}
       \hline
%%%%%%%%%%%%%%INTESTAZIONE COLONNE%%%%%%%%%%%%%%%%%%%%%%%%%%%%%%%%%%%%%%%%%%%%%%%%%%%%%%%%%%%%%%
      \textbf{Versione} & \textbf{Descrizione} & \textbf{Autore} \\
%%%%%%%%%%%%%%FINE INTESTAZIONE COLONNE%%%%%%%%%%%%%%%%%%%%%%%%%%%%%%%%%%%%%%%%%%%%%%%%%%%%%%%%%%%%%%
     \hline
%%%%%%%%%%% PARTE DA MODIFICARE %%%%%%%%%%%%%%%%%%%%%%%%%%%%%%%%%%%%%%%%%%%%%%%%%%%%%%%%%%%%%%%%%
      1.1 &    09$\slash$12$\slash$2008 Aggiornamento in preparazione RPP & Carollo Mirko\\
      \hline
      1.0 &    09$\slash$12$\slash$2008 Verifica finale in preparazione al rilascio & Dissegna Stefano\\
      \hline
             0.3 & 7/12/2008 Correzioni in seguito a revisione & Dissegna Stefano \\
       \hline
            0.2 & 7/12/2008 Impaginazione e modifiche minori & Dissegna Stefano \\
       \hline
            0.1 & 23/11/2008 Piano di qualifica per la progettazione & Dissegna Stefano \\
            \hline
            0.0 & 22/11/2008 Bozza iniziale & Dissegna Stefano \\
             \hline
%%%%%%%%%%% FINE PARTE DA MODIFICARE %%%%%%%%%%%%%%%%%%%%%%%%%%%%%%%%%%%%%%%%%%%%%%%%%%%%%%%%%%%
    \end{tabular*}
  \caption{tabella delle modifiche} %INSERIRE DIDASCALIA - SE NECESSARIA -
  \label{tab:modifiche}
  \end{table}
\end{center}
 
 
\newpage
\thispagestyle{fancy}
\tableofcontents
\thispagestyle{fancy}
\newpage
 
\sezione{Introduzione}
 
\subsezione{Scopo del documento}
Il presente documento definisce il livello di qualit\`a atteso del prodotto e del processo di realizzazione dello stesso, indica le metodologie da adottare per assicurare la qualit\`a e riporta le attivit\`a svolte inerenti la gestione della qualit\`a.
 
\subsezione{Scopo del prodotto}
Il prodotto intende permettere la realizzazione di grafica vettoriale tramite una comoda interfaccia web.
 
\subsezione{Glossario}
Si veda il \textit{Glossario}.
 
\subsezione{Riferimenti normativi}
Il presente documento \`e redatto in accordo con le norme interne di WebShape, raccolte nel documento \textit{NormeDiProgetto} consegnato assieme a questo documento, e consultabile inoltre dal repository pubblico al quale WebShape si appoggia per i suoi progetti.
 
\sezione{Visione generale della strategia di verifica}
 
\subsezione{Organizzazione, pianificazione strategica e temporale, responsabilit\`a}
Le attivit\`a di verifica si dovranno svolgere immediatamente in seguito ad una modifica, o ad un insieme di modifiche. Qualora non sia possibile svolgere l'attivit\`a in modo automatico, onde evitare di attivare un'intero ciclo di verifica con gli annessi costi per delle modifiche di scarso rilievo, l'attivit\`a sar\`a svolta quando le differenze tra la versione attuale del prodotto e la versione su cui \`e stata applicata l'ultima verifica sono sufficientemente rilevanti. \`E compito di chi effettua le modifiche di segnalare ai verificatori la necessit\`a di un'attivit\`a di verifica. Il responsabile della qualifica dovr\`a accertarsi che il tutto si svolga correttamente e che segua quanto descritto nel presente documento.
\paragraph{Verifica dei documenti} Il revisore, attraverso un'attenta lettura del documento, deve controllare, in ordine crescente di importanza, che:
\begin{elenconumerato}[\textbf{}]{\subsubsecindent}
\item la grammatica e la sintassi siano corrette
\item la formattazione del documento sia corretta e segua gli standard contenuti nelle \textit{NormeDiProgetto}
\item i concetti vengano espressi in modo chiaro e non ambiguo
\item i concetti espressi siano corretti
\item il documento sia completo
\item il documento risolva il problema corretto
\end{elenconumerato}
\paragraph{Verifica del progetto architetturale}
Il progetto architetturale dovr\`a essere controllato secondo le seguenti caratteristiche:
\begin{elenconumerato}[\textbf{}]{\subsubsecindent}
\item Aderenza alle \textit{NormeDiProgetto}.
\item Completezza: per controllare l'aderenza ai requisiti, dovr\`a essere verificata la tracciabilit\`a tra ogni requisito e la parte, o le parti, che lo soddisfano nel progetto.
\item Sufficienza: ogni elemento del progetto dovr\`a, o soddissfare un requisito, o essere una dipendenza, diretta o indiretta, di un elemento necessario a soddisfare un requisito, in modo che nessuna parte del progetto risulti superflua.
\item Estendibilit\`a e manutenibilit\`a: il progetto dovr\`a prevedere la possibilit\`a che il prodotto possa essere facilmente esteso e mantenuto in futuro.
\item Efficienza: sebbene si ritenga che l'ottimizzazione prematura vada evitata il pi\`u possibile, considerata la necessit\`a da parte dell'applicazione di un server, la fase di progettazione dovr\`a assicurare che il server sia in grado di gestire un numero elevato di client contemporaneamente senza che l'utente noti problemi di performance.
\end{elenconumerato}
\paragraph{Verifica del progetto di dettaglio}
Il progetto di dettaglio dovr\`a soddisfare i seguenti aspetti:
\begin{elenconumerato}[\textbf{}]{\subsubsecindent}
\item Aderenza alle \textit{NormeDiProgetto}.
\item Completezza: ogni elemento presente nel progetto architetturale dovr\`a essere presente anche nel progetto di dettaglio. Non dovranno inoltre essere presenti delle lacune tra un elemento del progetto e la sua effettiva realizzazione.
\item Sufficienza: non dovranno essere presenti elementi non strettamente necessari alla realizzazione del progetto architetturale.
\item Applicabilit\`a: il progetto dovr\`a essere realizzabile dai programmatori usando le tecnologie previste.
\end{elenconumerato}
\paragraph{Verifica del codice}
La verifica del codice si compone dei seguenti aspetti:
\begin{elenconumerato}[\textbf{}]{\subsubsecindent}
\item Verifica automatica: verr\`a effettuata tramite test di unit\`a predisposti dai verificatori. I test dovranno coprire tutto il codice, a eccezione delle parti il cui funzionamento dipende da input esterni non simulabili in un test di unit\`a, come ad esempio il codice che gestisce gli eventi generati dal movimento del mouse, oppure da effetti secondari i cui risultati non siano facilmente confrontabili in modo automatico, ad esempio come appare l'aspetto grafico del programma sullo schermo dell'utente. I test dovranno accertare il corretto funzionamento delle unit\`a di codice anche con input limite. Per effettuare i test di unit\`a ci si avvarr\`a del software \textit{QUnit}. Per controllare che i test coprano il codice nella sua interezza (salvo le eccezioni gi\`a descritte) si utilizzer\`a \textit{JSCoverage} per controllare quali linee non vengono eseguite durante i test di unit\`a. Una pi\`u dettagliata descrizione di questi strumenti \`e disponibile nelle \textit{NormeDiProgetto}.
\item Prova del prodotto: sar\`a effettuata eseguendo l'applicazione e interagendo con essa, assicurandosi che corrisponda a quanto progettato.
\item Confronto con altri software gi\`a esistenti: requisito fondamentale del prodotto \`e la capacit\`a di generare immagini in formato \underline{SVG}. Per assicurarsi che i file generati siano corretti, si caricheranno su software la cui correttezza \`e nota (ad esempio \textit{Inkscape}) e ci si assicurer\`a che le immagini visualizzate corrispondano con quanto disegnato dall'utente.
\item Analisi statica del codice per controllare che le norme di codifica siano rispettate ed alla ricerca degli errori pi\`u comuni. L'editor JavaScript utilizzato, \textit{emacs} con \textit{js2-mode}, permette di catturare la maggior parte di questi errori mentre il codice viene digitato, mentre altri errori verranno ricercati tramite il programma \textit{JSLint}.
\item Analisi del codice tramite \textit{walkthrough}, alla ricerca di eventuali errori. Questa modalit\`a di verifica  verr\`a applicata su tutto il codice al termine di ogni iterazione.
\end{elenconumerato}
 
 
\subsezione{Risorse necessarie, risorse disponibili}
\paragraph{Documenti} Ogni documento sar\`a verificato da un revisore. Sarebbe auspicabile che a documenti particolarmente importanti, come ad esempio la definizione di prodotto, venissero assegnati due revisori, ma i costi eccessivi associati a una tale scelta non lo permettono.
\paragraph{Codice} Il numero di verificatori da assegnare ad un dato modulo di codice varieranno in base alle esigenze del modulo stesso.
 
\subsezione{Strumenti, tecniche, metodi}
Per il tracciamento degli errori e delle loro soluzioni ci si avvarr\`a degli strumenti di bug tracking messi a disposizione da \textit{SourceForge} e dagli strumenti di versionamento in uso. Per il debug del codice Javascript direttamente da browser si utilizzer\`a Firebug, la famosa estensione di Firefox. Le discussioni tra i verificatori e gli autori delle modifiche avverrano tramite il sistema di commenti di \textit{GitHub} qualora si tratti di annotazioni di scarsa entit\`a, mentre discussioni pi\`u rilevanti avverranno nel gruppo di discussione interno all'azienda. Entrambi i sistemi mantengono in modo persistente le informazioni, permettendo dei riferimenti futuri alle stesse. I verificatori saranno avvisati della necessit\`a di un'attivit\`a di verifica tramite un messaggio nel gruppo di discussione interno.

\subsezione{Infrastrutture e procedure di automazione} 
Hanno lo scopo di accelerare tutte le attivit\`a di verifica e impedire errori umani.
Tali attivit\`a potranno quindi essere svolte dopo ogni singolo cambiamento, anche di piccola entit\`a. Saranno predisposte dai verificatori. 
Verrano usati i seguenti prodotti: \textit{QUnit , JSCoverage, js2-mode, JSLint}. Tutti i dettagli delle attivit\`a di verifica sono gi\`a stati esposti nel paragrafo 2.1
 
\sezione{Gestione amministrativa della revisione}
 
\subsezione{Comunicazione e risoluzione di anomalie}
Quando un'anomalia viene riscontrata durante una verifica, il verificatore provveder\`a ad aggiungere un ticket al sistema di bug tracking e a segnalarlo, usando i metodi sopra esposti, a chi dovr\`a correggere il problema. La segnalazione dell'errore dovr\`a contenere una descrizione chiara del problema riscontrato. Qualora il correttore trovasse poco chiara o ambigua la descrizione, o fosse in disaccordo con la stessa, aprir\`a una discussione con il verificatore nel gruppo di discussione interno. A correzione avvenuta, questo sar\`a segnalato al verificatore tramite il gruppo di discussione. Il verificatore applicher\`a nuovamente il procedimento di verifica alla correzione. Se quest'ultimo passo va a buon fine, il correttore porr\`a come risolto il ticket nel sistema di bug tracking.
 
\subsezione{Trattamento delle discrepanze}
Qualora un prodotto si discostasse da quanto descritto nell'analisi dei requisiti, il responsabile della qualit\`a provveder\`a a individuare la causa e la persona, o le persone, imputabili per la discrepanza. Il responsabile della qualit\`a, dopo averne discusso con lui, o loro, traccier\`a delle azioni correttive e si assicurer\`a che vengano attuate.
 
\subsezione{Procedure di controllo di qualit\`a di processo}
L'amministratore del progetto deve assicurarsi che gli standard descritti nelle \textit{NormeDiProgetto} siano rispettati, e deve segnalare eventuali infrazioni al responsabile, in modo che provveda a risolverle. Il responsabile della qualit\`a controlla che le attivit\`a di verifica vengano svolte, e che seguano le procedure descritte in questo documento.
\newpage
 
\sezione{Resoconto delle attivit\`a di verifica}
 

 
\subsezione{Dettaglio delle verifiche tramite analisi}
\subsezione{Dettaglio delle verifiche tramite prove (test)}
\subsezione{Dettaglio dell'esito delle revisioni}
 
 
\sezione{Pianificazione ed esecuzione del collaudo}
 
\subsezione{Specifica della campagna di validazione (collaudo incluso)}
La campagna di validazione si svolger\`a in due fasi distinte:
\begin{elenconumerato}[\textbf{}]{\subsubsecindent}
\item \textit{Alpha test}: verr\`a eseguito internamente all'azienda. Il test consister\`a nell'esecuzione dell'applicativo per controllare che quanto descritto nell'analisi dei requisiti, la quale ha valore contrattuale, sia presente nel prodotto. Il test sar\`a guidato dagli \textit{use case} presenti nell'analisi dei requisiti.
\item \textit{Beta test}: il prodotto verr\`a distribuito al committente a fini di test.
\end{elenconumerato}
\subsezione{Dettaglio dell'esito della campagna di validazione}
 
\end{document}
 